\documentclass[../equacions-diferencials-II.tex]{subfiles}

\begin{document}
\chapter{Equacions en derivades parcials}
\section{Equacions en derivades parcials de primer ordre}
    \subsection{Equació general de primer ordre}
    \begin{definition}[Equació en derivades parcials de primer ordre]
        \labelname{equació en derivades parcials de primer ordre}\label{def:equacio-en-derivades-parcials-de-primer-ordre}
        Siguin \(\Omega\subseteq\mathbb{R}^{2n+1}\) un obert i \(F\colon\Omega\longrightarrow\mathbb{R}\) una funció.
        Aleshores, denotant
        \[
            u_{i}(x)=\frac{\partial u(x)}{\partial x_{i}},
        \]
        direm que l'expressió
        \[
            F(x_{1},\dots,x_{n},u(x),u_{1}(x),\dots,u_{n}(x))=0
        \]
        és una equació en derivades parcials de primer ordre sobre \(\Omega\).
    \end{definition}
    \begin{definition}[Solució d'una equació en derivades parcials]
        \labelname{solució d'una equació en derivades parcials}\label{def:solucio-duna-equacio-en-derivades-parcials}
        Siguin
        \[
            F(x_{1},\dots,x_{n},u(x),u_{1}(x),\dots,u_{n}(x))=0
        \]
        una equació en derivades parcials de primer ordre sobre un obert \(\Omega\) i \(\Phi(x)\colon\Omega\longrightarrow\mathbb{R}\) una funció de classe \(\mathcal{C}^{2}(\Omega)\) tal que
        \[
            F(x_{1},\dots,x_{n},\Phi(x),\Phi_{1}(x),\dots,\Phi_{n}(x))=0
        \]
        per a tot \(x\in\Omega\).
        Aleshores direm que \(\Phi\) és una solució de l'equació en derivades parcials.
    \end{definition}
    \subsection{Equacions quasi-lineals de primer ordre}
    \begin{definition}[Equació quasi-lineal]
        \labelname{equació diferencial quasi-lineal}\label{def:equacio-diferencial-quasi-lineal}
        Siguin \(\Omega\subseteq\mathbb{R}^{n}\times\mathbb{R}\) un obert, \(\{P_{i}\}_{i=1}^{N}\) una família de funcions de \(\Omega\) a \(\mathbb{R}\) i \(R\colon\Omega\longrightarrow\mathbb{R}\) una funció.
        Aleshores direm que l'expressió
        \[
            \sum_{i=1}^{N}\left(P_{i}(x,u(x))\frac{\partial u(x)}{\partial x_{i}}\right)=R(x,u(x))
        \]
        és una equació diferencial quasi-lineal de primer ordre.
    \end{definition}
    \begin{example}
        Exemples d'equacions quasi-lineals són, amb \(K>0\), l'equació de la calor
        \[
            \frac{\partial^{2} u}{\partial x^{2}}=K\frac{\partial u}{\partial t}
        \]
        o l'equació de la corda vibrant
        \[
            \frac{\partial^{2} u}{\partial x^{2}}=K\frac{\partial^{2} u}{\partial t^{2}},
        \]
        mentre que exemples d'equacions en derivades parcials que no siguin quasi-lineals són l'equació no lineal de la calor
        \[
            \frac{\partial^{2} u}{\partial x^{2}}+f(u)=K\frac{\partial u}{\partial t}
        \]
        o la llei de la conservació escalar
        \[
            \frac{\partial u(t)}{\partial t}+\divergencia(f(u(t)))=0.
        \]
    \end{example}
\end{document}
