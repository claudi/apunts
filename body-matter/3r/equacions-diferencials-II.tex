\documentclass[../../main.tex]{subfiles}

\begin{document}
\part{Equacions diferencials ordinàries \rom{2}}
\begin{comment}
%    \emph{\hypersetup{urlcolor=black}\href{https://www.urbandictionary.com/define.php?term=wtf}{wtf} is this assignatura}
\section{Òrbites d'una equació diferencial}
    \subsection{Sistemes autònoms a \ensuremath{\mathbb{R}^{n}}}
\section{Sistemes autònoms a \ensuremath{\mathbb{R}^{n}}}
%    \subsection{Interpretació geomètrica} % Justificació dels retrats de fase? Això a EDOS I
    \subsection{Estructura de les òrbites}
    \subsection{Integrals primeres}
    \subsection{Superfícies invariants}
    \subsection{Retrat de fase i conjugació}
\section{Sistemes integrables}
    \subsection{Sistemes potencials}
    \subsection{Sistemes Hamiltonians}
    \subsection{El model de Lotka-Volterra}
\section{Sistemes no integrables}
    \subsection{Teorema del flux tubular}
    \subsection{Anàlisi qualitativa dels punts d'equilibri}
    \subsection{Comportament límit de les òrbites}
    \subsection{Teorema de Poincaré-Bendixson}
    \subsection{Funcions de Liapunov}
    \subsection{Cicles límit}
    \subsection{Criteri de Bendixson-Dulac}
%    \subsection{Models a l'ecologia}
    \subsection{Sistema de van der Pol}
\chapter{Equacions en derivades parcials}
\section{Equacions en derivades parcials de primer ordre}
    \subsection{Introducció a les equacions en derivades parcials}
    \subsection{Equacions lineals i quasi-lineals de primer odre}
\section{Equacions en derivades parcials de segon ordre}
    \subsection{Equacions de la corda finita}
    \subsection{Principi d'Alembert}
    \subsection{Problemes de contorn}
    \subsection{L'equació de la calor}
    \subsection{Problema de la barra infinita}
    \subsection{Separació de variables i sèries de Fourier}
    \subsection{L'equació de Laplace}
\end{comment}
%    \subsection{Teorema del flux tubular}
%    \begin{theorem}[Teorema del flux tubular]
%        \labelname{Teorema del flux tubular}\label{thm:Teorema del flux tubular}
%        \begin{proof}
%            %TODO
%        \end{proof}
%    \end{theorem}
%    \subsection{Teoremes massa difícils pels nostres cervells (i crèdits)}
%    \begin{theorem}[Teorema de Hartman]
%        \labelname{Teorema de Hartman}\label{thm:Teorema de Hartman}
%    \end{theorem}
%    \begin{theorem}[Teorema de Perron-Hadamard]
%        \labelname{Teorema de Perron-Hadamard}\label{thm:Teorema de Perron-Hadamard}
%    \end{theorem}
\subfile{./equacions-diferencials-II/1-sistemes-autonoms-al-pla.tex}
\subfile{./equacions-diferencials-II/2-equacions-en-derivades-parcials.tex}
\end{document}
