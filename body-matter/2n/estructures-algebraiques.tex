\documentclass[../../main.tex]{subfiles}

\begin{document}
\part{Estructures algebraiques}
\subfile{./estructures-algebraiques/1-teoria-de-grups.tex}
\subfile{./estructures-algebraiques/2-teoria-danells.tex}
\subfile{./estructures-algebraiques/3-teoria-de-cossos-finits.tex}
\printbibliography
El capítol de teoria de grups està molt ben explicat en \cite{NumerosGruposyAnillos}, i la teoria de cossos finits està complementada amb \cite{AntoineRosaCampsMoncasiIntroduccioAlgebraAbstracta} sobre la teoria de classe.

La bibliografia del curs inclou els textos \cite{NumerosGruposyAnillos, AntoineRosaCampsMoncasiIntroduccioAlgebraAbstracta, CedoAlgebraBasica, CohnBasicAlgebra, FelixConcepcionSebastianIntroduccionAlAlgebra, FraleighFirstCourseAbstractAlgebra, HungerfordAlgebra}.
\end{document}

% Teorema d'òrbita-estabilitzador i lema de Burnside
