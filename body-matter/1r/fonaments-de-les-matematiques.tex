\documentclass[./primer.tex]{subfiles}

\begin{document}
\part{Fonaments de les matemàtiques}
\subfile{./fonaments-de-les-matematiques/1-logica-matematica.tex}
\subfile{./fonaments-de-les-matematiques/2-teoria-de-conjunts.tex}
\subfile{./fonaments-de-les-matematiques/3-conjunts-amb-operacions-i-nombres.tex}
\printbibliography
La secció sobre els axiomes de Peano està fortament inspirada en \cite{notesKumar}.
La resta de la teoria és una combinació de \cite{AntoineRosaCampsMoncasiIntroduccioAlgebraAbstracta} i \cite{TemesFonaments}, uns apunts de l'assignatura que sospito que només són accessibles des del campus virtual.
La part axiomàtica de la secció de teoria de conjunts està inspirada en \cite{ACTEAguade}, i la resta segueix \cite{TemesFonaments}.

La bibliografia del curs inclou els textos \cite{AntoineRosaCampsMoncasiIntroduccioAlgebraAbstracta, CastelletLlerenaAlgebraLinealIGeometria, GodementAlgebra, NutsAndBoltsOfProofs, BujalanceBujalanceCostaProblemasMatematicaDiscreta, IntroductionToMathematicalReasoning, ChapterZeroSchumacher}.
\end{document}

%AFEGIR paradoxa de Russell
