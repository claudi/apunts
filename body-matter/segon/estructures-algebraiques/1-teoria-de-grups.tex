\documentclass[../estructures-algebraiques.tex]{subfiles}

\begin{document}
\chapter{Teoria de grups}
\section{Grups}
    \subsection{Propietats bàsiques dels grups}
    \begin{definition}[Grup]
        \labelname{grup}
        \label{def:grup}
        Siguin~\(G\neq\emptyset\) un conjunt i~\(\ast\colon G\times G\to G\) una operació que satisfà
        \begin{enumerate}
            \item Per a tot~\(x,y,z\in G\)
            \[
                x\ast(y\ast z)=(x\ast y)\ast z.
            \]
            \item Existeix un~\(e\in G\) tal que per a tot~\(x\in G\)
            \[
                x\ast e=e\ast x=x.
            \]
            \item Per a cada~\(x\in G\) existeix~\(x'\) tal que
            \[
                x\ast x'=x'\ast x=e.
            \]
        \end{enumerate}
        Aleshores~\(G\) és un grup amb la l'operació~\(\ast\).
        També direm~\(\ast\) dota al conjunt~\(G\) d'estructura de grup.
    \end{definition}
    \begin{proposition}
        \label{prop:unicitat-neutre-del-grup}
        Siguin~\(G\) un grup amb l'operació~\(\ast\) i~\(e\in G\) tal que~\(x\ast e=e\ast x=x\) per a tot~\(x\in G\).
        Aleshores~\(e\) és únic.
    \end{proposition}
    \begin{proof}
        Suposem que existeix un altre element de~\(G\) amb aquesta propietat, diguem-ne~\(\hat{e}\in G\).
        Aleshores hauria de ser
        \[
            e\ast\hat{e}=e,
        \]
        però per hipòtesi
        \[
            e\ast\hat{e}=\hat{e}.
        \]
        Per tant, ha de ser~\(e=\hat{e}\).
    \end{proof}
    \begin{definition}[Element neutre d'un grup]
        \labelname{l'element neutre d'un grup}
        \label{def:lelement-neutre-del-grup}
        Siguin~\(G\) un grup amb l'operació~\(\ast\) i~\(e\) un element de~\(G\) tal que~\(x\ast e=e\ast x=x\) per a tot~\(x\in G\).
        Aleshores direm que~\(e\) és l'element neutre de~\(G\).

        Aquesta definició té sentit per la proposició \myref{prop:unicitat-neutre-del-grup}.
    \end{definition}
    \begin{notation}
        \label{notation:potencies-per-loperacio-repetida-en-un-grup}
        Donat un grup~\(G\) amb l'operació~\(\ast\) escriurem
        \[
            (x_{1}\ast x_{2})\ast x_{3}=x_{1}\ast x_{2}\ast x_{3}.
        \]

        També denotarem
        \[
            x^{n}=x\ast\overset{n)}{\dots}\ast x.
        \]

        Si denotem la conjugació del grup per~\(+\) usarem la notació additiva i escriurem
        \[
            x_{1}+\dots+x_{n}
        \]
        per referir-nos a la conjugació de~\(+\) amb si mateix~\(n\) vegades.

        També denotarem
        \[
            nx=x+\overset{n)}{\cdots}+ x.
        \]
    \end{notation}
    \begin{proposition}
        \label{prop:podem-tatxar-pels-costats-en-grups}
        Siguin~\(G\) un grup amb l'operació~\(\ast\) i~\(a,b,c\) tres elements de~\(G\).
        Aleshores
        \begin{enumerate}
            \item\label{enum:podem-tatxar-pels-costats-en-grups-1}~\(a\ast c=b\ast c\implica a=b\).
            \item\label{enum:podem-tatxar-pels-costats-en-grups-2}~\(c\ast a=c\ast b\implica a=b\).
        \end{enumerate}
    \end{proposition}
    \begin{proof}
        Farem només la demostració del punt \eqref{enum:podem-tatxar-pels-costats-en-grups-1} ja que l'altre és anàloga.

        Com que per hipòtesi~\(G\) és un grup, per la definició de \myref{def:grup} tenim que existeix~\(c'\) tal que~\(c\ast c'=e\), on~\(e\) és l'element neutre~\(G\), i tenim
        \[
            a\ast c\ast c'=b\ast c\ast c',
        \]
        el que significa que
        \[
            a\ast e=b\ast e,
        \]
        i ens queda~\(a=b\).
    \end{proof}
    \begin{proposition}
        \label{prop:unicitat-inversa-en-grups}
        Siguin~\(G\) un grup amb l'operació~\(\ast\) i element neutre~\(e\) i~\(a\) un element de~\(G\).
        Aleshores existeix un únic~\(a'\in G\) tal que
        \[
            a\ast a'=a'\ast a=e.
        \]
    \end{proposition}
    \begin{proof}
        Notem que existeix un~\(a'\in G\) que satisfà l'equació per la definició de \myref{def:grup}, i per tant la proposició té sentit.

        Suposem doncs que existeix~\(a''\in G\) tal que
        \[
            a\ast a''=a''\ast a=e.
        \]
        Però aleshores tenim
        \[
            a\ast a''=e=a\ast a',
        \]
        i per la proposició \myref{prop:podem-tatxar-pels-costats-en-grups} ha de ser~\(a'=a''\), com volíem demostrar.
    \end{proof}
    \begin{definition}[Invers d'un element]
        \labelname{l'invers d'un element d'un grup}
        \label{def:linvers-dun-element-dun-grup}
        Siguin~\(G\) un grup amb l'operació~\(\ast\) i element neutre~\(e\) i~\(a\) un element de~\(G\).
        Per la definició de grup tenim que existeix un~\(a'\in G\) tal que
        \[
            a\ast a'=a'\ast a=e.
        \]
        Aleshores direm que~\(a'\) és l'invers de~\(a\) en~\(G\), i el denotarem per~\(a^{-1}\).

        Aquesta definició té sentit per la proposició \myref{prop:unicitat-inversa-en-grups} i la notació introduïda en \myref{notation:potencies-per-loperacio-repetida-en-un-grup}.
    \end{definition}
    \begin{proposition}
        \label{prop:linvers-de-lelement-neutre-dun-grup-es-ell-mateix}
        Sigui~\(G\) un grup amb l'operació~\(\ast\) i element neutre~\(e\).
        Aleshores
        \[
            e^{-1}=e.
        \]
    \end{proposition}
    \begin{proof}
        Per la definició de \myref{def:grup} tenim que
        \[
            e\ast e^{-1}=e^{-1}\ast e=e,
        \]
        i per ha de ser~\(e^{-1}=e\).
    \end{proof}
    \begin{proposition}
        \label{prop:grups:linvers-de-linvers-dun-element-es-lelement}
        Siguin~\(G\) un grup amb l'operació~\(\ast\) i~\(a\) un element de~\(G\).
        Aleshores
        \[
            \left(a^{-1}\right)^{-1}=a.
        \]
    \end{proposition}
    \begin{proof}
        Sigui~\(e\) l'element neutre de~\(G\).
        Com que~\(\left(a^{-1}\right)^{-1}\) és l'invers de~\(a^{-1}\) tenim
        \[
            \left(a^{-1}\right)^{-1}\ast a^{-1}=e
        \]
        però també tenim que
        \[
            a\ast a^{-1}=e.
        \]
        Per tant és
        \[
            a\ast a^{-1}=(a^{-1})^{-1}\ast a^{-1},
        \]
        i per la proposició \myref{prop:podem-tatxar-pels-costats-en-grups} ha de ser
        \[
            a=\left(a^{-1}\right)^{-1}.\qedhere
        \]
    \end{proof}
    \begin{proposition}
        \label{prop:invers-de-a-b-b-invers-a-invers}
        Siguin~\(G\) un grup amb l'operació~\(\ast\) i~\(a,b\) dos elements de~\(G\).
        Aleshores
        \[
            (a\ast b)^{-1}=b^{-1}\ast a^{-1}.
        \]
    \end{proposition}
    \begin{proof}
        Sigui~\(e\) l'element neutre de~\(G\).
        Considerem
        \begin{align*}
        (b^{-1}\ast a^{-1})\ast(a\ast b)&=b^{-1}\ast a^{-1}\ast a\ast b\\
        &=b^{-1}\ast e\ast b\\
        &=b^{-1}\ast b=e.
        \end{align*}
        i de manera anàloga trobem
        \[
            (a\ast b)\ast(b^{-1}\ast a^{-1})=e.
        \]
        Així doncs, per la proposició \myref{prop:unicitat-inversa-en-grups} tenim que~\(a\ast b\) és la inversa de~\(b^{-1}\ast a^{-1}\), és a dir
        \[
            (a\ast b)^{-1}=b^{-1}\ast a^{-1}.\qedhere
        \]
    \end{proof}
    \begin{lemma}
        \label{lema:solucions-uniques-en-grups-a-equacions}
        Siguin~\(G\) un grup amb l'operació~\(\ast\) i~\(a,b\) dos elements de~\(G\).
        Aleshores existeixen~\(x,y\in G\) únics tals que
        \[
            b\ast x=a\quad\text{i}\quad y\ast b=a.
        \]
    \end{lemma}
    \begin{proof}
        Fem només una de les demostracions, ja que l'altre és anàloga.
        Com que per hipòtesi~\(G\) és un grup, per la definició de \myref{def:grup} tenim que existeix~\(b^{-1}\in G\) tal que~\(b^{-1}\ast b=e\), on~\(e\) és l'element neutre de~\(G\).
        Per tant considerem
        \[
            b^{-1}\ast(b\ast x)=b^{-1}\ast a
        \]
        i per la definició de \myref{def:grup} tenim que això és equivalent a
        \[
            (b^{-1}\ast b)\ast x=b^{-1}\ast a,
        \]
        i de nou per la definició de grup, i per la definició de \myref{def:lelement-neutre-del-grup},
        \[
            e\ast x=x=b^{-1}\ast a ;
        \]
        i la unicitat ve donada per la proposició \myref{prop:unicitat-neutre-del-grup}.
    \end{proof}
    \begin{theorem}
        \label{thm:definicions-alternatives-de-grup}
        Siguin~\(G\) un conjunt i~\(\ast\colon G\times G\to G\) una operació binària que satisfà
        \(x\ast(y\ast z)=(x\ast y)\ast z\) per a tot~\(x,y,z\in G\).
        Aleshores els següents enunciats són equivalents:
        \begin{enumerate}
            \item\label{enum:definicions-alternatives-de-grup-1}~\(G\) és un grup amb l'operació~\(\ast\).
            \item\label{enum:definicions-alternatives-de-grup-2}~\(G\neq\emptyset\) i per a tot~\(a,b\in G\) existeix uns únics~\(x,y\in G\) tals que
            \[
                b\ast x=a\quad\text{i}\quad y\ast b=a.
            \]
            \item\label{enum:definicions-alternatives-de-grup-3} Existeix~\(e\in G\) tal que per a tot~\(x\in G\) tenim~\(x\ast e=x\) i existeix un~\(x^{-1}\in G\) tal que~\(x\ast x^{-1}=e\).
        \end{enumerate}
    \end{theorem}
    \begin{proof}
        Comencem demostrant \eqref{enum:definicions-alternatives-de-grup-1}\(\implica\)\eqref{enum:definicions-alternatives-de-grup-2}.
        Suposem que~\(G\) és un grup amb l'operació.
        Veiem que~\(G\) no és buit per la definició de \myref{def:grup}, i la segona part és el lema \myref{lema:solucions-uniques-en-grups-a-equacions}.

        Demostrem ara \eqref{enum:definicions-alternatives-de-grup-2}\(\implica\)\eqref{enum:definicions-alternatives-de-grup-3}.
        La primera part es pot veure fixant~\(x\in G\).
        Pel punt \eqref{enum:definicions-alternatives-de-grup-2} tenim que per a cada~\(a\in G\) existeix un únic~\(b\in G\) tal que
        \[
            a\ast b=x,
        \]
        i podem fer
        \[
            a\ast b\ast e=x\ast e
        \]
        i substituint ens queda
        \[
            x\ast e=x.
        \]
        Per veure la segona part notem que pel punt \eqref{enum:definicions-alternatives-de-grup-2} tenim que per a tot~\(x\in G\) existeix un~\(a\in G\) tal que
        \[
            x\ast a=e,
        \]
        i aleshores~\(a=x^{-1}\).

        Ara només ens queda veure \eqref{enum:definicions-alternatives-de-grup-3}\(\implica\)\eqref{enum:definicions-alternatives-de-grup-1}.
        Tenim que per a tot~\(x\in G\) existeix un~\(x^{-1}\) tal que~\(x\ast x^{-1}=e\), i de la mateixa manera, existeix un~\(y\in G\) tal que~\(x^{-1}\ast y=e\).
        Per tant
        \begin{align*}
        e&=x^{-1}\ast y\\
        &=x^{-1}\ast e\ast y\\
        &=x^{-1}\ast x\ast x^{-1}\ast y\\
        &=x^{-1}\ast x\ast e=x^{-1}\ast x.
        \end{align*}
        Així tenim que per a tot~\(x\in G\) es compleix~\(x\ast x^{-1}=x^{-1}\ast x=e\), d'on podem veure que~\(e\ast x=x\ast e\), i com que, per hipòtesi, l'operació~\(\ast\) satisfà~\(x\ast(y\ast z)=(x\ast y)\ast z\) per a tot~\(x,y,z\in G\) es compleix la definició de \myref{def:grup} i tenim que~\(G\) és un grup amb l'operació~\(\ast\).

        Així tenim \eqref{enum:definicions-alternatives-de-grup-1}\(\implica\)\eqref{enum:definicions-alternatives-de-grup-2}\(\implica\)\eqref{enum:definicions-alternatives-de-grup-3}\(\implica\)\eqref{enum:definicions-alternatives-de-grup-1}, com volíem veure.
    \end{proof}
    \subsection{Subgrups i subgrups normals}
    \begin{definition}[Subgrup]
        \labelname{subgrup}
        \label{def:subgrup}
        Siguin~\(G\) un grup amb l'operació~\(\ast\) i~\(H\subseteq G\) un subconjunt de~\(G\) tal que~\(H\) sigui un grup amb l'operació~\(\ast\).
        Aleshores diem que~\(H\) és un subgrup de~\(G\).

        També ho denotarem com~\(H\leq G\).
    \end{definition}
    \begin{observation}
        \label{obs:lelement-neutre-dun-grup-pertany-als-subgrups}
        \(e\in H\).
    \end{observation}
    \begin{proposition}
        \label{prop:condicio-equivalent-a-subgrup}
        Siguin~\(G\) un grup amb l'operació~\(\ast\) i~\(H\) un subconjunt de~\(G\).
        Aleshores~\(H\) és un subgrup de~\(G\) si i només si per a tot~\(x,y\in H\) tenim que~\(x\ast y^{-1}\in H\).
    \end{proposition}
    \begin{proof}
        Sigui~\(e\) l'element neutre de~\(G\).
        Demostrem primer que la condició és necessària (\(\implica\)).
        Això ho podem veure per la definició de \myref{def:grup}, ja que tenim que~\(y^{-1}\) existeix i pertany a~\(H\), i per tant~\(x\ast y^{-1}\) també pertany a~\(H\).

        Demostrem ara que la condició és suficient (\(\implicatper\)).
        Tenim que per a tot~\(x\in H\) es compleix
        \[
            x\ast x^{-1}=e,
        \]
        i per tant~\(e\in H\).
        També tenim que per a tot~\(x\in H\) es compleix
        \[
            e\ast x^{-1}=x^{-1},
        \]
        i per tant~\(x^{-1}\in H\).

        Ara només ens queda veure que~\(\ast\) és tancat en~\(H\); és a dir, que per a tot~\(x,y\in H\) tenim~\(x\ast y\in H\).
        Com que ja hem vist que~\(y^{-1}\) existeix i pertany a~\(H\), per la proposició \myref{prop:grups:linvers-de-linvers-dun-element-es-lelement} tenim
        \[
            x\ast y^{-1}=x\ast y
        \]
        i per hipòtesi~\(x\ast y\in H\).

        Per tant, per la definició de \myref{def:grup} tenim que~\(H\) és un grup amb l'operació~\(\ast\), i com que per hipòtesi~\(H\subseteq G\) per la definició de \myref{def:subgrup} tenim que~\(H\) és un subgrup de~\(G\).
    \end{proof}
    \begin{proposition}
        \label{prop:interseccio-de-subgrups-es-subgrup}
        Siguin~\(G\) un grup,~\(\{H_{i}\}_{i\in I}\) una família de subgrups de~\(G\) i~\(H=\bigcap_{i\in I}H_{i}\).
        Aleshores~\(H\) és un subgrup de~\(G\).
    \end{proposition}
    \begin{proof}
        Ho demostrarem amb la proposició \myref{prop:condicio-equivalent-a-subgrup}.
        Siguin~\(\ast\) l'operació de~\(G\) i~\(e\) l'element neutre de~\(G\).
        Tenim~\(H\subseteq G\) i~\(H\neq\emptyset\), ja que per l'observació \myref{obs:lelement-neutre-dun-grup-pertany-als-subgrups} tenim que~\(e\in H\).
        Comprovem ara que per a tot~\(x,y\in H\) tenim~\(x\ast y^{-1}\in H\).
        Tenim que si~\(x,y\in H\), per la definició de~\(H\),~\(x,y\in H_{i\in I}\); i com que~\(H_{i\in I}\) és un subgrup de~\(G\),~\(x\ast y^{-1}\in H_{i\in I}\) per la proposició \myref{prop:condicio-equivalent-a-subgrup}, i per tant~\(x\ast y^{-1}\in H\), com volíem veure.
    \end{proof}
    \begin{proposition}
        \label{prop:els-subgrups-generats-per-un-conjunt-existeixen}
        Siguin~\(G\) un grup,~\(S\neq\emptyset\) un subconjunt de~\(G\),
        \[
            \{H_{i}\}_{i\in I}=\{H\subseteq G\mid S\leq H\}
        \]
        una família de subgrups de~\(G\) i
        \[
            H=\bigcap_{i\in I}H_{i}.
        \]
        Aleshores~\(H\) i és un subgrup de~\(G\).
    \end{proposition}
    \begin{proof}
        Sigui~\(e\) l'element neutre de~\(G\).
        Comprovem que~\(H\neq\emptyset\).
        En tenim prou amb veure que~\(e\in H\).
        Per veure que~\(H\) és un subgrup de~\(G\) ho podem fer per la proposició \myref{prop:interseccio-de-subgrups-es-subgrup}.
    \end{proof}
    \begin{definition}[Mínim subgrup generat per un conjunt] %vspace
        \labelname{mínim subgrup generat per un conjunt}
        \label{def:minim-subgrup-generat-per-un-subconjunt}
        Siguin~\(G\) un grup,~\(S\) un subconjunt de~\(G\),
        \[
            \{H_{i}\}_{i\in I}=\{H\subseteq G\mid S\leq H\}
        \]
        una família de subgrups de~\(G\) i
        \[
            H=\bigcap_{i\in I}H_{i}.
        \]
        Aleshores direm que el subgrup~\(H\leq G\) és el mínim subgrup generat per~\(S\) i ho denotarem amb~\(\langle S\rangle\).

        Aquesta definició té sentit per la proposició \myref{prop:els-subgrups-generats-per-un-conjunt-existeixen}.
    \end{definition}
    \begin{proposition}
        \label{prop:forma-grups-ciclics}
        Siguin~\(G\) un grup i~\(g\) un element de~\(G\).
        Aleshores
        \[
            \langle\{g\}\rangle=\{g^{i}\}_{i\in\mathbb{Z}}.
        \]
    \end{proposition}
    \begin{proof}
        Sigui~\(\ast\) l'operació de~\(G\).
        Ho demostrem per doble inclusió.

        Comencem veient que
        \[
            \{g^{i}\}_{i\in\mathbb{Z}}\subseteq\langle\{g\}\rangle.
        \]
        Per la definició de \myref{def:minim-subgrup-generat-per-un-subconjunt} tenim que existeix una família de subconjunts de~\(G\) que denotarem per~\(\{H_{i}\}_{i\in I}\), amb
        \[
            \{g\}\subseteq H=\bigcap_{i\in I}H_{i}.
        \]
        Com que~\(\{H_{i}\}_{i\in I}\) són subgrups de~\(G\) tenim que, donat que~\(g\in H_{i}\),~\(g^{n}\in H_{i}\) per a tot~\(i\in I\) i tot~\(n\in\mathbb{Z}\) per la definició de \myref{def:grup}, i per tant~\(g^{n}\in H\), el que és equivalent a dir que~\(g^{n}\in \langle g\rangle\) per a tot~\(n\in\mathbb{Z}\)

        Ara veiem que
        \[
            \langle\{g\}\rangle\subseteq\{g^{i}\}_{i\in\mathbb{Z}}.
        \]
        Denotarem~\(H_{g}=\{g^{i}\}_{i\in\mathbb{Z}}\).
        Hem de veure que~\(H_{g}\) és un grup amb l'operació~\(\ast\).

        Observem que per a tot~\(g^{i},g^{j}\in H_{g}\) es satisfà~\(g^{i}\ast g^{-j}=g^{i-j}\in H_{g}\), i per tant~\(H_{g}\leq G\).
        Ara bé, com que~\(\{g\}\subseteq H_{g}\), tenim que~\(H_{g}\in\{H_{i}\}_{i\in I}\), és a dir, que~\(H_{g}\) pertany a la família de subconjunts de~\(G\) que contenen~\(\{g\}\); el que significa que~\(\langle\{g\}\rangle\leq H_{g}\), i per tant \[\langle\{g\}\rangle\subseteq\{g^{i}\}_{i\in\mathbb{Z}}.\qedhere\]
    \end{proof}
    \begin{definition}[Ordre d'un grup]
        \labelname{ordre d'un grup}
        \label{def:ordre-dun-grup}
        Sigui~\(G\) un grup.
        Direm que~\(\abs{G}\) és l'ordre del grup.
        Si~\(\abs{G}\) és finit direm que~\(G\) és un grup d'ordre finit, i si~\(\abs{G}\) no és finit direm que~\(G\) és un grup d'ordre infinit.
    \end{definition}
    \begin{proposition}
        \label{prop:potencia-element-neutre-en-un-grup}
        Siguin~\(G\) un grup amb element neutre~\(e\) i~\(g\) un element de~\(G\).
        Aleshores
        \[
            \abs{\langle\{g\}\rangle}=n\sii n=\min\{k\in\mathbb{N}\mid g^{k}=e\}.
        \]
    \end{proposition}
    \begin{proof}
        Sigui~\(\ast\) l'operació de~\(G\).
        Comencem amb la implicació cap a l'esquerra (\(\implicatper\)).
        Suposem doncs que~\(n=\min\{k\in\mathbb{N}\mid g^{k}=e\}\).
        Pel \myref{thm:divisio-euclidiana} tenim que per a tot~\(t\in\mathbb{Z}\) existeixen uns únics~\(Q\in\mathbb{Z}\),~\(r\in\mathbb{N}\), amb~\(r<n\) tals que~\(t=Qn+r\).
        Per tant
        \begin{align*}
        g^{t}&=g^{Qm+r}\\
        &=g^{Qm}\ast g^{r}\\
        &=(g^{m})^{Q}\ast g^{r}\\
        &=e^{Q}\ast g^{r}=g^{r}.
        \end{align*}
        Per tant, com que~\(0\leq r<n\),~\(\abs{\langle\{g\}\rangle}=n\).

        Fem ara la implicació cap a la dreta (\(\implica\)).
        Suposem doncs que~\(\abs{\langle\{g\}\rangle}=n\).
        Com que el grup és finit per a cada~\(i\in\mathbb{Z}\) existeix~\(j\in\mathbb{Z}\) tal que~\(g^{i}=g^{j}\) i, com que~\(\langle\{g\}\rangle\) és un grup, per la definició de \myref{def:grup} existeix~\(g^{-j}\in\langle\{g\}\rangle\) tal que~\(g^{i-j}=e\).

        Sigui doncs~\(t\in\mathbb{N}\) tal que~\(g^{t}=e\).
        Aleshores, pel \myref{thm:divisio-euclidiana} existeixen uns únics~\(Q,r\in\mathbb{N}\), amb~\(r<n\) tals que~\(t=Qn+r\).
        Per tant
        \begin{align*}
        g^{t}&=g^{Qn+r}\\
        &=g^{Qn}\ast g^{r}\\
        &=(g^{n})^{Q}\ast g^{r}\\
        &=e^{Q}\ast g^{r}\\
        &=g^{r}=e.
        \end{align*}
        i per tant~\(r=0\), i tenim~\(t=Qn\), i per tant~\(n=\min\{k\in\mathbb{N}\mid g^{k}=e\}\).
    \end{proof}
    \begin{definition}[Conjugació entre conjunts sobre grups]
        \labelname{conjugació entre conjunts sobre grups}
        \label{def:conjugacio-entre-conjunts-sobre-grups}
        Siguin~\(G\) un grup amb l'operació~\(\ast\) i~\(H\) un subconjunt de~\(G\).
        Aleshores definim
        \[
            GH=\{g\ast h\mid g\in G,h\in H\}\quad\text{i}\quad HG=\{h\ast g\mid g\in G,h\in H\}.
        \]
    \end{definition}
    \begin{definition}[Subgrup normal]
        \labelname{subgrup normal}
        \label{def:subgrup-normal}
        Siguin~\(G\) un grup i~\(H\) un subgrup de~\(G\).
        Aleshores direm que~\(H\) és un subgrup normal de~\(G\) si per a tot~\(x\in G\) tenim
        \[
            \{x\}H=H\{x\}.
        \]
        Ho denotarem com~\(H\trianglelefteq G\).
    \end{definition}
    \begin{proposition}
        \label{prop:condicions-equivalents-a-subgrup-normal}
        Siguin~\(G\) un grup i~\(H\) un subgrup de~\(G\).
        Aleshores són equivalents
        \begin{enumerate}
            \item\label{enum:condicions-equivalents-a-subgrup-normal-1}~\(\{x\}H=H\{x\}\) per a tot~\(x\in G\).
            \item\label{enum:condicions-equivalents-a-subgrup-normal-2}~\(\{x^{-1}\}H\{x\}=H\) per a tot~\(x\in G\).
            \item\label{enum:condicions-equivalents-a-subgrup-normal-3}~\(\{x^{-1}\}H\{x\}\subseteq H\) per a tot~\(x\in G\).
        \end{enumerate}
    \end{proposition}
    \begin{proof}
        Sigui~\(\ast\) l'operació de~\(G\).
        Comencem demostrant \eqref{enum:condicions-equivalents-a-subgrup-normal-1}\(\implica\)\eqref{enum:condicions-equivalents-a-subgrup-normal-2}.
        Suposem que~\(H\) és un subgrup normal de~\(G\), per la definició de \myref{def:subgrup-normal} tenim~\(\{x\}H=H\{x\}\) per a tot~\(x\in G\).
        Aleshores tenim
        \begin{align*}
        \{x\}H\{x^{-1}\}&=H\{x\}\{x^{-1}\}\\
        &=\{h\ast x\ast x^{-1}\mid h\in H\}\\
        &=\{h\mid h\in H\}=H.
        \end{align*}

        Continuem demostrant \eqref{enum:condicions-equivalents-a-subgrup-normal-2}\(\implica\)\eqref{enum:condicions-equivalents-a-subgrup-normal-3}.
        Suposem que~\(\{x\}H\{x^{-1}\}=H\).
        Tenim que~\(\{x\}H\{x^{-1}\}=H\subseteq H\).

        Demostrem ara \eqref{enum:condicions-equivalents-a-subgrup-normal-3}\(\implica\)\eqref{enum:condicions-equivalents-a-subgrup-normal-1}.
        Suposem doncs que~\(\{x^{-1}\}H\{x\}\subseteq H\) per a tot~\(x\in G\).
        Això significa que per a tot~\(h\in H\) existeix un~\(h'\in H\) tal que~\(x\ast h\ast x^{-1}=h'\), i aleshores, per la definició de \myref{def:grup},~\(x\ast h=h'\ast x\in H\), i per tant~\(x\ast h\in H\{x\}\) per a tot~\(x\in G\).
        Així hem vist que~\(\{x\}H\subseteq H\{x\}\).
        Per veure l'altre inclusió es pot donar un argument anàleg, i per tant~\(\{x\}H= H\{x\}\).
        %REFERENCIA I POTSER MENYS MANDRA

        I així hem vist que \eqref{enum:condicions-equivalents-a-subgrup-normal-1}\(\implica\)\eqref{enum:condicions-equivalents-a-subgrup-normal-2}\(\implica\)\eqref{enum:condicions-equivalents-a-subgrup-normal-3}\(\implica\)\eqref{enum:condicions-equivalents-a-subgrup-normal-1} i hem acabat.
    \end{proof}
    \subsection{Grups cíclics i grups abelians}
    \begin{definition}[Grup abelià]
        \labelname{grup abelià}
        \label{def:grup-abelia}
        Sigui~\(G\) un grup amb l'operació~\(+\) tal que per a tot~\(x,y\in G\) satisfà
        \[
            x+y=y+x.
        \]
        Aleshores direm que~\(G\) és un grup abelià.
    \end{definition}
    \begin{proposition}
        \label{prop:condicio-equivalent-a-grup-abelia}
        Sigui~\(G\) un grup amb l'operació~\(+\).
        Aleshores~\(G\) és un grup abelià si i només si per a tot~\(a,b\in G\) es compleix
        \[
            -(a+b)=-a-b.
        \]
    \end{proposition}
    \begin{proof}
        Que la condició és necessària (\(\implica\)) ho podem veure amb la definició de \myref{def:grup-abelia} i la proposició \myref{prop:invers-de-a-b-b-invers-a-invers}.

        Demostrem ara que la condició és suficient (\(\implicatper\)).
        Diem que l'element neutre de~\(G\) és~\(e\).
        Per la definició de \myref{def:grup} tenim que
        \begin{equation}
        \label{hipotesi:inverses-grups}
        (a+b)-(a+b)=e,
        \end{equation}
        que és equivalent a
        \[
            (a+b)-(a+b)-\left(-(a+b)\right)=-\left(-(a+b)\right),
        \]
        i aleshores
        \begin{align*}
        a+b&=-\left(-(a+b)\right)\\
        &=-\left(-b-a\right)\tag{Proposició \myref{prop:invers-de-a-b-b-invers-a-invers}}\\
        &=-\left(-(b+a)\right)\tag{Hipótesi \eqref{hipotesi:inverses-grups}}\\
        &=b+a,\tag{Proposició \myref{prop:grups:linvers-de-linvers-dun-element-es-lelement}}
        \end{align*}
        i per la definició de \myref{def:grup-abelia},~\(G\) és un grup abelià.
    \end{proof}
    \begin{definition}[Grup cíclic]
        \labelname{grup cíclic}
        \label{def:grup-ciclic}
        Siguin~\(G\) un grup amb l'operació~\(\ast\) i~\(g\) un element de~\(G\).
        Aleshores diem que el grup~\(\langle\{g\}\rangle\) amb l'operació~\(\ast\) és un grup cíclic i que~\(g\) és un generador del grup.
    \end{definition}
    \begin{proposition}
        Sigui~\(G\) un grup cíclic.
        Aleshores~\(G\) és un grup abelià.
    \end{proposition}
    \begin{proof}
        Sigui~\(\ast\) l'operació de~\(G\).
        Com que~\(G\) és un grup cíclic, per la definició de \myref{def:grup-ciclic} tenim que existeix un~\(g\) tal que~\(\langle\{g\}\rangle=G\).
        Siguin~\(a,b\) dos  elements de~\(G\), i com que~\(G\) és un grup cíclic, ha de ser~\(a=g^{m}\) i~\(b=g^{n}\) per a certs~\(m,n\in\mathbb{N}\).
        Ara bé, tenim que~\(g^{m}\ast g^{n}=g^{m}\ast g^{n}\), ja que
        \begin{align*}
        g^{n}\ast g^{m}&=g^{n+m}\\
        &=g^{m+n}=g^{m}\ast g^{n}
        \end{align*}
        i aleshores~\(a\ast b=b\ast a\), i per la definició de \myref{def:grup-abelia} hem acabat.
    \end{proof}
    \begin{proposition}
        Siguin~\(G\) un grup cíclic i~\(H\) un subgrup de~\(G\).
        Aleshores~\(H\) és un grup cíclic.
    \end{proposition}
    \begin{proof}
        Siguin~\(\ast\) l'operació de~\(G\) i~\(e\) l'element neutre de~\(G\).
        Per l'observació \myref{obs:lelement-neutre-dun-grup-pertany-als-subgrups} tenim que~\(e\in H\).
        Si~\(H=\{e\}\) no hi ha res a demostrar.
        Suposem que~\(H\neq\{e\}\), aleshores existeix un~\(g\in G\) tal que~\(g^{n}\in H\) per a cert~\(n\in\mathbb{N}\).
        Sigui doncs~\(m\) l'enter més petit tal que~\(g^{m}\in H\); volem demostrar que~\(H=\langle\{g^{m}\}\rangle\).

        Sigui~\(a\) un element de~\(H\).
        Aleshores com que~\(a\in H\subseteq G\),~\(a=g^{t}\) per a cert~\(t\in\mathbb{N}\), i pel \myref{thm:divisio-euclidiana} existeixen~\(Q,r\in\mathbb{N}\), amb~\(r<m\) tals que~\(t=Qm+r\), i per tant~\(g^{t}=g^{Qm+r}\).
        Aleshores tenim
        \[
            g^{r}=\left(g^{m}\right)^{-Q}\ast g^{t},
        \]
        i ha de ser~\(g^{r}\in H\), ja que~\(g^{m}\in H\), i per la definició de \myref{def:grup} tenim que~\(\left(g^{m}\right)^{-1}\in H\).
        Per tant~\(g^{r}\in H\).
        Però ara bé,~\(m\) era el mínim enter tal que~\(g^{m}\in H\), i~\(r<m\), per tant ha de ser~\(g^{r}=e\), és a dir,~\(r=0\) i per tant~\(t=Qm\); el que significa que~\(H=\{\left(g^{m}\right)^{Q}\mid Q\in\mathbb{N}\}\) i per la proposició \myref{prop:forma-grups-ciclics}~\(H=\langle\{g^{m}\}\rangle\) i hem acabat.
    \end{proof}
    \begin{proposition}
        \label{prop:existencia-i-unicitat-de-subgrups-ciclics-dordres-divisors}
        Siguin~\(G\) un grup cíclic d'ordre~\(n\) finit i~\(d\in\mathbb{N}\) un divisor de~\(n\).
        Aleshores existeix un únic subgrup de~\(G\) d'ordre~\(d\).
    \end{proposition}
    \begin{proof}
        %TODO
    \end{proof}
    \subsection{Grup quocient}
    \begin{proposition}
        \label{prop:relacio-entre-grups-es-dequivalencia}
        \label{TODO:grup-quocient}
        Siguin~\(G\) un grup amb l'operació~\(\ast\) i~\(H\) un subgrup de~\(G\).
        Aleshores la relació
        \[
            x\sim y\sii x\ast y^{-1}\in H\text{ per a tot }x,y\in G
        \]
        és una relació d'equivalència.
    \end{proposition}
    \begin{proof}
        Sigui~\(e\) l'element neutre del grup~\(G\).
        Comprovem les propietats de la definició de relació d'equivalència:
        \begin{enumerate}
            \item Reflexiva: Sigui~\(x\in G\).
            Aleshores~\(x\ast x^{-1}=e\), i tenim l'observació \myref{obs:lelement-neutre-dun-grup-pertany-als-subgrups}.
            \item Simètrica: Siguin~\(x,y\in G\) i suposem que~\(x\sim y\), això significa que~\(x\ast y^{-1}\in H\), i per la definició de \myref{def:grup} tenim que~\((x\ast y^{-1})^{-1}\in H\), ja que per hipòtesi~\(H\) és un grup, i per les proposicions \myref{prop:invers-de-a-b-b-invers-a-invers} i \myref{prop:grups:linvers-de-linvers-dun-element-es-lelement} tenim que
            \[
                (x\ast y^{-1})^{-1}=y\ast x^{-1},
            \]
            i això és~\(y\sim x\).
            \item Transitiva: Siguin~\(x,y,z\in G\) i suposem que~\(x\sim y\) i~\(y\sim z\).
            Per tant~\(x\ast y^{-1}\in H\), i~\(y\ast z^{-1}\in H\).
            Com que per hipòtesi~\(H\) és un grup, tenim que
            \[
                (x\ast y^{-1})\ast(y\ast z^{-1})\in H,
            \]
            que és equivalent a~\(x\ast z^{-1}\in H\), i per tant~\(x\sim z\).
        \end{enumerate}
        I per la definició de \myref{def:relacio-dequivalencia} hem acabat.
    \end{proof}
    \begin{proposition}\label{prop:grup-quocient}
        Siguin~\(G\) un grup amb l'operació~\(\ast\) i~\(H\) un subgrup de~\(G\) i~\(\sim\) una relació d'equivalència tal que
        \[
            x\sim y\sii x\ast y^{-1}\in H\text{ per a tot }x,y\in G.
        \]
        Aleshores el conjunt quocient~\(G/H\) amb l'operació
        \begin{align*}
        \ast\colon G/H\times G/H&\longrightarrow G/H\\
        [x]\ast[y]&\longmapsto[x\ast y]
        \end{align*}
        és un grup si i només si~\(H\) és un subgrup normal de~\(G\).
    \end{proposition}
    \begin{proof}
        %TODO
    \end{proof}
    \begin{definition}[Grup quocient]
        \labelname{grup quocient}\label{def:grup-quocient}\label{def:relacio-dequivalencia-entre-grups}\label{def:producte-entre-classes-versio-grups}
        Siguin~\(G\) un grup amb l'operació~\(\ast\) i~\(N\) un subgrup normal de~\(G\).
        Aleshores direm que el grup~\(G/N\) amb l'operació
        \begin{align*}
        \ast\colon G/N\times G/N&\longrightarrow G/N\\
        [x]\ast[y]&\longmapsto[x\ast y]
        \end{align*}
        és el grup quocient~\(G\) mòdul~\(N\).

        Aquesta definició té sentit per la proposició \myref{prop:grup-quocient}.
    \end{definition}
    \begin{lemma}
        \label{lema:operar-en-grups-es-bijectiu}
        Siguin~\(G\) un grup amb l'operació~\(\ast\),~\(H\) un subgrup de~\(G\),~\(x\) un element de~\(G\) i
        \begin{align*}
        f_{x}\colon H&\longrightarrow\{x\}H\\
        h&\longmapsto x\ast h
        \end{align*}
        una aplicació.
        Aleshores~\(f_{x}\) és bijectiva.
    \end{lemma}
    \begin{proof}
        Veiem que aquesta funció és bijectiva trobant la seva inversa:
        \begin{align*}
        f_{x}^{-1}\colon\{x\}H&\longrightarrow H\\
        y&\longmapsto x^{-1}\ast y
        \end{align*}
        i comprovant~\(f_{x}(f_{x}^{-1}(h))=h\) i~\(f_{x}^{-1}(f_{x}(h))=h\).
        Per tant~\(f\) és bijectiva\footnote{de fet,~\(f_{x}^{-1}=f_{x^{-1}}\)}.
        %REFERENCIES fonaments
    \end{proof}
    \begin{observation}
        \(\{x\}G=G\).
    \end{observation}
    \begin{theorem}[Teorema de Lagrange]
        \labelname{Teorema de Lagrange}
        \label{thm:Teorema-de-Lagrange}
        Siguin~\(G\) un grup d'ordre finit i~\(H\) un subgrup de~\(G\).
        Aleshores~\(\abs{H}\) divideix~\(\abs{G}\).
    \end{theorem}
    \begin{proof}
        Sigui~\(\ast\) l'operació de~\(G\).
        Fixem~\(x\in G\) i considerem la funció
        \begin{align*}
        f_{x}\colon H&\longrightarrow\{x\}H\\
        h&\longmapsto x\ast h
        \end{align*}

        Pel lema \myref{lema:operar-en-grups-es-bijectiu} trobem~\(\abs{H}=\abs{\{x\}H}\).
        Tenim que~\(\abs{G}\) és el resultat de multiplicar el número de classes d'equivalència pel nombre d'elements d'una de les classes, és a dir %REFERENCIES fonaments
        \[
            \abs{G}=\abs{G/H}\abs{H}
        \]
        i per tant~\(\abs{H}\) divideix~\(\abs{G}\).
    \end{proof}
    \begin{corollary} % matar salt de línia
        Sigui~\(G\) un grup d'ordre~\(p\) primer.
        Aleshores~\(G\) és un grup cíclic.
    \end{corollary}
    \begin{proof}
        Sigui~\(e\) l'element neutre de~\(G\).
        Prenem un element~\(g\in G\) diferent de~\(e\) i considerem el subgrup de~\(G\) generat per~\(g\).
        Pel \myref{thm:Teorema-de-Lagrange} tenim que l'ordre de~\(\langle\{g\}\rangle\) divideix l'ordre de~\(G\) i com que per hipòtesi l'orde de~\(G\) és primer i~\(g^{1}\neq e\), ja que~\(g\) és per hipòtesi diferent de l'element neutre, tenim que~\(\abs{\langle\{g\}\rangle}=p\), i per tant~\(\langle\{g\}\rangle=G\) i per la definició de \myref{def:grup-ciclic} tenim que~\(G\) és un grup cíclic.
    \end{proof}
    \begin{definition}[L'índex d'un subgrup en un grup]
        \labelname{l'índex d'un subgrup en un grup}
        \label{def:lindex-dun-subgrup-en-un-grup}
        Siguin~\(G\) un grup i~\(H\) un subgrup de~\(G\).
        Aleshores definim
        \[
            [G:H]=\frac{\abs{G}}{\abs{H}}
        \]
        com l'índex de~\(H\) a~\(G\).
    \end{definition}
\section{Tres Teoremes d'isomorfisme entre grups}
    \subsection{Morfismes entre grups}
    \begin{definition}[Morfisme entre grups]
        \labelname{morfisme entre grups}
        \label{def:morfisme-entre-grups}
        \labelname{monomorfisme entre grups}
        \label{def:monomorfisme-entre-grups}
        \labelname{epimorfisme entre grups}
        \label{def:epimorfisme-entre-grups}
        \labelname{isomorfisme entre grups}
        \label{def:isomorfisme-entre-grups}
        \labelname{endomorfisme entre grups}
        \label{def:endomorfisme-entre-grups}
        \labelname{automorfisme entre grups}
        \label{def:automorfisme-entre-grups}
        Siguin~\(G_{1}\) un grup amb l'operació~\(\ast\),~\(G_{2}\) un grup amb l'operació~\(\circ\) i~\(f\colon G_{1}\to G_{2}\) una aplicació que, per a tot~\(x,y\in G_{1}\) satisfà
        \[
            f(x\ast y)=f(x)\circ f(y).
        \]
        Aleshores diem que~\(f\) és un morfisme entre grups.
        Definim també
        \begin{enumerate}
            \item Si~\(f\) és injectiva direm que~\(f\) és un monomorfisme entre grups.
            \item Si~\(f\) és exhaustiva direm que~\(f\) és un epimorfisme entre grups.
            \item Si~\(f\) és bijectiva direm que~\(f\) és un isomorfisme entre grups.
            També escriurem~\(G_{1}\cong G_{2}\) i direm que~\(G_{1}\) i~\(G_{2}\) són grups isomorfs.
            \item Si~\(G_{1}=G_{2}\) direm que~\(f\) és un endomorfisme entre grups.
            \item Si~\(G_{1}=G_{2}\) i~\(f\) és bijectiva direm que~\(f\) és un automorfisme entre grups.
        \end{enumerate}
    \end{definition}
    \begin{proposition}
        \label{prop:morfismes-conserven-neutre-i-linvers-commuta-amb-el-morfisme}
            Siguin~\(G_{1}\) un grup amb element neutre~\(e\),~\(G_{2}\) un grup amb element neutre~\(e'\) i~\(f\colon G_{1}\to G_{2}\) un morfisme entre grups.
            Aleshores
        \begin{enumerate}
            \item\label{enum:morfismes-conserven-neutre-i-linvers-commuta-amb-el-morfisme-1}~\(f(e)=e'\).
            \item\label{enum:morfismes-conserven-neutre-i-linvers-commuta-amb-el-morfisme-2}~\(f\left(x^{-1}\right)=f(x)^{-1}\) per a tot~\(x\in G_{1}\).
        \end{enumerate}
    \end{proposition}
    \begin{proof}
        Siguin~\(\ast\) l'operació del grup~\(G_{1}\) i~\(\circ\) l'operació del grup~\(G_{2}\).
        Veiem primer el punt \eqref{enum:morfismes-conserven-neutre-i-linvers-commuta-amb-el-morfisme-1}.
        Per la definició de morfisme tenim que per a tot~\(x\in G_{1}\)
        \begin{align*}
        f(x)\circ f(e)&=f(x\ast e)\tag{\myref{def:morfisme-entre-grups}}\\
        &=f(x)\tag{\myref{def:lelement-neutre-del-grup}}\\
        &=f(x)\circ e'
        \end{align*}
        i per la proposició \myref{prop:podem-tatxar-pels-costats-en-grups} tenim~\(f(e)=e'\).

        Per demostrar el punt \eqref{enum:morfismes-conserven-neutre-i-linvers-commuta-amb-el-morfisme-2} en tenim prou en veure que per a tot~\(x\in G\)
        \begin{align*}
        f(x)\circ f(x^{-1})&=f(x\ast x^{-1})\tag{\myref{def:morfisme-entre-grups}}\\
        &=f(e)\tag{\myref{def:linvers-dun-element-dun-grup}}\\
        &=f(x^{-1}\ast x)\tag{\myref{def:morfisme-entre-grups}}\\
        &=f(x^{-1})\circ f(x)
        \end{align*}
        i pel punt \eqref{enum:morfismes-conserven-neutre-i-linvers-commuta-amb-el-morfisme-1} d'aquesta proposició~\(f(x)\circ f(x^{-1})=f(x^{-1})\circ f(x)=e'\), i per la proposició \myref{prop:unicitat-inversa-en-grups} tenim que~\(f(x^{-1})=f(x)^{-1}\), com volíem.
    \end{proof}
    \begin{proposition}\label{prop:conjugacio-de-morfismes-entre-grups-es-morfisme-entre-grups}
        Siguin~\(G\),~\(H\) i~\(K\) tres grups i~\(f\colon G\longrightarrow H\) i~\(g\colon H\longrightarrow K\) dos morfismes entre grups.
        Aleshores~\(g(f)\colon G\longrightarrow K\) és un morfisme entre grups.
    \end{proposition}
    \begin{proof}
        Siguin~\(\ast\),~\(\circ\) i~\(+\) les operacions de~\(G\),~\(H\) i~\(K\), respectivament.
        Per la definició de \myref{def:morfisme-entre-grups} tenim que per a tot~\(g_{1},g_{2}\in H\) i~\(h_{1},h_{2}\in H\) tenim~\(f(g_{1}\ast g_{2})=f(g_{1})\circ f(g_{2})\) i~\(g(h_{1}\circ h_{2})=g(h_{1})+g(h_{2})\).
        Per tant
        \[
            g(f(g_{1}\ast g_{2}))=g(f(g_{1})\circ f(g_{2}))=g(f(g_{1}))+g(f(g_{2})),
        \]
        i per la definició de \myref{def:morfisme-entre-grups} hem acabat.
    \end{proof}
    \begin{proposition}
        \label{prop:condicions-equivalents-abelia-i-ciclic-per-isomorfismes}
        Siguin~\(G_{1}\) i~\(G_{2}\) dos grups tals que
        \[
            G_{1}\cong G_{2}.
        \]
        Aleshores
        \begin{enumerate}
            \item\label{enum:condicions-equivalents-abelia-i-ciclic-per-isomorfismes-1}~\(G_{1}\) és un grup abelià si i només si~\(G_{2}\) és un grup abelià.
            \item\label{enum:condicions-equivalents-abelia-i-ciclic-per-isomorfismes-2}~\(G_{1}\) és un grup cíclic si i només si~\(G_{2}\) és un grup cíclic.
        \end{enumerate}
    \end{proposition}
    \begin{proof}
        Siguin~\(\ast\) l'operació de~\(G_{1}\) i~\(\circ\) l'operació de~\(G_{2}\).
        Per la definició de \myref{def:isomorfisme-entre-grups} tenim que existeix un~\(f\colon G_{1}\to G_{2}\) un isomorfisme entre grups.

        Comencem demostrant el punt \eqref{enum:condicions-equivalents-abelia-i-ciclic-per-isomorfismes-1}.
        Suposem doncs que~\(G_{1}\) és un grup abelià.
        Per la definició de \myref{def:grup-abelia} tenim que per a tot~\(a,b\in G_{1}\) es compleix~\(a\ast b=b\ast a\).
        Aleshores tenim
        \[
            f(a\ast b)=f(b\ast a)
        \]
        i per la definició de \myref{def:morfisme-entre-grups} tenim que
        \[
            f(a)\circ f(b)=f(b)\circ f(a),
        \]
        i per tant, com que per la definició de \myref{def:isomorfisme-entre-grups}~\(f\) és un bijectiu,~\(G_{2}\) satisfà la definició de \myref{def:grup-abelia}.

        Demostrem ara el punt \eqref{enum:condicions-equivalents-abelia-i-ciclic-per-isomorfismes-2}.
        Suposem doncs que~\(G_{1}\) és un grup cíclic.
        Per la definició de \myref{def:grup-ciclic} tenim que~\(G_{1}=\{g^{i}\}_{i\in\mathbb{Z}}\) per a un cert~\(g\in G_{1}\).
        Per tant, com que~\(f\) és bijectiva per la definició de \myref{def:isomorfisme-entre-grups} tenim que per a tot~\(x\in G_{2}\) es compleix~\(x=f(g^{i})\) per a un cert~\(i\in\mathbb{Z}\), i per la definició de \myref{def:morfisme-entre-grups} tenim que\footnote{el primer és amb l'operació~\(\ast\) i el segon amb l'operació~\(\circ\).}~\(f(g^{i})=f(g)^{i}\), i per la definició de \myref{def:grup-ciclic}~\(G_{2}\) és un grup cíclic.
    \end{proof}
    \begin{definition}[Nucli i imatge d'un morfisme entre grups]
        \labelname{nucli d'un morfisme entre grups}
        \label{def:nucli-dun-morfisme-entre-grups}
        \labelname{imatge d'un morfisme entre grups}
        \label{def:imatge-dun-morfisme-entre-grups}
        Siguin~\(G_{1}\) un grup amb element neutre~\(e\),~\(G_{2}\) un grup amb  element neutre~\(e'\) i~\(f\colon G_{1}\to G_{2}\) un morfisme entre grups.
        Aleshores definim el nucli de~\(f\) com
        \[
            \ker(f)=\{x\in G_{1}\mid f(x)=e'\},
        \]
        i la imatge de~\(f\) com
        \[
            \Ima(f)=\{f(x)\in G_{2}\mid x\in G_{1}\}.
        \]
    \end{definition}
    \begin{observation}
        \label{obs:nucli-dun-morfisme-entre-grups-es-subconjunt-del-grup-dentrada-imatge-nes-del-de-sortida}
        \(\ker(f)\subseteq G_{1}\),~\(\Ima(f)\subseteq G_{2}\).
    \end{observation}
    \begin{proposition}
        \label{prop:el-nucli-dun-morfisme-es-un-subgrup-normal-del-grup-de-sortda}
        \label{prop:la-imatge-dun-morfisme-es-un-subgrup-del-grup-darribada}
        Siguin~\(G_{1}\) i~\(G_{2}\) dos grups i~\(f\colon G_{1}\to G_{2}\) un morfisme entre grups.
        Aleshores
        \begin{enumerate}
            \item\label{enum:el-nucli-dun-morfisme-es-un-subgrup-normal-del-grup-de-sortda-1}~\(\ker(f)\) és un subgrup normal de~\(G_{1}\).
            \item\label{enum:la-imatge-dun-morfisme-es-un-subgrup-del-grup-darribada-2}~\(\Ima(f)\) és un subgrup de~\(G_{2}\).
        \end{enumerate}
    \end{proposition}
    \begin{proof}
        Aquest enunciat té sentit per l'observació \myref{obs:nucli-dun-morfisme-entre-grups-es-subconjunt-del-grup-dentrada-imatge-nes-del-de-sortida}.

        Siguin~\(\ast\) l'operació de~\(G_{1}\) i~\(\circ\) l'operació de~\(G_{2}\) i~\(e\) l'element neutre de~\(G_{1}\) i~\(e'\) l'element neutre de~\(G_{2}\).
        Primer comprovem el punt \eqref{enum:el-nucli-dun-morfisme-es-un-subgrup-normal-del-grup-de-sortda-1}.
        Comencem veient que~\(\ker(f)\) és un subgrup de~\(G_{1}\).
        Per la proposició \myref{prop:condicio-equivalent-a-subgrup} tenim que ens cal amb veure que si~\(a,b\in\ker(f)\), aleshores~\(a\ast b^{-1}\in\ker(f)\).
        Això és cert ja que si~\(a,b\in\ker(f)\) aleshores~\(f(a)=e'\) i~\(f(b^{-1})=e'\), i per tant~\(a\ast b^{-1}=e\ast e^{-1}=e\), el que significa que~\(f(a\ast b^{-1})=e'\), i tenim~\(a\ast b^{-1}\in\ker(f)\).

        Comprovem ara que el subgrup és normal.
        Per la proposició \myref{prop:condicions-equivalents-a-subgrup-normal} en tenim prou en veure que per a tot~\(x\in\ker(f)\) i~\(g\in G\),~\(x\ast g\ast x^{-1}\in\ker(f)\).
        Això ho veiem notant que si~\(g\in\ker(f)\),~\(f(g)=e'\), i per tant~\(f(x\ast g\ast x^{-1})=f(x)\circ e'\circ f(x^{-1})\)i això és~\(f(x\ast x^{-1})=e'\), i per tant~\(x\ast g\ast x^{-1}\in\ker(f)\).

        Acabem veient el punt \eqref{enum:la-imatge-dun-morfisme-es-un-subgrup-del-grup-darribada-2}.
        De nou per la proposició \myref{prop:condicio-equivalent-a-subgrup} tenim que si per a tot~\(f(a),f(b)\in\Ima(f)\) tenim~\(f(a)\circ f(b)^{-1}\in\Ima(f)\) aleshores~\(\Ima(f),\) és un subgrup de~\(G_{2}\).
        Això és cert, ja que per la definició de \myref{def:morfisme-entre-grups} i la proposició \myref{prop:morfismes-conserven-neutre-i-linvers-commuta-amb-el-morfisme} tenim~\(f(a)\circ f(b)^{-1}=f(a\ast b^{-1})\); i per la definició de grup~\(a\ast b^{-1}\in G_{1}\), i per la definició de \myref{def:morfisme-entre-grups} tenim que~\(f(a)\circ f(b)^{-1}\in\Ima(f)\), i per tant~\(\Ima(f)\) és un subgrup de~\(G_{2}\), com volíem veure.
    \end{proof}
    \begin{proposition}
        \label{prop:condicions-equivalents-a-monomorfisme-i-epimorfisme-per-nucli-i-imatge}
        Siguin~\(G_{1}\) un grup amb element neutre~\(e\),~\(G_{2}\) un grup, i~\(f\colon G_{1}\to G_{2}\) un morfisme entre grups.
        Aleshores
        \begin{enumerate}
            \item\label{enum:condicions-equivalents-a-monomorfisme-i-epimorfisme-per-nucli-i-imatge-1}~\(f\) és un monomorfisme si i només si~\(\ker(f)=\{e\}\).
            \item\label{enum:condicions-equivalents-a-monomorfisme-i-epimorfisme-per-nucli-i-imatge-2}~\(f\) és un epimorfisme si i només si~\(\Ima(f)=G_{2}\).
        \end{enumerate}
    \end{proposition}
    \begin{proof}
        Siguin~\(\ast\) l'operació de~\(G_{1}\),~\(\circ\) l'operació de~\(G_{2}\) i~\(e'\) l'element neutre de~\(G_{2}\).
        Comencem fent la demostració del punt \eqref{enum:condicions-equivalents-a-monomorfisme-i-epimorfisme-per-nucli-i-imatge-1} per la implicació cap a la dreta (\(\implica\)).
        Suposem doncs que~\(f\) és un monomorfisme, i per tant injectiva.
        Per la definició de \myref{def:nucli-dun-morfisme-entre-grups} tenim que~\(\ker(f)=\{x\in G_{1}\mid f(x)=e'\}\).
        Suposem~\(x\in G_{1}\), és a dir,~\(f(x)=e'\).
        Ara bé, com que~\(f\) és injectiva per la proposició \myref{prop:morfismes-conserven-neutre-i-linvers-commuta-amb-el-morfisme} ha de ser~\(\ker(f)=\{e\}\).%REF injectiva

        Demostrem ara la implicació cal a l'esquerra (\(\implicatper\)).
        Suposem doncs que~\(\ker(f)=\{e\}\).
        Siguin~\(x,y\in G_{1}\) dos elements que satisfacin~\(f(x)=f(y)\).
        Com que, per la proposició \myref{prop:el-nucli-dun-morfisme-es-un-subgrup-normal-del-grup-de-sortda}~\(\ker(f)\) és un subgrup de~\(G_{1}\), tenim que~\(x\ast y^{-1}\in G_{1}\), i per tant
        \begin{align*}
        f(x\ast y^{-1})&=f(x)\circ f(y^{-1})\tag{\myref{def:morfisme-entre-grups}}\\
        &=f(x)\circ f(y)^{-1}\tag{Proposició \myref{prop:morfismes-conserven-neutre-i-linvers-commuta-amb-el-morfisme}}\\
        &=f(y)\circ f(y)^{-1}=e',
        \end{align*}
        i per tant~\(x\ast y^{-1}\in\ker(f)\), però per hipòtesi teníem~\(\ker(f)=\{e\}\), i per tant ha de ser~\(x\ast y^{-1}=e\), el que és equivalent a~\(x=y\), i per tant~\(f\) és injectiva.
        %REF definició

        Demostrem ara el punt \eqref{enum:condicions-equivalents-a-monomorfisme-i-epimorfisme-per-nucli-i-imatge-2} començant per la implicació cap a la dreta (\(\implica\)).
        Suposem doncs que~\(f\) és un epimorfisme, i per tant exhaustiva, i per tant per a cada~\(y\in G_{2}\) existeix un~\(x\in G_{1}\) tal que~\(f(x)=y\), i per la definició d'\myref{def:imatge-dun-morfisme-entre-grups} tenim que~\(\Ima(f)=G_{2}\).

        Acabem demostrant la implicació cap a l'esquerra (\(\implicatper\)).
        Suposem doncs que~\(\Ima(f)=G_{2}\) i prenem~\(y\in G_{2}\).
        Aleshores per la definició d'\myref{def:imatge-dun-morfisme-entre-grups} tenim que existeix un~\(x\in G_{1}\) tal que~\(f(x)=y\), i per tant~\(f\) és exhaustiva.
        %REF definició
    \end{proof}
    \begin{proposition}
        \label{prop:les-imatges-dels-subgrups-son-subgrups-el-mateix-amb-subgrups-normals}
        Siguin~\(G_{1}\) i~\(G_{2}\) dos grups i~\(f\colon G_{1}\to G_{2}\) un morfisme entre grups.
        Aleshores
        \begin{enumerate}
            \item\label{enum:les-imatges-dels-subgrups-son-subgrups-el-mateix-amb-subgrups-normals-1} Si~\(H_{1}\leq G_{1}\implica\{f(h)\in G_{2}\mid h\in H_{1}\}\leq G_{2}\).
            \item\label{enum:les-imatges-dels-subgrups-son-subgrups-el-mateix-amb-subgrups-normals-2} Si~\(H_{2}\leq G_{2}\implica\{h\in G_{1}\mid f(h)\in H_{2}\}\leq G_{1}\).
            \item\label{enum:les-imatges-dels-subgrups-son-subgrups-el-mateix-amb-subgrups-normals-3} Si~\(H_{2}\trianglelefteq G_{2}\implica\{h\in G_{1}\mid f(h)\in H_{2}\}\trianglelefteq G_{1}\).
        \end{enumerate}
    \end{proposition}
    \begin{proof}
        Siguin~\(\ast\) l'operació de~\(G_{1}\) i~\(\circ\) l'operació de~\(G_{2}\).
        Comprovem primer el punt \eqref{enum:les-imatges-dels-subgrups-son-subgrups-el-mateix-amb-subgrups-normals-1}.
        Suposem doncs que~\(H_{1}\) és un subgrup de~\(G_{1}\).
        Denotarem~\(H=\{f(h)\in G_{2}\mid h\in H_{1}\}\).
        Siguin~\(x,y\in H_{1}\); per la proposició \myref{prop:condicio-equivalent-a-subgrup} només ens cal veure que~\(f(x)\circ f(y)^{-1}\in H\).
        Això és
        \begin{align*}
        f(x)\circ f(y)^{-1}&=f(x)\circ f(y^{-1})\tag{Proposició \myref{prop:morfismes-conserven-neutre-i-linvers-commuta-amb-el-morfisme}}\\
        &=f(x\ast y^{-1}).\tag{\myref{def:morfisme-entre-grups}}
        \end{align*}
        Ara bé, com que~\(x,y\in H_{1}\) i~\(H_{1}\) és un subgrup de~\(G_{1}\), per la proposició \myref{prop:condicio-equivalent-a-subgrup} tenim que~\(x\ast y^{-1}\in H_{1}\), i per tant~\(f(x\ast y^{-1})\in H\), i per la definició de \myref{def:morfisme-entre-grups} i la proposició \myref{prop:morfismes-conserven-neutre-i-linvers-commuta-amb-el-morfisme} tenim que~\(f(x)\circ f(y)^{-1}\in H\), i per tant~\(H\) és un subgrup de~\(G_{2}\), com volíem veure.

        Comprovem ara el punt \eqref{enum:les-imatges-dels-subgrups-son-subgrups-el-mateix-amb-subgrups-normals-2}.
        Suposem doncs que~\(H_{2}\) és un subgrup de~\(G_{2}\) i denotem~\(H=\{h\in G_{1}\mid f(h)\in H_{2}\}\).
        Per la proposició \myref{prop:condicio-equivalent-a-subgrup} només ens cal veure que per a tot~\(x,y\in H\) es satisfà~\(x\ast y^{-1}\in H\).
        Si~\(x,y\in H\) aleshores tenim que~\(f(x),f(y)\in H_{2}\), i com que~\(H_{2}\) és un grup, aleshores per la definició de \myref{def:grup} ha de ser~\(f(x)\circ f(y^{-1})\in H_{2}\) Aleshores, per la definició de \myref{def:morfisme-entre-grups} tenim~\(f(x)\circ f(y^{-1})=f(x\ast y^{-1})\), i per tant~\(x\ast y^{-1}\in H\) i així tenim que~\(H\) és un subgrup de~\(G_{1}\).

        Veiem el punt \eqref{enum:les-imatges-dels-subgrups-son-subgrups-el-mateix-amb-subgrups-normals-3} per acabar.
        Suposem doncs que~\(H_{2}\) és un subgrup normal de~\(G_{2}\) i definim~\(H=\{h\in G_{1}\mid f(h)\in H_{2}\}\).
        Per demostrar-ho prenem~\(g\in G_{1}\),~\(h\in H_{1}\) tal que~\(f(h)\in H\) i fem
        \begin{align*}
        f(g)\circ f(h)\circ f(g)^{-1}&=f(g)\circ f(h)\circ f(g^{-1})\tag{Proposició \myref{prop:morfismes-conserven-neutre-i-linvers-commuta-amb-el-morfisme}}\\
        &=f(g\ast h\ast g^{-1})\tag{\myref{def:morfisme-entre-grups}}
        \end{align*}
        Ara bé, com que~\(H_{2}\) és un subgrup normal de~\(G_{2}\), tenim que, per a tot~\(g\in G_{1}\),~\(f(g\ast h\ast g^{-1})\in H_{2}\), i per tant~\(g\ast h\ast g^{-1}\in H\), que satisfà la definició de \myref{def:subgrup-normal} per la proposició \myref{prop:condicions-equivalents-a-subgrup-normal}.
    \end{proof}
    \begin{theorem}[Teorema de representació de Cayley]
        \labelname{Teorema de representació de Cayley}
        \label{thm:Cayley-Teorema-de-representacio}
        Sigui~\(G\) un grup amb l'operació~\(\ast\).
        Aleshores~\(G\) és isomorf a un subgrup de~\(\GrupSimetric_{G}\) amb l'operació~\(\circ\), on~\(\GrupSimetric_{G}\) és el grup simètric dels elements de~\(G\).
        %REPASSAR
    \end{theorem}
    \begin{proof}
        Sigui~\(e\) l'element neutre de~\(G\).
        Definim
        \begin{alignat}{2}
        \varphi\colon G&\longrightarrow\GrupSimetric_{G}&&\label{eq:thm:Cayley-Teorema-de-representacio-1}\\
        g&\longmapsto\sigma_{g}&\colon G&\longrightarrow G\label{eq:thm:Cayley-Teorema-de-representacio-2}\\
        &&x&\longmapsto g\ast x\nonumber
        \end{alignat}
        Tenim que~\(\sigma_{g}\) és bijectiva ja que és una permutació.
        %REFERENCIES fonaments + demostrar donant inversa
        Comprovarem que~\(\varphi\) és un monomorfisme entre grups.
        Veiem primer que és un morfisme entre grups.
        Prenem~\(g,g'\in G\).
        Per la definició \eqref{eq:thm:Cayley-Teorema-de-representacio-1} tenim que~\(\varphi(g\ast g')=\sigma_{g\ast g'}\).
        Per veure que~\(\sigma_{g\ast g'}=\sigma_{g}\circ\sigma_{g'}\) observem que per a tot~\(x\in G\)
        \begin{align*}
        \sigma_{g\ast g'}(x)&=g\ast g'\ast x\\
        &=g\ast\sigma_{g'}(x)\\
        &=\sigma_{g}\circ\sigma_{g'}(x),
        \end{align*}
        i per la definició de \myref{def:morfisme-entre-grups} tenim que~\(\varphi\) és un morfisme entre grups.
        Veiem ara que~\(\varphi\) és un monomorfisme.
        Per la definició de \myref{def:nucli-dun-morfisme-entre-grups} tenim que
        \[
            \ker(\varphi)=\{x\in G\mid f(x)=\text{Id}_{G}\}.
        \]
        Ara bé,~\(\sigma_{g}=\text{Id}\) és, per la definició \eqref{eq:thm:Cayley-Teorema-de-representacio-2}, equivalent a dir que~\(g\ast x=x\) per a tota~\(x\in G\), i per la definició de \myref{def:lelement-neutre-del-grup} això és si i només si~\(g=e\), i per tant
        \[
            \ker(\varphi)=\{e\},
        \]
        i per la proposició \myref{prop:condicions-equivalents-a-monomorfisme-i-epimorfisme-per-nucli-i-imatge} tenim que~\(\varphi\) és un monomorfisme, com volíem veure.

        Per tant, per la proposició \myref{prop:les-imatges-dels-subgrups-son-subgrups-el-mateix-amb-subgrups-normals} tenim que
        \[
            G\cong \Ima(\varphi)\leq\GrupSimetric_{G}.\qedhere
        \]
    \end{proof}
%    \begin{corollary} %TODO %FER %FALS
%        Si~\(G\) té ordre~\(n!\) aleshores~\(G\cong\GrupSimetric_{n}\).
%    \end{corollary}
    \subsection{Teoremes d'isomorfisme entre grups}
    \begin{theorem}
        \label{thm:Teorema-fonamental-dels-isomorfismes}%[Teorema Fonamental dels Isomorfismes]
            Siguin~\(G_{1}\) un grup amb l'operació~\(\ast\),~\(G_{2}\) un grup amb l'operació~\(\circ\) i~\(f\colon G_{1}\to G_{2}\) un morfisme entre grups.
            Aleshores~\(G_{1}/\ker(f)\cong\Ima(f)\).
    \end{theorem}
    \begin{proof}
        Siguin~\(e\) l'element neutre de~\(G_{1}\) i~\(e'\) l'element neutre de~\(G_{2}\).
        Definim l'aplicació
        \begin{align}
        \label{thm:Teorema-fonamental-dels-isomorfismes:eq1}
        \varphi\colon G_{1}/\ker(f)&\longleftrightarrow\Ima(f)\\
        [x]&\longmapsto f(x)\nonumber
        \end{align}
        Comprovem primer que aquesta aplicació està ben definida:

        Suposem que~\([x]=[x']\).
        Això és que~\(x'\in\{x\}\ker(f)\), i equivalentment~\(x'=x\ast h\) per a cert~\(h\in\ker(f)\).
        Per tant
        \begin{align*}
        \varphi([x'])&=\varphi([x\ast h])\tag{Definició \eqref{def:producte-entre-classes-versio-grups}}\\
        &=f(x\ast h)\tag{Definició \eqref{thm:Teorema-fonamental-dels-isomorfismes:eq1}}\\
        &=f(x)\circ f(h)\tag{\myref{def:morfisme-entre-grups}}\\
        &=f(x)\circ e'\tag{\myref{def:nucli-dun-morfisme-entre-grups}}\\
        &=f(x)=\varphi([x])\tag{Definició \eqref{thm:Teorema-fonamental-dels-isomorfismes:eq1}}
        \end{align*}
        i per tant~\(\varphi\) està ben definida.
        Veiem ara que~\(\varphi\) és un morfisme entre grups.
        Tenim que
        \begin{align*}
        \varphi([x]\ast[y])&=\varphi([x\ast y])\tag{Definició \eqref{def:producte-entre-classes-versio-grups}}\\
        &=f(x\ast y)\tag{Definició \eqref{thm:Teorema-fonamental-dels-isomorfismes:eq1}}\\
        &=f(x)\circ f(y)\tag{\myref{def:morfisme-entre-grups}}\\
        &=\varphi([x])\circ \varphi([y]),\tag{Definició \eqref{thm:Teorema-fonamental-dels-isomorfismes:eq1}}
        \end{align*}
        i per la definició de \myref{def:morfisme-entre-grups}~\(\varphi\) és un morfisme entre grups.
        Continuem demostrant que~\(\varphi\) és injectiva.
        Per la definició de \myref{def:nucli-dun-morfisme-entre-grups} tenim que~\(\ker(\varphi)=\{[x]\in G/\ker(f)\mid\varphi([x])=e\}\), i per tant~\(\ker(\varphi)=\ker(f)\), ja que~\(f(x)=e\) si i només si~\(x\in\ker(f)\), i per tant~\(\ker(\varphi)=[e]\) i per la proposició \myref{prop:condicions-equivalents-a-monomorfisme-i-epimorfisme-per-nucli-i-imatge}~\(\varphi\) és injectiva.

        Per veure que~\(\varphi\) és exhaustiva veiem que si~\([x]\in G_{1}/\ker(f)\), per la definició de \myref{def:relacio-dequivalencia-entre-grups} tenim que~\(x=x'\ast y\) per a uns certs~\(x'\in G_{2}\),~\(h\in\ker(f)\), i per tant
        \begin{align*}
        \varphi([x])&=\varphi([x'\ast h])\\
        &=\varphi([x']\ast[h])\tag{\myref{def:producte-entre-classes-versio-grups}}\\
        &=f(x')\circ f(e)\tag{\myref{def:morfisme-entre-grups}}\\
        &=f(x')\tag{\myref{def:grup}}\\
        &=f(x\ast h^{-1})\\
        &=f(x)\circ f(h^{-1})\tag{\myref{def:morfisme-entre-grups}}\\
        &=f(x)\circ f(h)^{-1}\tag{Proposició \myref{prop:morfismes-conserven-neutre-i-linvers-commuta-amb-el-morfisme}}\\
        &=f(x)\circ e^{-1}=f(x).\tag{Proposició \myref{prop:linvers-de-lelement-neutre-dun-grup-es-ell-mateix}}
        \end{align*}

        Així veiem que~\(\Ima(\varphi)=\Ima(f)\).
        Per tant~\(\varphi\) és un isomorfisme, i per la definició d'\myref{def:isomorfisme-entre-grups} tenim~\(G_{1}/\ker(f)\cong\Ima(f)\), com volíem veure.
        %REFERENCIES BIECTIVITAT ALS ISO-
    \end{proof}
    \begin{theorem}[Primer Teorema de l'isomorfisme]
        \labelname{Primer Teorema de l'isomorfisme entre grups}
        \label{thm:Primer-Teorema-de-lisomorfisme-entre-grups}
            Siguin~\(G_{1}\) i~\(G_{2}\) dos grups i~\(f\colon G_{1}\to G_{2}\) un epimorfisme entre grups.
            Aleshores
        \begin{enumerate}
            \item\label{enum:Primer-Teorema-de-lisomorfisme-entre-grups-1}~\(G_{1}/\ker(f)\cong G_{2}\).
            \item\label{enum:Primer-Teorema-de-lisomorfisme-entre-grups-2} L'aplicació
            \begin{align}
            \label{eq:primer-teorema-de-lisomorfisme-eq2}
            \varphi_{1}\colon\{H\mid\ker(f)\leq H\leq G\}&\longleftrightarrow\{K\mid K\leq G_{2}\}\\
            H&\longmapsto\{f(h)\in G_{2}\mid h\in H\}\nonumber
            \end{align}
            és bijectiva.
%            \marginpar{Si ningú ve del futur per aturar-te, com de dolenta pot ser la decisió que estàs prenent?}
            \item\label{enum:Primer-Teorema-de-lisomorfisme-entre-grups-3} L'aplicació
            \begin{align}
            \label{eq:primer-teorema-de-lisomorfisme-eq3}
            \varphi_{2}\colon\{H\mid\ker(f)\leq H\trianglelefteq G\}&\longleftrightarrow\{K\mid K\trianglelefteq G_{2}\}\\
            H&\longmapsto\{f(h)\in G_{2}\mid h\in H\}\nonumber
            \end{align}
            és bijectiva.
        \end{enumerate}
    \end{theorem}
    \begin{proof}
        Siguin~\(\ast\) l'operació de~\(G_{1}\),~\(e\) l'element neutre de~\(G_{1}\) i~\(e'\) l'element neutre de~\(G_{2}\).

        El punt \eqref{enum:Primer-Teorema-de-lisomorfisme-entre-grups-1} és conseqüència del Teorema \myref{thm:Teorema-fonamental-dels-isomorfismes}, ja que si~\(f\) és exhaustiva,~\(\Ima(f)=G_{2}\), i per tant~\(G_{1}/\ker(f)\cong G_{2}\).

        Per veure el punt \eqref{enum:Primer-Teorema-de-lisomorfisme-entre-grups-2} comencem demostrant que~\(\varphi_{1}\) està ben definida.
        Siguin~\(H_{1}=H_{2}\in\{H\mid\ker(f)\leq H\leq G\}\).
        Aleshores, per la hipòtesi \eqref{eq:primer-teorema-de-lisomorfisme-eq2} tenim~\(\varphi_{1}(H_{1})=\{f(h)\in G_{2}\mid h\in H_{1}\}\) i~\(\varphi_{1}(H_{2})=\{f(h)\in G_{2}\mid h\in H_{2}\}\), i com que~\(f\) és una aplicació, i per tant ben definida,~\(\varphi_{1}(H_{1})=\varphi_{1}(H_{2})\).

        Continuem comprovant que~\(\varphi_{1}\) és bijectiva.
        Per veure que és injectiva prenem~\(H_{1},H_{2}\in\{H\mid\ker(f)\leq H\leq G\}\) tals que~\(\varphi_{1}(H_{1})=\varphi_{1}(H_{2})\).
        Això, per la hipòtesi \eqref{eq:primer-teorema-de-lisomorfisme-eq2} és
        \[
            \{f(h)\in G_{2}\mid h\in H_{1}\}=\{f(h)\in G_{2}\mid h\in H_{2}\}.
        \]
        Per tant siguin~\(h_{1}\in H_{1}\),~\(h_{2}\in H_{2}\) tals que~\(f(h_{1})=f(h_{2})\).
        Equivalentment, per la proposició \myref{prop:unicitat-inversa-en-grups} i la definició de \myref{def:linvers-dun-element-dun-grup} i la proposició \myref{prop:linvers-de-lelement-neutre-dun-grup-es-ell-mateix} tenim les igualtats~\(f(h_{2}^{-1}\ast h_{1})=f(h_{1}^{-1}\ast h_{2})=e'\), i per la definició de \myref{def:nucli-dun-morfisme-entre-grups} tenim~\(h_{2}^{-1}\ast h_{1},h_{1}^{-1}\ast h_{2}\in\ker(f)\), i per la hipòtesi \eqref{eq:primer-teorema-de-lisomorfisme-eq2} això és~\(h_{2}^{-1}\ast h_{1}\in\ker(f)\subseteq H_{2}\) i~\(h_{1}^{-1}\ast h_{2}\in\ker(f)\subseteq H_{1}\).
        Observem que això és que~\(h_{1}\in\{h_{2}\}\ker(f))\subseteq H_{2}\) i~\(h_{2}\in\{h_{1}\}\ker(f))\subseteq H_{1}\).
        Això vol dir que~\(H_{1}\subseteq H_{2}\) i~\(H_{2}\subseteq H_{1}\), i per doble inclusió això és~\(H_{1}=H_{2}\), com volíem veure.
        %FER Referència doble inclusió

        Per veure que~\(\varphi_{1}\) és exhaustiva tenim que per la proposició \myref{prop:les-imatges-dels-subgrups-son-subgrups-el-mateix-amb-subgrups-normals} i per la hipòtesi \eqref{eq:primer-teorema-de-lisomorfisme-eq2} tenim que donat un conjunt~\(K\) tal que~\(K\leq G_{2}\) aleshores el conjunt~\(H=\{k\in G_{1}\mid f(h)\in K\}\) satisfà~\(H\leq G_{1}\), i per la definició de \myref{def:nucli-dun-morfisme-entre-grups} tenim que es compleix~\(\ker(f)\leq H\leq G_{1}\), i per tant~\(\varphi_{1}(H)=K\), i per tant~\(\varphi\) és exhaustiva i per tant bijectiva.
        %FER referències fonaments.

        Es pot demostrar el punt \eqref{enum:Primer-Teorema-de-lisomorfisme-entre-grups-3} amb el mateix argument que hem donat per demostrar el punt \eqref{enum:Primer-Teorema-de-lisomorfisme-entre-grups-2}.
    \end{proof}
    \begin{proposition}
        \label{prop:producte-de-subgrups-amb-un-de-normal-es-subgrup}
        Siguin~\(G\) un grup i~\(H\),~\(K\) subgrups de~\(G\).
        Aleshores
        \begin{enumerate}
            \item\label{enum:producte-de-subgrups-amb-un-de-normal-es-subgrup-1} Si~\(K\trianglelefteq G\), aleshores~\(HK\leq G\).
            \item\label{enum:producte-de-subgrups-amb-un-de-normal-es-subgrup-2} Si~\(H,K\trianglelefteq G\), aleshores~\(HK\trianglelefteq G\).
        \end{enumerate}
    \end{proposition}
    \begin{proof}
        Sigui~\(\ast\) l'operació de~\(G\).
        Comencem veient el punt \eqref{enum:producte-de-subgrups-amb-un-de-normal-es-subgrup-1}.
        Per la proposició \myref{prop:condicio-equivalent-a-subgrup} només ens cal comprovar que per a tot~\(x,y\in HK\) es satisfà~\(x\ast y^{-1}\in HK\).
        Siguin doncs~\(x,y\in HK\), que podem reescriure com~\(x=h_{1}\ast k_{1}\) i~\(y=h_{2}\ast k_{2}\).
        Calculem~\(x\ast y^{-1}\):
        \begin{align*}
        x\ast y^{-1}&=h_{1}\ast k_{1}\ast (h_{2}\ast k_{2})^{-1}\\
        &=h_{1}\ast k_{1}\ast k_{2}^{-1}\ast h_{2}^{-1}\tag{Proposició \myref{prop:invers-de-a-b-b-invers-a-invers}}\\
        &=h_{1}\ast k_{1}\ast h_{2}^{-1}\ast k_{2}^{-1},\tag{\myref{def:subgrup-normal}}\\
        &=h_{1}\ast h_{2}^{-1}\ast k_{1}\ast k_{2}^{-1},\tag{\myref{def:subgrup-normal}}
        \end{align*}
        i com que, per la definició de \myref{def:grup} tenim~\(h_{1}\ast h_{2}^{-1}\in H\) i~\(k_{1}\ast k_{2}^{-1}\in K\), veiem que~\(x\ast y^{-1}\in HK\), com volíem demostrar.

        La demostració del punt \eqref{enum:producte-de-subgrups-amb-un-de-normal-es-subgrup-2} és anàloga a la del punt \eqref{enum:producte-de-subgrups-amb-un-de-normal-es-subgrup-1}.
    \end{proof}
    \begin{lemma}
        \label{lema:Segon-Teorema-de-lisomorfisme-entre-grups}
        Siguin~\(G\) un grup,~\(H\) un subgrup de~\(G\) i~\(K\) un subgrup normal de~\(G\).
        Aleshores~\(H\cap K\trianglelefteq H\).
    \end{lemma}
    \begin{proof}
        Siguin~\(\ast\) l'operació de~\(G\).
        Prenem~\(x\in H\) i~\(y\in H\cap K\).
        Per la proposició \myref{prop:condicions-equivalents-a-subgrup-normal} només hem de veure que per a tot~\(x\in H\) i~\(y\in H\cap K\), es satisfà~\(x^{-1}\ast y\ast x\in H\).
        Ara bé, com que~\(K\) és un grup normal, per la mateixa proposició \myref{prop:condicions-equivalents-a-subgrup-normal}, com que per hipòtesi~\(y\in H\cap K\), i en particular~\(y\in K\), tenim que~\(x^{-1}\ast y\ast x\in K\).
        Per veure que~\(x^{-1}\ast y\ast x\in H\) tenim prou amb veure que~\(x,y\in H\), i com que~\(H\) és un subgrup de~\(G\), i per tant un grup, per la definició de \myref{def:grup} tenim que~\(x^{-1}\ast y\ast x\in H\), i per tant~\(x^{-1}\ast y\ast x\in H\cap K\).
    \end{proof}
    \begin{theorem}[Segon Teorema de l'isomorfisme]
        \labelname{Segon Teorema de l'isomorfisme entre grups}
        \label{thm:Segon-Teorema-de-lisomorfisme-entre-grups}
        Siguin~\(G\) un grup,~\(H\) un subgrup de~\(G\) i~\(K\) un subgrup normal de~\(G\).
        Aleshores
        \[
            (HK)/K\cong H/(H\cap K).
        \]
    \end{theorem}
    \begin{proof}
        Aquest enunciat té sentit pel lema \myref{lema:Segon-Teorema-de-lisomorfisme-entre-grups}.

        Siguin~\(\ast\) l'operació de~\(G\) i~\(e\) l'element neutre de~\(G\).
        Definim
        \begin{align}\label{hipotesi:definicio-f-per-segon-teorea-de-lisomorfisme}
        f\colon HK&\longrightarrow H/(H\cap K)\\
        h\ast k&\longmapsto [h].\nonumber
        \end{align}
        Demostrarem que~\(f\) és un epimorfisme; però primer cal veure que~\(f\) està ben definida.
        Prenem doncs~\(h_{1}\ast k_{1},h_{2}\ast k_{2}\in HK\) amb~\(h_{1},h_{2}\in H\) i~\(k_{1},k_{2}\in K\) tals que~\(h_{1}\ast k_{1}=h_{2}\ast k_{2}\), i per tant~\(h_{2}^{-1}\ast h_{1}=k_{2}\ast k_{1}^{-1}\).
        Ara bé, com que per hipòtesi i per la definició de \myref{def:linvers-dun-element-dun-grup} tenim que~\(h_{1},h_{2}^{-1}\in H\) i a la vegada~\(k_{2},k_{1}^{-1}\in K\), per la definició de \myref{def:grup} tenim~\(h_{2}^{-1}\ast h_{1}\in H\) i~\(k_{2}\ast k_{1}^{-1}\in K\) i com que~\(h_{2}^{-1}\ast h_{1}=k_{2}\ast k_{1}^{-1}\) tenim que~\([h_{2}^{-1}\ast h_{1}]=[k_{2}^{-1}\ast k_{1}]=[e]\), i per la definició de \myref{def:relacio-dequivalencia-entre-grups} %areglar referències o definir producte individualment
        tenim que~\([h_{1}]=[h_{2}]\) i per tant~\(f\) està ben definida.

        Veiem ara que~\(f\) és un morfisme entre grups.
        Prenem~\(h_{1},h_{2}\in H\) i~\(k_{1},k_{2}\in K\), i per tant~\(h_{1}\ast k_{1},h_{2}\ast k_{2}\in HK\), i fem
        \begin{align*}
        f(h_{1}\ast k_{1}\ast h_{2}\ast k_{2})&=[h_{1}\ast h_{2}]\tag{Definició \eqref{hipotesi:definicio-f-per-segon-teorea-de-lisomorfisme}}\\
        &=[h_{1}]\ast[h_{2}]\tag{Definició \eqref{def:producte-entre-classes-versio-grups}}\\
        &=f(h_{1}\ast k_{1})\ast f(h_{2}\ast k_{2})\tag{Definició \eqref{hipotesi:definicio-f-per-segon-teorea-de-lisomorfisme}}
        \end{align*}
        i per tant~\(f\) satisfà la definició de \myref{def:morfisme-entre-grups}.

        Continuem veient que~\(f\) és exhaustiva.
        Prenem~\([h]\in H/(H\cap K)\).
        Per la definició \eqref{hipotesi:definicio-f-per-segon-teorea-de-lisomorfisme} tenim que~\(f(h\ast k)=[h]\) per a qualsevol~\(k\in K\), i per tant~\(f\) és exhaustiva.

        Per tant~\(f\) és un epimorfisme, i per tant, pel \myref{thm:Primer-Teorema-de-lisomorfisme-entre-grups} tenim
        \[
            HK/\ker(f)\cong H/(H\cap K).
        \]
        Ara bé, per la definició de \myref{def:nucli-dun-morfisme-entre-grups} tenim que~\(\ker(f)=\{h_{1}\ast k_{2}\in HK\mid f(h_{1}\ast k_{1})=[e]\}\), i per tant~\(\ker(f)=K\) i trobem
        \[
            HK/K\cong H/(H\cap K).\qedhere
        \]
    \end{proof}
    \begin{theorem}[Tercer Teorema de l'isomorfisme]
        \labelname{Tercer Teorema de l'isomorfisme entre grups}
        \label{thm:Tercer-Teorema-de-lisomorfisme-entre-grups}
        Siguin~\(G\) un grup i~\(H\),~\(K\) dos subgrups normals de~\(G\) amb~\(K\subseteq H\).
        Aleshores
        \[
            G/H\cong (G/K)/(H/K).
        \]
    \end{theorem}
    \begin{proof}
        Definim les aplicacions
        \begin{align*}
        \varphi_{1}\colon G&\longrightarrow G/K&\text{i}&&\varphi_{2}\colon G/K&\longrightarrow(G/K)/(H/K)\\
        g&\longmapsto[g]&&&[g]&\longmapsto\overline{[g]}.
        \end{align*}

        Veiem que~\(\varphi_{1}\) i~\(\varphi_{2}\) són morfismes.

        Per la proposició \myref{prop:conjugacio-de-morfismes-entre-grups-es-morfisme-entre-grups} tenim que~\(\varphi_{2}(\varphi_{1})\colon G\longrightarrow(G/K)/(H/K)\) és un epimorfisme entre grups, %fer referencies fonaments.
        i pel \myref{thm:Primer-Teorema-de-lisomorfisme-entre-grups} trobem
        \[
            G/\ker(\varphi_{2}(\varphi_{1}))\cong(G/K)/(H/K).
        \]

        Veiem ara que~\(\ker(\varphi_{2}(\varphi_{1}))=H\).
        Per la definició de \myref{def:relacio-dequivalencia-entre-grups} tenim que~\(G/K=\{gK\mid g\in G\}\) i~\(H/K=\{hK\mid h\in H\}\), i per tant
        \begin{equation}
        \label{eq:Tercer-Teorema-de-lisomorfisme-eq1}
        (G/K)/(H/K)=\{gKhK\mid g\in G, h\in H\},
        \end{equation}
        però com que, per hipòtesi,~\(H\) i~\(K\) són subgrups normals de~\(G\), per la definició de \myref{def:subgrup-normal} podem reescriure \eqref{eq:Tercer-Teorema-de-lisomorfisme-eq1} com

        \begin{equation}\label{eq:Tercer-Teorema-de-lisomorfisme-eq2}
        (G/K)/(H/K)=\{ghK\mid g\in G,h\in H\}.
        \end{equation}
        Ara bé, com que per hipòtesi~\(K\subseteq H\) podem reescriure \eqref{eq:Tercer-Teorema-de-lisomorfisme-eq2} com
        \[
            (G/K)/(H/K)=\{gH\mid g\in G\},
        \]
        i trobem, per la definició de \myref{def:nucli-dun-morfisme-entre-grups}, que~\(\ker(\varphi_{2}(\varphi_{1}))=H\), i per tant
        \[
            G/H\cong (G/K)/(H/K).\qedhere
        \]
    \end{proof}
\section{Tres Teoremes de Sylow}
    \subsection{Accions sobre grups}
    \begin{definition}[Acció d'un grup sobre un conjunt]
        \labelname{acció d'un grup sobre un conjunt}
        \label{def:accio-dun-grup-sobre-un-conjunt}
        Siguin~\(G\) un grup amb l'operació~\(\ast\) i element neutre~\(e\),~\(X\) un conjunt no buit i
        \begin{align*}
        \cdot\colon G\times X&\longrightarrow X\\
        (g,x)&\longmapsto g\cdot x
        \end{align*}
        una operació que satisfaci
        \begin{enumerate}
            \item~\(e\cdot x=x\) per a tot~\(x\in X\).
            \item~\((g_{1}\ast g_{2})\cdot x=g_{1}\cdot(g_{2}\cdot x)\) per a tot~\(x\in X\),~\(g_{1},g_{2}\in G\).
        \end{enumerate}
        Aleshores direm que~\(\cdot\) és una acció de~\(G\) sobre~\(X\).
        També direm que~\(X\) és un~\(G\)-conjunt amb l'acció~\(\cdot\).
    \end{definition}
    \begin{proposition}
        \label{prop:relacio-dorbites-es-dequivalencia}
        Siguin~\(X\) un~\(G\)-conjunt amb l'acció~\(\cdot\) i~\(\sim\) una relació sobre el conjunt~\(X\) tal que per a tot~\(x_{1},x_{2}\in X\) diem que~\(x_{1}\sim x_{2}\) si i només si existeix un~\(g\in G\) tal que~\(x_{1}=g\cdot x_{2}\).
        Aleshores la relació~\(\sim\) és una relació d'equivalència.
    \end{proposition}
    \begin{proof}
        Siguin~\(\ast\) l'operació de~\(G\) i~\(e\) l'element neutre del grup~\(G\).
        Comprovem que~\(\sim\) satisfà la definició de \myref{def:relacio-dequivalencia}:
        \begin{enumerate}
            \item Reflexiva: Sigui~\(x\in X\).
            Per la definició d'\myref{def:accio-dun-grup-sobre-un-conjunt} tenim que~\(x=x\cdot e\), i per tant~\(x\sim x\).
            \item Simètrica: Siguin~\(x_{1},x_{2}\in X\) tals que~\(x_{1}\sim x_{2}\).
            Per tant existeix~\(g\in G\) tal que~\(x_{1}=g\cdot x_{2}\).
            Per la definició d'\myref{def:accio-dun-grup-sobre-un-conjunt} tenim que~\(g\cdot x_{2}\in X\), i per tant podem prendre~\(g^{-2}\cdot(g\ast x_{2})\), que és equivalent a~\(g^{-2}\cdot(g\cdot x_{2})=g^{-2}\cdot x_{1}\), i així~\(x_{2}=g^{-1}\cdot x_{1}\), i per tant~\(x_{2}\sim x_{1}\).
            \item Transitiva: Siguin~\(x_{1},x_{2},x_{3}\in X\) tals que~\(x_{1}\sim x_{2}\) i~\(x_{2}\sim x_{3}\).
            Per tant existeixen~\(g_{1},g_{2}\in G\) tals que~\(x_{1}=g_{1}\cdot x_{2}\) i~\(x_{2}=g_{2}\cdot x_{3}\), i per tant~\(x_{1}=g_{1}\cdot(g_{2}\cdot x_{3})\), i per la definició d'\myref{def:accio-dun-grup-sobre-un-conjunt} això és~\(x_{1}=(g_{1}\ast g_{2})\cdot x_{3}\), i com que~\(G\) és un grup,~\(g_{1}\ast g_{2}\in G\), i tenim que~\(x_{1}\sim x_{3}\).
        \end{enumerate}
        per tant~\(\sim\) és una relació d'equivalència.
    \end{proof}
    \begin{definition}[Òrbita d'un element d'un~\(G\)-conjunt]
        \labelname{l'òrbita d'un element d'un \ensuremath{G}-conjunt}\label{def:orbita-dun-element-dun-G-conjunt}
        Siguin~\(X\) un~\(G\)-conjunt amb l'acció~\(\cdot\) i~\(\sim\) una relació d'equivalència sobre~\(X\) tal que per a tot~\(x_{1},x_{2}\in X\) diem que~\(x_{1}\sim x_{2}\) si i només si existeix un~\(g\in G\) tal que~\(x_{1}=g\cdot x_{2}\).
        Aleshores direm que~\(\mathcal{O}(x)=[x]\) és l'òrbita de~\(x\).

        Aquesta definició té sentit per la proposició \myref{prop:relacio-dorbites-es-dequivalencia}.
    \end{definition}
    \begin{definition}[Estabilitzador d'un element per una acció]
        \labelname{l'estabilitzador d'un element per una acció}
        \label{def:lestabilitzador-dun-element-per-una-accio}
        Siguin~\(X\) un~\(G\)-conjunt amb l'acció~\(\cdot\) i~\(x\) un element de~\(X\).
        Aleshores direm que el conjunt
        \[
            \St(x)=\{g\in G\mid g\cdot x=x\}
        \]
        és l'estabilitzador de~\(x\) per l'acció~\(\cdot\).
    \end{definition}
    \begin{proposition}
        \label{prop:linvers-dun-element-pertany-a-lestabilitzador-per-una-accio}
        Siguin~\(X\) un~\(G\)-conjunt amb l'acció~\(\cdot\) i~\(\St(x)\) l'estabilitzador d'un element~\(x\) de~\(X\) per l'acció~\(\cdot\).
        Aleshores~\(g\in\St(x)\) si i només si~\(g^{-1}\in\St(x)\).
    \end{proposition}
    \begin{proof}
        Sigui~\(\ast\) l'operació de~\(G\).
        Per la definició de \myref{def:lestabilitzador-dun-element-per-una-accio} tenim que~\(g\in\St(x)\) si i només si~\(g\cdot x=x\).
        Ara bé, si prenem~\(g^{-1}\cdot (g\cdot x)=g^{-1}\cdot x\), i per la definició d'\myref{def:accio-dun-grup-sobre-un-conjunt} tenim que~\(g^{-1}\cdot(g\cdot x)=(g^{-1}\ast g)\cdot x\), i per la definició de \myref{def:linvers-dun-element-dun-grup} tenim que~\(g^{-1}\cdot x\) i per tant~\(g^{-1}\in\St(x)\).
    \end{proof}
    \begin{proposition}
        \label{prop:lestabilitzador-es-un-subgrup}
        Siguin~\(X\) un~\(G\)-conjunt amb l'acció~\(\cdot\) i~\(\St(x)\) l'estabilitzador de~\(x\) per l'acció~\(\cdot\).
        Aleshores~\(\St(x)\) és un subgrup de~\(G\).
    \end{proposition}
    \begin{proof}
        Sigui~\(\ast\) l'operació de~\(G\).
        Per la proposició \myref{prop:condicio-equivalent-a-subgrup} només ens cal veure que per a tot~\(g_{1},g_{2}\in\St(x)\) es compleix~\(g_{1}\ast g_{2}^{-1}\in\St(x)\).

        Prenem doncs~\(g_{1},g_{2}\in\St(x)\).
        Per la proposició \myref{prop:linvers-dun-element-pertany-a-lestabilitzador-per-una-accio} tenim que~\(g_{2}^{-1}\in\St(x)\), i per tant, per la definició de \myref{def:lestabilitzador-dun-element-per-una-accio} tenim que~\((g_{1}\ast g_{2}^{-1})\cdot x=x\), i per tant~\(g_{1}\ast g_{2}^{-1}\in\St(x)\).
    \end{proof}
    \begin{proposition}
        \label{prop:cardinal-del-grup-dividit-per-cardinal-de-lestabilitzador-es-el-cardinal-de-lorbita}
        Siguin~\(G\) un grup d'ordre finit,~\(X\) un~\(G\)-conjunt finit amb una acció~\(\cdot\) i~\(\St(x)\) l'estabilitzador d'un element~\(x\) de~\(X\) per l'acció~\(\cdot\).
        Aleshores
        \[
            \abs{G/\St(x)}=\abs{\mathcal{O}(x)}=[G:\St(x)].
        \]
    \end{proposition}
    \begin{proof}
        Sigui~\(\ast\) l'operació de~\(G\).
        Considerem
        \begin{align*}
        f\colon G/\St(x)&\longrightarrow\mathcal{O}(x)\\
        [g]&\longmapsto g\cdot x
        \end{align*}
        Volem veure que~\(f\) és una aplicació bijectiva, per tant mirem si està ben definida:

        Siguin~\([g_{1}],[g_{2}]\in G/\St(x)\) tals que~\([g_{1}]=[g_{2}]\).
        Per tant~\(g_{1}=g_{2}\ast g'\) per a cert~\(g'\in\St(x)\), i per la definició de \myref{def:lestabilitzador-dun-element-per-una-accio} tenim~\(g_{1}\cdot x=x\),~\((g_{2}\ast g')\cdot x=x\), i per la definició d'\myref{def:accio-dun-grup-sobre-un-conjunt} això és~\(g_{2}\cdot(g'\cdot x)=x\), i per la proposició \myref{prop:linvers-dun-element-pertany-a-lestabilitzador-per-una-accio}~\(g_{2}\cdot x=x\), i per tant~\(f\) està ben definida.

        Veiem ara que~\(f\) és injectiva.
        Prenem~\(g\cdot x,g'\cdot x\in\mathcal{O}(x)\) tals que~\(g\cdot x=g'\cdot x\).
        Això, per la definició d'\myref{def:accio-dun-grup-sobre-un-conjunt}, és equivalent a dir~\(x=(g^{-1}\ast g')\cdot x\), i per la definició de \myref{def:lestabilitzador-dun-element-per-una-accio} tenim que~\(g^{-1}\ast g\in\St(x)\).
        Per tant, per la definició de \myref{def:relacio-dequivalencia-entre-grups} tenim que~\([g]=[g']\), i per tant~\(f\) és injectiva.

        Per veure que~\(f\) és exhaustiva veiem que per a qualsevol~\(g\cdot x\in\mathcal{O}(x)\),~\(f([g])=g\cdot x\), i per tant~\(f\) és exhaustiva i tenim que~\(\abs{\St(x)}=\abs{\mathcal{O}(x)}\), ja que per la definició d'\myref{def:aplicacio-bijectiva} tenim que~\(f\) és una bijecció.
        %REFerencies fonaments! Cardinalitat
    \end{proof}
    \subsection{Teoremes de Sylow}
    \begin{definition}[\(p\)-subgrup de Sylow]
        \labelname{\ensuremath{p}-subgrup de Sylow}
        \label{def:p-subgrup-de-Sylow}
        Siguin~\(G\) un grup tal que~\(\abs{G}=p^{n}m\) amb~\(p\) primer que no divideix~\(m\) i~\(P\) un subgrup de~\(G\) amb~\(\abs{P}=p^{n}\).
        Aleshores direm que~\(P\) és un~\(p\)-subgrup de Sylow de~\(G\).
    \end{definition}
    \begin{lemma}
        \label{lema:Primer-Teorema-de-Sylow}
        Siguin~\(p\) un primer i~\(m\) un enter positiu tal que~\(p\) no divideixi~\(m\).
        Aleshores per a tot~\(n\) natural tenim que
        \begin{equation}
            \label{eq:lema:Primer-Teorema-de-Sylow-0}
            \binom{p^{n}m}{p^{n}}
        \end{equation}
        no és divisible per~\(p\).
    \end{lemma}
    \begin{proof}
        Tenim que
        \begin{equation}
            \label{eq:lema:Primer-Teorema-de-Sylow-1}
            \binom{p^{n}m}{p^{n}}
            =\frac{p^{n}m(p^{n}m-1)\cdots(p^{n}m-p^{n}+1)}{p^{n}(p^{n}-1)\cdots(p^{n}-p^{n}+1)}=\prod_{i=0}^{p^{n}-1}\frac{p^{n}m-i}{p^{n}-i}.
        \end{equation}
        Com que, per hipòtesi,~\(p\) és primer aquesta expressió només serà divisible per~\(p\) si ho són els elements del numerador.
        Fixem doncs~\(i\in\{0,\dots,p^{n}+1\}\) i estudiem l'\(i\)-èsim terme del producte de \eqref{eq:lema:Primer-Teorema-de-Sylow-1}.
        Notem primer que si~\(i=0\) aquest terme no és divisible per~\(p\).
        Imposem ara també que~\(i\neq0\).
        Si el denominador,~\(p^{n}m-i\), és divisible per~\(p\) tindrem que~\(p^{n}m-i=p^{k}m'\), on~\(m'\) no és divisible per~\(p\), i per tant~\(k\) serà l'exponent més gran que satisfaci la igualtat.
        Si aïllem~\(i\) obtindrem~\(i=p^{k}(p^{n-k}m-m')\).
        Ara bé, tenim doncs que~\(p^{n}-i=p^{n}-p^{k}(p^{n-k}m-m')\), i això és~\(p^{n}-p^{k}(p^{n-k}m-m')=p^{k}(p^{n-k}(1-m)-m')\), i per tant tindrem que l'\(i\)-èsim terme del producte de \eqref{eq:lema:Primer-Teorema-de-Sylow-1} serà de la forma
        \[
            \frac{p^{n}m-i}{p^{n}-i}=\frac{p^{k}m'}{p^{k}(p^{n-k}(1-m)-m')}=\frac{m'}{p^{n-k}(1-m)-m'},
        \]
        i així veiem que aquest~\(i\)-èsim terme del producte no serà divisible per~\(p\); i com que això és cert per a qualsevol~\(i\in\{0,\dots,p^{n}+1\}\) tenim que \eqref{eq:lema:Primer-Teorema-de-Sylow-0} tampoc ho serà, com volíem veure.
    \end{proof}
    \begin{theorem}[Primer Teorema de Sylow]
        \labelname{Primer Teorema de Sylow}
        \label{thm:Primer-Teorema-de-Sylow}
        Siguin~\(G\) un grup tal que~\(\abs{G}=p^{n}m\) amb~\(p\) primer que no divideix~\(m\).
        Aleshores existeix un subconjunt~\(P\subseteq G\) tal que~\(P\) sigui un~\(p\)-subgrup de Sylow de~\(G\).
    \end{theorem}
    \begin{proof} % Demostració de Wielandt
        Sigui~\(\mathcal{P}_{p^{n}}(G)=\{H\subseteq G\mid\lvert H\rvert=p^{n}\}=\{H_{1},\dots,H_{k}\}\) el conjunt de subconjunts d'ordre~\(p^{n}\) de~\(G\).
        Aleshores tenim que
        \begin{displaymath}
        k=\abs{\mathcal{P}_{p^{n}}(G)}=\binom{p^{n}m}{p^{n}},
        \end{displaymath}
        i pel lema \myref{lema:Primer-Teorema-de-Sylow} tenim que~\(p\) no divideix~\(k\).
        %Fer això a fonaments, definint \mathcal{P}_{n}(G)

        Siguin~\(\ast\) l'operació de~\(G\) i~\(e\) l'element neutre de~\(G\).
        Definim
        \begin{align}
            \label{eq:thm:Primer-Teorema-de-Sylow-1}
            \cdot\colon G\times\mathcal{P}_{p^{n}}(G)&\longrightarrow \mathcal{P}_{p^{n}}(G)\\
            (g,X)&\longmapsto\{g\}X.\nonumber
        \end{align}
        Veurem que~\(\cdot\) és una acció de~\(G\) sobre~\(\mathcal{P}_{p^{n}}(G)\).
        Primer hem de veure que, efectivament,~\(\cdot\) està ben definida.
        Prenem~\(g_{1},g_{2}\in G\) i~\(X_{1},X_{2}\in \mathcal{P}_{p^{n}}(G)\) tals que~\(g_{1}=g_{2}\) i~\(X_{1}=X_{2}\).
        Per tant tenim~\(g_{1}\cdot X_{1}=g_{2}\cdot X_{2}\) ja que~\(\{g_{1}\}X_{1}=\{g_{2}\}X_{2}\).
        Per veure que~\(g_{1}\cdot X_{1}\in\mathcal{P}_{p^{n}}(G)\) en tenim prou amb veure que, per a tot~\(X\in\mathcal{P}_{p^{n}}(G)\), si fixem~\(g\in G\) l'aplicació~\(g\cdot X\) té inversa, que per la definició de \myref{def:linvers-dun-element-dun-grup} és~\(x^{-1}\cdot X\), i per tant~\(X\in\mathcal{P}_{p^{n}}(G)\).

        Comprovem que~\(\cdot\) satisfà la definició d'\myref{def:accio-dun-grup-sobre-un-conjunt}.
        Tenim que~\(e\cdot X=X\) per a tot~\(X\in\mathcal{P}_{p^{n}}(G)\) ja que, per la definició de \myref{def:conjugacio-entre-conjunts-sobre-grups},~\(eX=X\).

        De nou per la definició de \myref{def:conjugacio-entre-conjunts-sobre-grups} veiem que per a tot~\(g_{1},g_{2}\in G\) i~\(X\in\mathcal{P}_{p^{n}}(G)\) tenim que \begin{align*}
        (g_{1}\ast g_{2})\cdot X&=\{g_{1}\ast g_{2}\}X\\
        &=\{g_{1}\}\{g_{2}\}X\\
        &=\{g_{1}\}(\{g_{2}\}X)=g_{1}\cdot(g_{2}\cdot X).
        \end{align*}
        i per tant~\(\cdot\) satisfà la definició d'\myref{def:accio-dun-grup-sobre-un-conjunt}.

        Veiem ara que existeix almenys un element~\(X\in\mathcal{P}_{p^{n}}(G)\) tal que la seva òrbita,~\(\mathcal{O}(X)\), tingui ordre no divisible per~\(p\).
        Per veure això observem que per la definició de \myref{def:orbita-dun-element-dun-G-conjunt} veiem que~\(\mathcal{O}(X)\) és un classe d'equivalència, i per tant l'ordre del conjunt~\(\mathcal{P}_{p^{n}}\) és la suma dels ordres de les òrbites dels seus elements,~\(\mathcal{O}(X)\), i si~\(p\) dividís l'ordre de~\(\mathcal{O}(X)\) per a tot~\(X\in\mathcal{P}_{p^{n}}\) tindríem que~\(p\) també divideix~\(k\), però ja hem vist que això no pot ser.
        Per tant existeix almenys un element~\(X\in\mathcal{P}_{p^{n}}\) tal que~\(\abs{\mathcal{O}(X)}\) no és divisible per~\(p\).
        Fixem aquest conjunt~\(X\).

        Prenem l'estabilitzador de~\(X\),~\(\St(X)\).
        Per la proposició \myref{prop:cardinal-del-grup-dividit-per-cardinal-de-lestabilitzador-es-el-cardinal-de-lorbita} tenim que~\(\abs{\St(X)}\) divideix~\(p^{n}\).
        Prenem també~\(x_{o}\in X\) i~\(g\in\St(X)\).
        Per la definició de \myref{def:lestabilitzador-dun-element-per-una-accio} tenim que~\(\{g\}X=X\), i per tant~\(g\ast x_{0}\in X\), i equivalentment~\(g\in X\{x_{0}^{-1}\}\).
        Així veiem que~\(\St(X)\subseteq X\{x_{0}^{-1}\}\), i per tant tenim que~\(\abs{\St(X)}\leq\abs{X\{x_{0}\}}\).
        Observem que~\(X\{x_{0}\}\in\mathcal{P}_{p^{n}}(G)\) i per tant~\(\abs{X\{x_{0}\}}=p^{n}\).
        Ara bé, l'ordre de~\(\St(X)\) divideix~\(p^{n}\), però acabem de veure que~\(\abs{\St(X)}\leq p^{n}\).
        Per tant ha de ser~\(\abs{\St(X)}=p^{n}\), i per tant, per la proposició \myref{prop:lestabilitzador-es-un-subgrup} tenim que~\(\St(X)\leq G\), i per tant, per la definició de \myref{def:p-subgrup-de-Sylow},~\(\St(X)\) és un~\(p\)-subgrup de Sylow.
    \end{proof}
    \begin{corollary}[Teorema de Cauchy per grups]
        \labelname{Teorema de Cauchy per grups}
        \label{thm:Teorema-de-Cauchy-per-grups}
        Siguin~\(G\) un grup d'ordre finit i~\(p\) un primer que divideix l'ordre de~\(G\).
        Aleshores existeix un element~\(g\in G\) tal que l'ordre de~\(g\) sigui~\(p\).
    \end{corollary}
    \begin{proof}
        Sigui~\(e\) l'element neutre de~\(G\).
        Pel \myref{thm:Primer-Teorema-de-Sylow} tenim que existeix un~\(p\)-subgrup de Sylow~\(P\) de~\(G\), que per la definició de \myref{def:p-subgrup-de-Sylow} té ordre~\(p^{n}\) per a cert~\(n\in\mathbb{N}\).
        Ara bé, pel \myref{thm:Teorema-de-Lagrange} tenim que per a tot~\(x\in P\) diferent del neutre el grup cíclic generar per~\(x\) ha de tenir ordre~\(p^{t}\) amb~\(t\in\{2,\dots,n\}\), i per tant l'element~\(x^{p^{t-1}}\) té ordre~\(p\), ja que
        \[
            \left(x^{p^{t-1}}\right)^{p}=x^{p^{t}}=e.\qedhere
        \]
    \end{proof}
    \begin{lemma}
        \label{lema:Segon-Teorema-de-Sylow}
        Siguin~\(G\) un grup d'ordre~\(p^{n}\) on~\(p\) és un primer,~\(X\) un~\(G\)-conjunt finit amb una acció~\(\cdot\) i
        \[
            X_{G}=\{x\in X\mid g\cdot x=x\text{ per a tot }g\in G\}
        \]
        un conjunt.
        Aleshores
        \[
            \abs{X_{G}}\equiv\abs{X}\pmod{p}.
        \]
    \end{lemma}
    \begin{proof}
        Siguin~\(\mathcal{O}(x_{1}),\dots,\mathcal{O}(x_{r})\) les òrbites dels elements de~\(X\).
        Aleshores, com que per la definició de \myref{def:orbita-dun-element-dun-G-conjunt} aquestes són classes d'equivalència, tenim que %REFERENCIES fonaments
        \[
            X=\bigcup_{i=1}^{r}\mathcal{O}(x_{i}),
        \]
        i com que aquestes òrbites són disjuntes per ser classes d'equivalència
        \begin{equation}
    \label{eq:lema:Segon-Teorema-de-Sylow-1}
        \abs{X}=\sum_{i=0}^{r}\abs{\mathcal{O}(x_{i})}.
        \end{equation}
        Ara bé, per les proposicions \myref{prop:cardinal-del-grup-dividit-per-cardinal-de-lestabilitzador-es-el-cardinal-de-lorbita} i \myref{prop:lestabilitzador-es-un-subgrup} i el \myref{thm:Teorema-de-Lagrange} tenim que l'ordre~\(\mathcal{O}(x_{i})\) divideix l'ordre de~\(X\), i per tant els únics elements amb òrbites que tinguin un ordre que no sigui divisible per~\(p\) són els elements del conjunt~\(X_{G}\); i com que les òrbites d'aquests elements tenen un únic element tenim que
        \[
            \abs{X_{G}}=\sum_{x\in X_{G}}\abs{\mathcal{O}(x)},
        \]
        i per tant, recordant que totes les altres òrbites tenen ordre divisible per~\(p\), trobem, amb \eqref{eq:lema:Segon-Teorema-de-Sylow-1}, que
        \[
            \abs{X_{G}}\equiv\abs{X}\pmod{p}.\qedhere
        \]
    \end{proof}
    \begin{theorem}[Segon Teorema de Sylow]
        \labelname{Segon Teorema de Sylow}
        \label{thm:Segon-Teorema-de-Sylow}
        Siguin~\(G\) un grup d'ordre finit,~\(p\) un primer que divideixi l'ordre de~\(G\) i~\(P_{1}\),~\(P_{2}\) dos~\(p\)-subgrups de Sylow de~\(G\).
        Aleshores existeix~\(g\in G\) tal que
        \[
            \{g\}P_{1}\{g^{-1}\}=P_{2}.
        \]
    \end{theorem}
    \begin{proof}
        Observem que aquest enunciat té sentit pel \myref{thm:Primer-Teorema-de-Sylow}.

        Definim el conjunt~\(X=\{\{x\}P_{1}\mid x\in G\}\) i
        \begin{align}\label{eq:thm:Segon-Teorema-de-Sylow-1}
        \cdot\colon P_{2}\times X&\longrightarrow X\\
        (y,\{x\}P_{1})&\longmapsto \{y\}\{x\}P_{1}.\nonumber
        \end{align}

        Primer veurem que~\(\cdot\) és una acció.
        Per veure que~\(\cdot\) està ben definida prenem~\(\{x\}P_{1},\{x'\}P_{1}\in X\) tals que~\(\{x\}P_{1}=\{x'\}P_{1}\).
         Aleshores per a tot~\(y\in P_{2}\) tindrem~\(y\cdot\{x\}P_{1}=\{y\}\{x\}P_{1}\) i~\(y\cdot\{x'\}P_{1}=\{y\}\{x'\}P_{1}\), i com que per hipòtesi~\(\{x\}P_{1}=\{x'\}P_{1}\), ha de ser~\(\{y\}\{x\}P_{1}=\{y\}\{x'\}P_{1}\).

        Siguin~\(\ast\) l'operació de~\(G\) i~\(e\) l'element neutre de~\(G\).
        Comprovem que~\(\cdot\) satisfà la definició d'\myref{def:accio-dun-grup-sobre-un-conjunt}.
        Veiem que per a tot~\(y\in P_{2}\),~\(\{x\}P_{1}\in X\) es compleix que~\(y\cdot\{x\}P_{1}\in X\).
        Per la definició \eqref{eq:thm:Segon-Teorema-de-Sylow-1} tenim~\(y\cdot\{x\}P_{1}\in X=\{y\}\{x\}P_{1}=\{y\ast x\}P_{1}\), i com que per hipòtesi~\(G\) és un grup i~\(x,y\in G\), per la definició de \myref{def:grup} tenim que~\(y\ast x\in G\), i per tant~\(y\cdot\{x\}P_{1}\in X\).

        Tenim que
        \begin{align*}
        e\cdot \{x\}P_{1}&=\{e\}\{x\}P_{1}\\
        &=\{e\ast x\}P_{1}=\{x\}P_{1}.\tag{\myref{def:lelement-neutre-del-grup}}
        \end{align*}
        i per últim veiem que per a tot~\(y,y'\in G\) tenim~\((y\ast y')\cdot P_{1}=y\cdot(y'\cdot P_{1})\).
        Això és
        \begin{align*}
        (y\ast y')\cdot P_{1}&=\{y\ast y'\}P_{1}\\
        &=\{y\}\{y'\}P_{1}\\
        &=\{y\}(\{y'\}P_{1})=y\cdot(y'\cdot P_{1}).\tag{Definició \eqref{eq:thm:Segon-Teorema-de-Sylow-1}}
        \end{align*}
        i per la definició d'\myref{def:accio-dun-grup-sobre-un-conjunt} tenim que~\(X\) és un~\(G\)-conjunt.

        Definim el conjunt
        \begin{equation}
    \label{eq:thm:Segon-Teorema-de-Sylow-2}
        X_{P_{2}}=\{\{x\}P_{1}\in X\mid y\cdot\{x\}P_{1}=\{x\}P_{1}\text{ per a tot }y\in P_{2}\}.
        \end{equation}
        Aleshores pel lema \myref{lema:Segon-Teorema-de-Sylow} tenim que
        \begin{equation*}
        \abs{X_{P_{2}}}\equiv\abs{X}\pmod{p}.
        \end{equation*}
        Ara bé, per la definició de \myref{def:lindex-dun-subgrup-en-un-grup} i \eqref{eq:thm:Segon-Teorema-de-Sylow-1} tenim que~\(\abs{X}=[G:P_{1}]\), i per hipòtesi tenim que~\(\abs{X}=\frac{p^{n}m}{p^{n}}=m\), en particular~\(\abs{X_{P_{2}}}\neq0\).
        Així veiem que existeix almenys un element que satisfà la definició de~\(X_{P_{2}}\), \eqref{eq:thm:Segon-Teorema-de-Sylow-2}; és a dir, existeix almenys un~\(\{x\}P_{1}\) tal que~\(y\cdot\{x\}P_{1}=\{x\}P_{1}\) per a tot~\(y\in P_{2}\), i per tant tenim que~\(\{y\}\{x\}P_{1}=\{x\}P_{1}\), i per tant~\(\{x^{-1}\}\{y\}\{x\}P_{1}\in \{x^{-1}\}\{x\}P_{1}\), i equivalentment~\(x^{-1}\ast y\ast x\in P_{1}\) per a tot~\(y\in P_{2}\), i per tant~\(\{x^{-1}\}P_{2}\{x\}\subseteq P_{1}\), però, per hipòtesi, al ser els dos~\(p\)-subgrups de Sylow, per la definició de \myref{def:p-subgrup-de-Sylow} trobem~\(\abs{P_{1}}=\abs{P_{2}}=\abs{\{x^{-1}\}P_{2}\{x\}}\) i tenim que~\(\{x\}P_{1}\{x^{-1}\}=P_{2}\).
    \end{proof}
    \begin{corollary}
        \label{corollary:Segon-Teorema-de-Sylow} %TODO FER revisar doble implicació
        Un grup~\(G\) finit té un únic~\(p\)-subgrup de Sylow si i només si aquest és un subgrup normal.
    \end{corollary}
    \begin{theorem}[Tercer Teorema de Sylow]
        \labelname{Tercer Teorema de Sylow}
        \label{thm:Tercer-Teorema-de-Sylow}
        Siguin~\(G\) un grup d'ordre~\(p^{n}m\) on~\(p\) és un primer que no divideix~\(m\) i~\(n_{p}\) el número de~\(p\)-subgrups de Sylow de~\(G\).
        Aleshores
        \(n_{p}\equiv1\pmod{p}\) i~\(n_{p}\) divideix l'ordre de~\(G\).
    \end{theorem}
    \begin{proof}
        Definim el conjunt
        \[
            X=\{T\subseteq G\mid T\text{ és un }p\text{-subgrup de Sylow de }G\}.
        \]
        Pel \myref{thm:Primer-Teorema-de-Sylow} tenim que~\(X\) és no buit\footnote{Tindrem que~\(\abs{X}=n_{p}\).} i fixem~\(P\in X\).

        Definim
        \begin{align}
            \label{eq:thm:Tercer-Teorema-de-Sylow-1}
            \cdot\colon P\times X&\longrightarrow X\\
            (g,T)&\longmapsto\{g\}T\{g^{-1}\}.\nonumber
        \end{align}
        Siguin~\(\ast\) l'operació de~\(G\) i~\(e\) l'element neutre de~\(G\).
        Anem a veure que~\(\cdot\) és una acció.
        Veiem que~\(\cdot\) està ben definida, ja que si~\(T\in X\), aleshores per a tot~\(x\in G\), i en particular per a tot~\(x\in P\) ja que~\(P\leq G\), tenim~\(\abs{\{x\}T\{x^{-1}\}}=\abs{T}\), i per tant~\(\abs{\{x\}T\{x^{-1}\}}\in X\) per la definició de \myref{def:p-subgrup-de-Sylow}.
        Veiem ara que~\(\cdot\) satisfà les condicions de la definició d'\myref{def:accio-dun-grup-sobre-un-conjunt}.
        Sigui~\(e\) l'element neutre de~\(G\).
        Tenim que per a tot~\(T\in X\)
        \begin{align*}
            e\cdot T&=\{e\}T\{e^{-1}\}\tag{Definició \eqref{eq:thm:Tercer-Teorema-de-Sylow-1}}\\
            &=T
        \end{align*}
        i per a tot~\(g_{1},g_{2}\in P\) i~\(T\in X\)
        \begin{align*}
            (g_{1}\ast g_{2})\cdot T&=\{g_{1}\ast g_{2}\}T\{{g_{1}\ast g_{2}}^{-1}\}            \tag{Definició \eqref{eq:thm:Tercer-Teorema-de-Sylow-1}}\\
            &=\{g_{1}\ast g_{2}\}T\{g_{2}^{-1}\ast g_{1}^{-1}\}\tag{Proposició \myref{prop:invers-de-a-b-b-invers-a-invers}}\\
            &=\{g_{1}\}\{g_{2}\}T\{g_{2}^{-1}\}\{g_{1}^{-1}\}\\
            &=\{g_{1}\}(g_{2}\cdot T)\{g_{1}^{-1}\}\tag{Definició \eqref{eq:thm:Tercer-Teorema-de-Sylow-1}}\\
            &=g_{1}\cdot(g_{2}\cdot T)\tag{Definició \eqref{eq:thm:Tercer-Teorema-de-Sylow-1}}
        \end{align*}
        i per tant, per la definició d'\myref{def:accio-dun-grup-sobre-un-conjunt}~\(X\) és un~\(P\)-conjunt amb l'acció~\(\cdot\).

        Definim el conjunt
        \[
            X_{P}=\{T\in X\mid g\cdot T=T\text{ per a tot }g\in P\},
        \]
        i per la definició \eqref{eq:thm:Tercer-Teorema-de-Sylow-1} tenim que si~\(T\in X_{P}\) aleshores per a tot~\(g\in G\) es compleix~\(\{g\}T\{g^{-1}\}=T\).
        Ara bé, això és que~\(T=P\) per a tot~\(T\in X_{P}\), i per tant~\(\abs{X_{P}}=1\).
        Aleshores pel lema \myref{lema:Segon-Teorema-de-Sylow} tenim que
        \[
            \abs{X}\equiv\abs{X_{P}}\pmod{p},
        \]
        o equivalentment
        \[
            \abs{X}\equiv1\pmod{p}.
        \]

        Per veure que~\(\abs{X}\) divideix l'ordre de~\(G\) prenem~\(P\in X\) i tenim, pel \myref{thm:Segon-Teorema-de-Sylow} i la definició de \myref{def:orbita-dun-element-dun-G-conjunt}, que
        \[
            \mathcal{O}(P)=X,
        \]
        on~\(\mathcal{O}(P)\) és l'òrbita de~\(P\), i per tant
        \[
            \abs{\mathcal{O}(P)}=\abs{X},
        \]
        i per les proposicions \myref{prop:lestabilitzador-es-un-subgrup} i \myref{prop:cardinal-del-grup-dividit-per-cardinal-de-lestabilitzador-es-el-cardinal-de-lorbita} i el \myref{thm:Teorema-de-Lagrange} tenim que~\(\abs{X}\) divideix l'ordre de~\(G\).
    \end{proof}
    \begin{corollary}
        Si~\(G\) té ordre~\(p^{n}q^{m}\) on~\(p,q\) són primers amb~\(p<q\) aleshores~\(n_{q}=1\), i pel corol·lari \myref{corollary:Segon-Teorema-de-Sylow}, el~\(p\)-subgrup de Sylow de~\(G\) és un subgrup normal de~\(G\).
    \end{corollary}
\end{document}
