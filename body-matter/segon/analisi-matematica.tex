\documentclass[../segon.tex]{subfiles}

\begin{document}
\part{Anàlisi matemàtica}
\subfile{./analisi-matematica/1-series.tex}
\subfile{./analisi-matematica/2-integrals-impropies.tex}
\subfile{./analisi-matematica/3-series-de-Fourier.tex}
\printbibliography
Els apunts estan escrits seguint la teoria donada a classe i complementats amb \cite{ApuntsMorelo}.
La secció de reordenació de sèries està fortament inspirada també en \cite{HickmanRiemannSeriesTheoremNotes}.
He copiat un exemple de \cite{KeithDifferentiatingUnderIntegralSignNotes}.

La bibliografia del curs inclou els textos \cite{GalindoGuiaPracticaCalculoInfinitesimal,OrtegaIntroduccioAnalisiMatematica,PerelloCalculInfinitesimal,RudinPrinciplesOfMathematicalAnalysis,TolstovFourier}.
\end{document}

% Reordenació de sèries https://math.uchicago.edu/~j.e.hickman/163%20Lecture%20notes/Lecture%207%20and%208.pdf
% Útil en general http://math.uchicago.edu/~j.e.hickman/math163
% Sèries de funcions amb Perelló (Càlcul infinitesimal)
% Encara no se d'on treure integrals impròpies
% Fourier amb Tolstov (Fourier Series)

% sin(arccos(f(x))), utilitzar sin(x)=sqrt(1+cos(f(x))^2)
% ens queda sqrt(1+f(x)^2)

% http://www.math.uconn.edu/~kconrad/blurbs/analysis/diffunderint.pdf lol

% Exemple: \[\varphi=1+2\sin\left(\frac{\pi}{10}\right)\] % 2cos(pi/5) = 1.618...
