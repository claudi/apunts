\documentclass[../probabilitat.tex]{subfiles}

\begin{document}
\chapter{Variables i vectors aleatoris}
\section{Distribució i funció de distribució d'una variable aleatòria}
\subsection{Variables aleatòries}
    \begin{definition}[Variable aleatòria]
        \labelname{variable aleatòria}\label{def:variable-aleatoria}
        Siguin~\(\Algebra{A}\) una~\(\sigma\)-àlgebra sobre un conjunt~\(\Omega\) i~\(X\colon\Omega\longrightarrow\RR\) una aplicació tal que per a tot conjunt~\(B\in\borel(\RR)\) tenim
        \[
            X^{-1}(B)\in\Algebra{A}.
        \]
        Aleshores direm que~\(X\) és una variable aleatòria sobre~\(\Algebra{A}\).
    \end{definition}
    \begin{example}
        %TODO
    \end{example}
    \begin{notation}
        \label{notation:variables-aleatories}
        Sigui~\(X\colon\Omega\longrightarrow\RR\) una variable aleatòria sobre un~\(\sigma\)-àlgebra~\(\Algebra{A}\).
        Aleshores denotem
        \begin{enumerate}
            \item~\(\{X\in B\}=\{\omega\in\Omega\mid X(\omega)\in B\}\).
            \item~\(\{X=a\}=\{X\in\{a\}\}\).
            \item~\(\{a< X<b\}=\{X\in(a,b)\}\).
            \item~\(\{a\leq X<b\}=\{X\in[a,b)\}\).
            \item~\(\{a< X\leq b\}=\{X\in(a,b]\}\).
            \item~\(\{a\leq X\leq b\}=\{X\in[a,b]\}\).
        \end{enumerate}
    \end{notation}
    \begin{proposition}
        \label{prop:condicio-suficient-per-ser-una-variable-aleatoria}
        Siguin~\(\Algebra{A}\) una~\(\sigma\)-àlgebra sobre un conjunt~\(\Omega\) i~\(X\colon\Omega\longrightarrow\RR\) una aplicació tal que per a tot~\(x\in\RR\) tenim que
        \[
            X^{-1}((-\infty,x])\in\Algebra{A}.
        \]
        Aleshores~\(X\) és una variable aleatòria.
    \end{proposition}
    \begin{proof}
        %TODO
    \end{proof}
    \begin{proposition}
        \label{prop:les-variables-aleatories-formen-un-anell}
        Siguin~\(X\) i~\(Y\) dues variables aleatòries i~\(\lambda\in\mathbb{R}\) un escalar.
        Aleshores~\(X+y\),~\(XY\) i~\(\lambda X\) són variables aleatòries.
    \end{proposition}
    \begin{proof}
        %TODO
    \end{proof}
    \begin{proposition}
        \label{prop:les-variables-aleatories-formen-un-cos}
        Sigui~\(X\colon\Omega\longrightarrow\RR\) una variable aleatòria sobre una~\(\sigma\)-àlgebra tal que per a tot~\(\omega\in\Omega\) tenim que~\(X(\omega)\neq0\).
        Aleshores~\(1/X\) és una variable aleatòria.
    \end{proposition}
    \begin{proof}
        %TODO
    \end{proof}
    \begin{proposition}
        \label{prop:el-cos-de-variables-aleatories-es-tancat}
        Sigui~\((X_{n})_{n\in\NN}\) una successió de variables aleatòries sobre una~\(\sigma\)-àlgebra sobre un conjunt~\(\Omega\) tals que per a tot~\(\omega\in\Omega\) tenim que la successió~\((X_{n}(\omega))_{n\in\NN}\) és convergent.
        Aleshores la funció
        \begin{align*}
            X\colon\Omega&\longrightarrow\mathbb{R} \\
            \omega&\longmapsto\lim_{n\to\infty}X_{n}(\omega)
        \end{align*}
        és una variable aleatòria.
    \end{proposition}
    \begin{proof}
        %TODO
    \end{proof}
\subsection{Distribució o llei d'una variable aleatòria}
    \begin{proposition}
        \label{prop:distribucio-duna-variable-aleatoria}
        Siguin~\(X\colon\Omega\longrightarrow\RR\) una variable aleatòria i
        \begin{align*}
            \Probabilitat_{X}\colon\borel(\RR)&\longrightarrow\RR \\
            B&\longmapsto\Probabilitat(X^{-1}(B)).
        \end{align*}
        Aleshores~\((\RR,\borel(\RR),\Probabilitat_{X})\) és un espai de probabilitat.
    \end{proposition}
    \begin{proof}
        %TODO
    \end{proof}
    \begin{definition}[Distribució d'una variable aleatòria]
        \labelname{distribució d'una variable aleatòria}\label{def:distribucio-duna-variable-aleatoria}
        Sigui~\(X\colon\Omega\longrightarrow\RR\) una variable aleatòria.
        Aleshores direm que la funció
        \begin{align*}
            \Probabilitat_{X}\colon\borel(\RR)&\longrightarrow\RR \\
            B&\longmapsto\Probabilitat(X^{-1}(B))
        \end{align*}
        és la distribució de~\(X\).
        També direm que és la llei de~\(X\).
    \end{definition}
    \begin{example}
        %TODO
    \end{example}
\subsection{Igualtat i igualtat quasi segura de variables aleatòries}
    \begin{definition}[Igualtat de variables aleatòries]
        \labelname{variables aleatòries amb la mateixa distribució}\label{def:variables-aleatories-amb-la-mateixa-distribucio}
        \labelname{igualtat de variables aleatòries}\label{def:igualtat-de-variables-aleatories}
        Siguin~\(X\) i~\(Y\) dues variables aleatòries tals que per a tot~\(B\in\borel(\RR)\) tenim que
        \[
            \Probabilitat_{X}(B)=\Probabilitat_{Y}(B).
        \]
        Aleshores direm que~\(X\) i~\(Y\) tenen la mateixa distribució, i ho denotarem com
        \[
            X\igualsendistribucio Y.
        \]
    \end{definition}
    \begin{example}
        %TODO
    \end{example}
    \begin{definition}[Igualtat quasi segura de variables aleatòries]
        \labelname{igualtat quasi segura de variables aleatòries}\label{def:igualtat-quasi-segura-de-variables-aleatories}
        Siguin~\(X\) i~\(Y\) dues variables aleatòries sobre una~\(\sigma\)-àlgebra~\(\Algebra{A}\) tals que~\(\Probabilitat\{X=Y\}=1\).
        Aleshores direm que~\(X\) i~\(Y\) són iguals quasi segurament i ho denotarem com
        \[
            X\qs Y.
        \]
    \end{definition}
    \begin{observation}
        \label{obs:dues-variables-iguals-quasi-segurament-son-iguals-en-distribucio}
        Siguin~\(X\) i~\(Y\) dues variables aleatòries tals que~\(X\qs Y\).
        Aleshores~\(X\igualsendistribucio Y\).
    \end{observation}
    \begin{example}
        %TODO
    \end{example}
\subsection{Funció de distribució d'una variable aleatòria}
    \begin{definition}[Funció de distribució]
        \labelname{funció de distribució d'una variable aleatòria}\label{def:funcio-de-distribucio-duna-variable-aleatoria}
        Siguin~\(X\) una variable aleatòria i
        \begin{align*}
            F\colon\RR&\longrightarrow[0,1] \\
            x&\longmapsto\Probabilitat\{X\leq x    \}
        \end{align*}
        una funció.
        Aleshores direm que~\(F\) és la funció de distribució de~\(X\).
    \end{definition}
    \begin{proposition}
        \label{prop:les-funcions-de-distribucio-son-creixents}
        Sigui~\(F\) una funció de distribució.
        Aleshores~\(F\) és creixent.
    \end{proposition}
    \begin{proof}
        %TODO
    \end{proof}
    \begin{proposition} %TODO REVISAR
        \label{prop:les-funcions-de-distribucio-son-continues-per-la-dreta}
        \label{prop:les-funcions-de-distribucio-tenen-limit-per-lesquerra}
        Sigui~\(F\) una funció de distribució.
        Aleshores per a tot~\(x\in\RR\) tenim que
        \[
            \lim_{\substack{t\to x\\t<x}}F(t)=F(x)\qquad\text{i}\qquad\lim_{\substack{t\to x\\t>x}}F(t)=L\in\RR.
        \]
    \end{proposition}
    \begin{proof}
        %TODO
    \end{proof}
    \begin{proposition}
        \label{prop:limits-laterals-duna-funcio-de-distribucio}
        Sigui~\(F\) una funció de distribució.
        Aleshores
        \[
            \lim_{x\to-\infty}F(x)=0\qquad\text{i}\qquad\lim_{x\to\infty}F(x)=1.
        \]
    \end{proposition}
    \begin{proof}
        %TODO
    \end{proof}
    \begin{proposition}
        \label{prop:les-funcions-de-distribucio-tienen-com-a-maxim-un-nombre-numerable-de-punts-de-discontinuitat}
        Sigui~\(F\) una funció de distribució.
        Aleshores el conjunt de punts de discontinuïtat de~\(F\) és com a màxim numerable.
    \end{proposition}
    \begin{proof}
        %TODO % Fer desprès de FVR
    \end{proof}
    \begin{proposition}
        \label{prop:funcions-de-distribucio-per-calcular-variables-aleatories-en-intervals-tancats}
        Sigui~\(F\) una funció de distribució d'una variable aleatòria~\(X\).
        Aleshores per a tot~\(s\),~\(t\in\RR\) tenim que
        \[
            \Probabilitat\{s<X\leq t\}=F(t)-F(s).
        \]
    \end{proposition}
    \begin{proposition}
        \label{prop:funcions-de-distribucio-per-calcular-variables-aleatories-en-semirectes}
        Sigui~\(F\) una funció de distribució d'una variable aleatòria~\(X\).
        Aleshores per a tot~\(x\in\RR\) tenim que
        \[
            \Probabilitat\{X<x\}=\lim_{\substack{t\to x\\t<x}}F(t).
        \]
    \end{proposition}
    \begin{proposition}
        \label{prop:funcions-de-distribucio-per-calcular-variables-aleatories-en-punts}
        Sigui~\(F\) una funció de distribució d'una variable aleatòria~\(X\).
        Aleshores
        \[
            \Probabilitat\{X=x\}=F(x)-\lim_{\substack{t\to x\\t<x}}F(t).
        \]
    \end{proposition}
\section{Variables aleatòries discretes}
\subsection{Variables aleatòries discretes}
    \begin{definition}[Variable aleatòria discreta]
        \labelname{variable aleatòria discreta}\label{def:variable-aleatoria-discreta}
        \labelname{suport d'una variable aleatòria discreta}\label{def:suport-duna-variable-aleatoria-discreta}
        Sigui~\(X\) una variable aleatòria tal que existeix un conjunt finit o numerable~\(S\subseteq\RR\) tal que
        \[
            \Probabilitat\{X\in S\}=1,
        \]
        i tal que per a tot~\(x\in S\) tenim
        \[
            P\{X=x\}\neq0.
        \]
        Aleshores direm que~\(X\) és una variable aleatòria discreta.
        També direm que~\(S\) és el suport de~\(X\).
    \end{definition}
    \begin{example}
        %TODO
    \end{example}
    \begin{observation}
        \label{obs:els-conjunts-finits-o-numerables-son-borelians}
        Sigui~\(S\subseteq\RR\) un conjunt finit o numerable.
        Aleshores~\(S\) és borelià.
    \end{observation}
    \begin{proof}
        %TODO
    \end{proof}
\subsection{Funció de probabilitat d'una variable aleatòria}
    \begin{definition}[Funció de probabilitat]
        \labelname{funció de probabilitat d'una variable aleatòria discreta}\label{def:funcio-de-probabilitat-duna-variable-aleatoria-discreta}
        \labelname{repartiment de massa d'una variable aleatòria discreta}\label{def:repartiment-de-massa-duna-variable-aleatoria-discreta}
        Siguin~\(X\) una variable aleatòria discreta amb suport~\(S\) i
        \begin{align*}
            p\colon S&\longrightarrow[0,1] \\
            x&\longmapsto\Probabilitat\{X=x\}.
        \end{align*}
        una funció.
        Aleshores direm que~\(p\) és la funció de probabilitat de~\(X\).
        També direm que~\(p\) és el repartiment de massa de~\(X\).
    \end{definition}
    \begin{observation}
        \label{prop:la-funcio-de-probabilitat-duna-variable-aleatoria-discreta-suma-1-en-el-suport}
        Sigui~\(X\) una variable aleatòria discreta amb suport~\(S\) i funció de probabilitat~\(p\).
        Aleshores
        \[
            \sum_{x\in S}p(s)=1.
        \]
    \end{observation}
    \begin{proof}
        %TODO
    \end{proof}
    \begin{proposition}
        \label{prop:probabilitat-dun-conjunt-en-una-variable-aleatoria-discreta}
        Siguin~\(X\) una variable aleatòria discreta amb suport~\(S\) i funció de probabilitat~\(p\), i~\(B\in\borel(\RR)\) un borelià.
        Aleshores
        \[
            \Probabilitat\{X=B\}=\sum_{\substack{x\in S\\x\in B}}p(x).
        \]
    \end{proposition}
    \begin{proof}
        %TODO
    \end{proof}
    %Potser caldrà reescriure aquestes coses per fer-les formals. I anar-les arrosegant per fer les solucions poc a poc? Però aleshores es pot fer molt llarg. Decidiré quan em posi a fer-les.
    \begin{example}[Variable aleatòria degenerada]
        \labelname{variable aleatòria degenerada}\label{ex:variable-aleatoria-degenerada}
        Siguin~\((\Omega,\Algebra{A},\Probabilitat)\) un espai de probabilitat i~\(a\in\RR\) un real.
        Volem veure que la funció
        \begin{align*}
            X\colon\Omega&\longrightarrow\RR \\
            x&\longmapsto a
        \end{align*}
        és una variable aleatòria.
        També volem calcular i dibuixar la seva funció de probabilitat i de distribució.
    \end{example}
    \begin{solution}
        %TODO
    \end{solution}
    \begin{example}[Llei de Bernoulli]
        \labelname{llei de Bernoulli}\label{def:llei-de-Bernoulli}
        Sigui~\(X\) una variable aleatòria discreta amb suport~\(S=\{0,1\}\).
        Volem calcular i dibuixar la seva funció de probabilitat i de distribució.
    \end{example}
    \begin{solution}
        %TODO
    \end{solution}
    \begin{example}[Llei uniforme sobre~\(n\) punts]
        \labelname{llei uniforme sobre~\(n\) punts}\label{def:llei-uniforme-sobre-n-punts}
        Sigui~\(X\) una variable aleatòria discreta amb suport~\(S=\{x_{1},\dots,x_{n}\}\) tal que per a tot~\(i\in\{1,\dots,n\}\) tenim que
        \[
            P\{X=x_{i}\}=\frac{1}{n}.
        \]
        Aleshores escriurem~\(X\sim\Unif(x_{1},\dots,x_{n})\).
        Volem calcular i dibuixar la seva funció de probabilitat i de distribució.
    \end{example}
    \begin{solution}
        %TODO
    \end{solution}
    % Fer la resta de distribucions/lleis
    %TODO Binomial
    %TODO Poisson
    \begin{example}
        %Aplicació d'alguna de les lleis
    \end{example}
\section{Variables aleatòries absolutament contínues} % i mixtes
\subsection{Funcions de densitat i de distribució}
    \begin{definition}[Variable aleatòria absolutament contínua]
        \labelname{variable aleatòria absolutament contínua}\label{def:variable-aleatoria-absolutament-continua}
        \labelname{funció de densitat d'una variable aleatòria absolutament contínua}\label{def:funcio-de-densitat-duna-variable-aleatoria-abolsutament-continua}
        \labelname{funció de distribució d'una variable aleatòria absolutament contínua}\label{def:funcio-de-distribucio-duna-variable-aleatoria-absolutament-continua}
        Sigui~\(X\) una variable aleatòria tal que existeix una funció~\(f\colon\RR\longleftarrow\RR\) satisfent
        \begin{enumerate}
            \item Per a tot~\(x\in\RR\) tenim que~\(f(x)\geq0\).
            \item La funció~\(f\) és integrable sobre tot~\(\RR\).
            \item~\(\int_{-\infty}^{\infty}f(x)dx=1\).
            \item Per a tot~\(a\),~\(b\in\RR\cup\{\pm\infty\}\) tenim que
            \[
                \Probabilitat\{a\leq X\leq b\}=\int_{a}^{b}f(x)dx.
            \]
        \end{enumerate}
        Aleshores direm que~\(X\) és una variable aleatòria absolutament contínua.
        També direm que~\(f\) és la funció densitat de~\(X\) i que la funció
        \[
            F(x)=\Probabilitat\{X=x\}=\int_{-\infty}^{x}f(t)dt
        \]
        és la funció de distribució de~\(X\).
    \end{definition}
    \begin{observation}
        \label{obs:la-funcio-distribucio-duna-variable-aleatoria-absolutament-continua-es-continua}
        Sigui~\(F\) la funció de distribució d'una variable aleatòria absolutament contínua~\(X\).
        Aleshores~\(F\) és contínua.
        % i absolutament contínua
    \end{observation}
%    \begin{example}
%        %TODO % Veure si ho ha un exemple previ a les lleis i distribucions
%    \end{example}
    \begin{example}[Llei uniforme]
    \labelname{llei uniforme}\label{ex:llei-uniforme}
        Siguin~\(a\),~\(b\in\RR\) dos reals amb~\(a<b\) i~\(X\) una variable aleatòria tal que la seva funció de densitat sigui
        \begin{align*}
            f(x)=\frac{1}{b-a}\ind_{(a,b)}(x).
        \end{align*}
        Volem calcular la seva funció de distribució i dibuixar-les.

        Direm que~\(X\sim\Unif(a,b)\).
    \end{example}
    \begin{solution}
        %TODO
    \end{solution}
    \begin{example}[Llei exponencial]
        \labelname{llei exponencial}\label{ex:llei-exponencial}
        Siguin~\(\lambda\in\RR\) un real amb~\(\lambda>0\) i~\(X\) una variable aleatòria tal que la seva funció de densitat sigui
        \begin{align*}
            f(x)=\lambda\e^{-\lambda x}\ind_{(0,\infty)}(x).
        \end{align*}
        Volem calcular la seva funció de distribució i dibuixar-les.

        Direm que~\(X\sim\Exp(\lambda)\).
    \end{example}
    \begin{solution}
        %TODO
    \end{solution}
    % Propietat de la falta de memòria
    \begin{example}[Distribució gaussiana]
        \labelname{distribució gaussiana}\label{ex:distribucio-gaussiana}
        Siguin~\(\mu\),~\(\sigma\in\RR\) un real amb~\(\sigma>0\) i~\(Z\) una variable aleatòria tal que la seva funció de densitat sigui
        \begin{align*}
            f(x)=\frac{1}{\sigma\sqrt{2\pi}}\e^{-\frac{(x-\mu)^{2}}{2\sigma^{2}}}.
        \end{align*}
        Volem comprovar que aquesta és una funció de densitat i dibuixar-la per~\(\mu=2\) i~\(\sigma=1\).

        Direm que~\(Z\sim\Gauss(\mu,\sigma^{2})\).
    \end{example}
    \begin{solution}
        %TODO
    \end{solution}
    \begin{example}[Distribució Gamma]
        \labelname{distribució Gamma}\label{ex:distribucio-Gamma}
        Siguin~\(r\),~\(\alpha\in\RR\) dos reals amb~\(r>0\) i~\(\alpha>0\).
        Volem veure que existeix una variable aleatòria~\(X\) amb funció de distribució
        \[
            f(x)=\frac{\alpha^{r}}{\Gamma(r)}x^{r-1}\e^{-\alpha x}\ind_{(0,\infty)}(x).
        \]
        També volem dibuixar~\(f(x)\).
        % Determinar valors guays.

        Sigui una variable aleatòria~\(X\) amb funció de distribució~\(f(x)\).
        Aleshores direm que~\(X\sim\Gamm(r,\alpha)\).
    \end{example}
    \begin{solution}
        %TODO
    \end{solution}
%    \begin{example}[Distribució beta]
%        \labelname{distribució beta}\label{ex:distribució beta}
%    \end{example}
\subsection{Criteri per calcular densitats de variables aleatòries}
    \begin{proposition}
        \label{prop:criteri-per-calcular-densitats-de-variables-aleatories}
        Sigui~\(X\) una variable aleatòria amb funció de distribució~\(F\colon\RR\longrightarrow\RR\) tal que
        \begin{enumerate}
            \item La funció~\(F\) és contínua~\(\RR\).
            \item La funció~\(F\) és derivable en tot~\(\RR\) excepte en una quantitat finita numerable de punts.
            \item La funció~\(F'\) és contínua en tot~\(\RR\) excepte en una quantitat finita de punts.
        \end{enumerate}
        Aleshores per a tot~\(x\in\RR\) tenim que
        \[
            F(x)=\int_{-\infty}^{x}F'(t)\diff t.
        \]
    \end{proposition}
    \begin{proof}
        %TODO
    \end{proof}
    \begin{example}
        %TODO
    \end{example}
%\subsection{Funcions de variables aleatòries amb densitat}
    % Exercicis xungos del final (pàg. 59)
\section{Vectors aleatoris}
\subsection{Vectors de variables aleatòries}
\end{document}
