\documentclass[../equacions-diferencials-II.tex]{subfiles}

\begin{document}
\chapter{Sistemes autònoms al pla}
%\section{Òrbites}
%    \subsection{Sistemes autònoms a \ensuremath{\mathbb{R}^{n}}}
%    \subsection{Interpretació geomètrica}
%    \subsection{Estructura de les òrbites}
%    \subsection{Superfícies invariants}
%\section{Sistemes integrables}
%    \subsection{Integral primera}
%%    \subsection{Propietats dels sistemes integrables}
%    \subsection{Sistemes potencials}
%    \subsection{Sistemes Hamiltonians}
%    \subsection{Model de Lotka-Voltera}
\section{Sistemes no integrables}
%    \subsection{Teorema del flux tubular}
%    \subsection{Anàlisi qualitativa dels punts d'equilibri}
    \subsection{Comportament límit de les òrbites} % 4.1 Equações Diferenciais Ordinárias
    \begin{definition}[Conjunts \(\omega\)-límit i \(\alpha\)-límit]
        \labelname{conjunt \ensuremath{\omega}-límit}\label{def:conjunt-omega-limit}
        \labelname{conjunt \ensuremath{\alpha}-límit}\label{def:conjunt-alpha-limit}
        Sigui \(\varphi_{x}\colon\mathbb{R}\longrightarrow\mathbb{R}^{n}\) l'òrbita d'un punt \(x\in\mathbb{R}^{n}\).
        Aleshores definim
        \[
            \omega(x)=\{y\in\mathbb{R}^{n}\mid\text{existeix }(t_{n})_{n\in\mathbb{N}}\text{ amb }t_{n}\overset{n\to\infty}{\longrightarrow}+\infty\text{ tal que }\varphi_{x}(t_{n})\overset{n\to\infty}{\longrightarrow}y\}
        \]
        com el conjunt \(\omega\)-límit de \(x\) i
        \[
            \alpha(x)=\{y\in\mathbb{R}^{n}\mid\text{existeix }(t_{n})_{n\in\mathbb{N}}\text{ amb }t_{n}\overset{n\to\infty}{\longrightarrow}-\infty\text{ tal que }\varphi_{x}(t_{n})\overset{n\to\infty}{\longrightarrow}y\}
        \]
        com el conjunt \(\alpha\)-límit de \(x\).
    \end{definition}
    \begin{observation}
        Siguin \(x\in\mathbb{R}^{n}\) un punt i
        \[
            \dot{x}=f(x)\quad\text{i}\quad\dot{x}=-f(x)
        \]
        dues equacions diferencials.
        Aleshores el conjunt \(\omega\)-límit de \(x\) en la primera equació diferencial és el conjunt \(\alpha\)-límit de \(x\) en la segona equació diferencial i, anàlogament, el conjunt \(\alpha\)-límit de \(x\) en la primera equació diferencial és el conjunt \(\omega\)-límit de \(x\) en la segona equació diferencial.
    \end{observation}
    \begin{definition}[Conjunts invariants]
        \labelname{conjunt positivament invariant}\label{def:conjunt-positivament-invariant}
        \labelname{conjunt negativament invariant}\label{def:conjunt-negativament-invariant}
        \labelname{conjunt invariant}\label{def:conjunt-invariant}
        Sigui \(\varphi_{x}\) l'òrbita d'un punt \(x\in\mathbb{R}^{n}\).
        Aleshores definim
        \[
            \gamma^{+}(x)=\{\varphi_{x}(t)\mid t\geq0\},\quad\gamma^{-}(x)=\{\varphi_{x}(t)\mid t\leq0\}\quad\text{i}\quad\gamma(x)=\gamma^{+}(x)\cup\gamma^{-}(x),
        \]
        i direm que un conjunt \(C\subseteq\mathbb{R}^{n}\) és positivament invariant si per a tot \(y\in C\) tenim que \(\gamma^{+}(y)\subseteq C\), que és negativament invariant si per a tot \(y\in C\) tenim que \(\gamma^{-}(y)\subseteq C\) i que és invariant si per a tot \(y\in C\) tenim que \(\gamma(y)\subseteq C\).
    \end{definition}
    \begin{theorem}
        \label{thm:propietats-topologiques-dels-omega-limits}
        Siguin
        \(\dot{x}=f(x)\)
        una equació diferencial sobre \(\mathbb{R}^{n}\), \(\tancat{K}\) un compacte de \(\mathbb{R}^{n}\) i \(p\in\mathbb{R}^{n}\) un punt tals que \(\gamma^{+}(x)\subset\tancat{K}\).
        Aleshores
        \begin{enumerate}
            \item\label{enum1:thm:propietats-topologiques-dels-omega-limits} \(\omega(x)\neq\emptyset\).
            \item\label{enum2:thm:propietats-topologiques-dels-omega-limits} \(\omega(x)\) és un tancat.
            \item\label{enum3:thm:propietats-topologiques-dels-omega-limits} \(\omega(x)=\omega(\varphi_{x}(t))\) per a tot \(t\in\mathbb{R}\).
            \item\label{enum4:thm:propietats-topologiques-dels-omega-limits} \(\omega(x)\) és connex.
        \end{enumerate}
    \end{theorem}
    \begin{proof}
        Comencem veient el punt \eqref{enum1:thm:propietats-topologiques-dels-omega-limits}.
        Per hipòtesi tenim que \(\gamma^{+}(x)\subset\tancat{K}\).
        Ara bé, tenim que \(\varphi_{x}(0)=x\in\gamma^{+}(x)\), i per tant \(x\in\tancat{K}\) i tenim que \(\tancat{K}\neq\emptyset\).
        %TODO (pàg. 132 Equações Diferenciais Ordinárias)
    \end{proof}
    \subsection{Teorema de Poincaré-Bendixson}
    \begin{definition}[Secció transversal]
        \labelname{secció transversal}\label{def:seccio-transversal}
        Siguin \(\dot{x}=f(x)\) una equació diferencial sobre un obert \(\obert{U}\subseteq\mathbb{R}^{2}\) i \(\Sigma\subseteq\obert{U}\) una corba tal que per a tot \(x,y\in\Sigma\) es satisfà \(f(x)=f(y)\).
        Aleshores direm que \(\Sigma\) és una secció transversal de l'equació diferencial.
        % \Sigma no és una corba, és la seva imatge
    \end{definition}
%    \begin{lemma}
%        \label{lemma:Teorema de Poincaré-Bendixson 1}
%        Siguin \(\dot{x}=f(x)\) una equació diferencial sobre un obert \(\obert{U}\subseteq\mathbb{R}^{2}\), \(\Sigma\subseteq\obert{U}\) una secció transversal i \(p\in\obert{U}\) i \(q\in\Sigma\cap\omega(p)\) dos punts. Aleshores existeix una successió de punts \((\varphi_{p}(t_{n}))_{n\in\mathbb{N}}\) tals que
%        \[\lim_{n\to\infty}\varphi_{p}(t_{n})=q.\]
%        \begin{proof}
%            %TODO
%        \end{proof}
%    \end{lemma}
    \begin{definition}[Aplicació de retorn]
        \labelname{aplicació de retorn}\label{def:aplicacio-de-retorn}
        Siguin \(\dot{x}=f(x)\) una equació diferencial sobre un obert \(\obert{U}\subseteq\mathbb{R}^{2}\) i \(A\subseteq\obert{U}\) un conjunt tals que per a tot \(p\in A\) punt regular i tota secció transversal \(\Sigma\subseteq A\) existeix un entorn \(\obert{V}\subseteq\obert{U}\) de \(p\) tal que per a tot \(x\in\Sigma\cap\obert{V}\) existeix un \(t\neq0\) tal que \(\varphi_{x}(t)\in\Sigma\cap\obert{V}\).
        Aleshores direm que \(A\) admet una aplicació de retorn.
    \end{definition}
    \begin{definition}[Gràfic]
        \labelname{gràfic}\label{def:grafic}
        Siguin \(\dot{x}=f(x)\) una equació diferencial sobre un obert \(\obert{U}\subseteq\mathbb{R}^{2}\) i \(A\subseteq\obert{U}\) un conjunt invariant que conté punts regulars i punts crítics tal que per a tot punt \(x\in A\) tenim \(\alpha(x)=\{p\}\) i \(\omega(x)=\{q\}\) per a certs punts crítics \(p,q\in A\), i tals que \(A\) admet una aplicació de retorn.
        Aleshores direm que \(A\) és un gràfic.
    \end{definition}
    \begin{theorem}[Teorema de Poincaré-Bendixson]
        \labelname{Teorema de Poincaré-Bendixson}\label{thm:Teorema-de-Poincare-Bendixson}
        Siguin \(\dot{x}=f(x)\) una equació diferencial en \(\Omega\subseteq\mathbb{R}^{2}\) amb un nombre finit de punts crítics i \(\tancat{K}\subseteq\Omega\) un compacte tal que \(\gamma^{+}(x)\subseteq\tancat{K}\).
        Aleshores
        \begin{enumerate}
            \item\label{enum1:thm:Teorema-de-Poincare-Bendixson} si \(\omega(x)\) només conté punts crítics tenim que \(\omega(x)=\{p\}\).
            \item\label{enum2:thm:Teorema-de-Poincare-Bendixson} si \(\omega(x)\) conté punts crítics i punts regulars tenim que \(\omega(x)\) és un gràfic.
            \item\label{enum3:thm:Teorema-de-Poincare-Bendixson} si \(\omega(x)\) no conté punts crítics tenim que \(\omega(x)\) és una òrbita periòdica.
        \end{enumerate}
    \end{theorem}
    \begin{proof}
%            Aquest enunciat té sentit pel Teorema \myref{thm:propietats-topologiques-dels-omega-limits}. %TODO
    \end{proof}
%    \begin{definition}[Corona circular]
%        \labelname{corona circular}\label{def:corona circular}
%    \end{definition}
    \begin{theorem}
        \label{thm:dins-duna-orbita-periodica-hi-ha-un-punt-critic}
        Sigui \(\varphi\colon\mathbb{R}\longrightarrow\mathbb{R}^{2}\) una òrbita periòdica d'una equació diferencial \(\dot{x}=f(x)\).
        Aleshores existeix un \(p\in\interior(\varphi(\mathbb{R}))\) tal que \(p\) és un punt crític.
    \end{theorem}
    \begin{proof}
        %TODO
    \end{proof}
    \subsection{Criteri de Bendixson-Dulac}
    \begin{theorem}[Criteri de Bendixson-Dulac]
        \labelname{criteri de Bendixson-Dulac}\label{thm:Criteri-de-Bendixson-Dulac}
    \end{theorem}
    \begin{proof}
        %TODO
    \end{proof}
    % 4.2 Equações Diferenciais Ordinárias
%    \subsection{Funcions de Liapunov}
    % 5 Lições de Equações Diferenciais Ordinárias
%    \subsection{Cicles límit}
\end{document}
