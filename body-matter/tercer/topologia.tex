\documentclass[../tercer.tex]{subfiles}

\begin{document}
\part{Topologia}
\subfile{./topologia/1-lespai-topologic.tex}
\subfile{./topologia/2-altres-topologies.tex}
\subfile{./topologia/3-espais-topologics.tex}
\subfile{./topologia/4-espais-connexos.tex}
\printbibliography
La majoria del contingut està escrit seguint \cite{ACTEAguade}, que també s'utilitza per les classes de teoria de l'assignatura.
He extret la demostració d'un Teorema de \cite{SchultzFreeActionsOnFiniteGroupsOnHausdorffSpaces}.

La bibliografia del curs inclou els textos \cite{AFirstCourseInAlgebraicTopologyKosniowski,ABasicCourseInAlgebraicTopologyKosniowski,TopologyKlaus,ACTEAguade}.
\end{document}

% Treure la merda de unió disjunta. Fer-ho com a \cite{AFirstCourseInAlgebraicTopologyKosniowski}. Pàgina 54 del llibre, abans del teorema 8.11, per X/A i el tema de connexitat

% https://arxiv.org/pdf/0708.2136.pdf (L) Sorprenentment elegant (Espais d'Alexandroff)
% http://math.ucr.edu/~res/math205C-2011/freeactions.pdf El quocient d'un espai Hausdorff compacte per un grup és Hausdorff compacte
