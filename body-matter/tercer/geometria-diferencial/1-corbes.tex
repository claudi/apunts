\documentclass[../geometria-diferencial.tex]{subfiles}

\begin{document}
\chapter{Corbes}
\section{Parametritzacions i longitud}
\begin{comment}
\section{Difeomorfismes de classe \ensuremath{\mathcal{C}^{\infty}}}
    \subsection{Funcions analítiques}
    \begin{definition}[Classe de diferenciabilitat infinita]
        \labelname{classe de diferenciabilitat infinita}\label{def:classe de diferenciabilitat infinita}
        Sigui~\(\obert{U}\subseteq\mathbb{R}^{d}\) un obert i
        \begin{align*}
            f\colon\obert{U}&\longrightarrow\mathbb{R}^{m} \\
            x&\longmapsto(f_{1}(x),\dots,f_{m}(x))
        \end{align*}
        una funció tal que per a tot~\(i\in\{1,\dots,m\}\) la funció~\(f_{i}\) és de classe~\(\mathcal{C}^{k}\) per a tot natural~\(k\in\mathbb{N}\).
        Aleshores direm que~\(f\) és de classe de diferenciabilitat infinita.
        També direm que~\(f\) és de classe~\(\mathcal{C}^{\infty}\).
    \end{definition}
    \begin{definition}[Funció analítica]
        \labelname{funció analítica}\label{def:funció analítica}
        Sigui~\(\obert{U}\subseteq\mathbb{R}^{d}\) un obert,
        \begin{align*}
            f\colon\obert{U}&\longrightarrow\mathbb{R}^{m} \\
            x&\longmapsto(f_{1}(x),\dots,f_{m}(x))
        \end{align*}
        una funció de classe~\(\mathcal{C}^{\infty}\) i~\(x_{0}\in\obert{U}\) un punt tal que existeix una sèrie de potències~\(\sum_{n=0}^{\infty}a_{n}(x-x_{0})^{n}\) tal que existeix un entorn~\(N_{x_{0}}\subseteq\obert{U}\) satisfent que per a tot~\(x\in N_{x_{0}}\) la sèrie de potències~\(\sum_{n=0}^{\infty}a_{n}(x-x_{0})^{n}\) convergeix puntualment a~\(f(x)\).
        Aleshores direm que~\(f\) és una funció analítica.
        També direm que~\(f\) és de classe~\(\mathcal{C}^{\omega}\).
    \end{definition}
    \begin{example}
        \label{ex:els polinomis són funcions analítiques}
        Volem veure que tot polinomi és una funció analítica.
    \end{example}
    \begin{solution}
        Prenem un polinomi
        \[
            p(x)=a_{0}+a_{1}x+a_{2}x^{2}+\dots+a_{n}x^{n}
        \]
        i un natural~\(k\in\mathbb{N}\).
        Si~\(k\leq n\) tenim que % REFS
        \[
            p^{(k)}(x)=k!a_{k}+(k+1)!a_{k+1}x+\dots+\frac{n!}{(n-k)!}a_{n}x^{n-k},
        \]
        i si~\(k>n\) tenim que
        \[
            p^{(k)}(x)=0.
        \]
        Aleshores per la definició de \myref{def:classe de diferenciabilitat infinita} tenim que~\(p(x)\) és de classe~\(\mathcal{C}^{\infty}\).
        Observem que
        \[
            p(x)=\sum_{i=0}^{n}a_{i}x^{i}
        \]
        i per la definició de \myref{def:serie-de-potencies} tenim que~\(p(x)\) és una sèrie de potències, i per la definició de \myref{def:convergencia-puntual} trobem que~\(p(x)\) convergeix puntualment a~\(\sum_{i=0}^{n}a_{i}x^{i}\) per a tot~\(x\in\mathbb{R}\), i per la definició de \myref{def:funció analítica} tenim que~\(p(x)\) és una funció analítica per a tot~\(x\in\mathbb{R}\).
    \end{solution}
    \begin{definition}[Difeomorfisme de classe \ensuremath{\mathcal{C}^{\infty}}]
        \labelname{difeomorfisme de classe \ensuremath{\mathcal{C}^{\infty}}}\label{def:difeomorfisme de classe C infinit}\label{def:difeomorfisme de classe de diferenciabilitat infinita}
        Siguin~\(\obert{U}\) i~\(\obert{V}\) dos oberts de~\(\mathbb{R}^{d}\) i~\(\Phi\colon\obert{U}\longleftrightarrow\obert{V}\) un difeomorfisme tal que les funcions~\(\Phi\) i~\(\Phi^{-1}\) siguin de classe~\(\mathcal{C}^{\infty}\).
        Aleshores direm que~\(\Phi\) és un difeomorfisme de classe de diferenciabilitat infinita o que~\(\Phi\) és un difeomorfisme de classe~\(\mathcal{C}^{\infty}\).

        Aquesta definició té sentit per la definició de \myref{def:difeomorfisme}.
    \end{definition}
\end{comment}
    \subsection{Reparametrització d'una corba}
    \begin{definition}[Corba]
        \labelname{corba}\label{def:corba}
        Siguin~\(I\subseteq\mathbb{R}\) un interval i~\(\alpha\colon I\longrightarrow\mathbb{R}^{n}\) una funció contínua.
        Aleshores direm que~\(\alpha\) és una corba sobre~\(I\).
    \end{definition}
    \begin{example}
        \label{ex:corba-que-uniex-dos-punts}
        Volem veure que donats dos punts~\(a\) i~\(b\) de~\(\mathbb{R}^{n}\) existeix una corba que els uneix.
    \end{example}
    \begin{solution}
        Observem que la funció contínua % Veure que és contínua per ser un polinomi
        \begin{align*}
            \alpha\colon[0,1]&\longrightarrow\mathbb{R}^{n} \\
            t&\longmapsto a+t(b-a)
        \end{align*}
        satisfà que~\(\alpha(0)=a\) i~\(\alpha(1)=b\), i per la definició de \myref{def:corba} trobem que és una corba sobre~\([0,1]\).
    \end{solution}
    \begin{definition}[Corba regular]
        \labelname{corba regular}\label{def:corba-regular}
        Sigui~\(\alpha\) una corba diferenciable sobre~\(I\) tal que per a tot~\(t\in I\) es satisfà~\(\alpha'(t)\neq\vec{0}\).
        Aleshores direm que~\(\alpha\) és regular.
    \end{definition}
    \begin{definition}[Reparametrització]
        \labelname{reparametrització d'una corba}\label{def:reparametritzacio-duna-corba}
        \labelname{canvi de paràmetre}\label{def:canvi-de-parametre}
        Siguin~\(\alpha\) una corba sobre~\(I\) i~\(h\colon J\longrightarrow I\) un difeomorfisme.
        Aleshores direm que la funció
        \begin{align*}
            \beta\colon J&\longrightarrow\mathbb{R}^{n} \\
            t&\longmapsto(\alpha\circ h)(t)
        \end{align*}
        és una reparametrització en~\(J\) de~\(\alpha\) i que~\(h\) és el canvi de paràmetre.
    \end{definition}
    \begin{proposition}
        Sigui~\(\beta\) una reparametrització en~\(J\) d'una corba regular~\(\alpha\) sobre~\(I\).
        Aleshores~\(\beta\) és una corba regular.
    \end{proposition}
    \begin{proof}
        Per la definició de \myref{def:reparametritzacio-duna-corba} trobem que existeix un canvi de paràmetre~\(h:J\longrightarrow I\) tal que~\(\beta=(\alpha\circ h)(t)\).
        Aleshores tenim que
        \begin{align*}
            \beta\colon J&\longrightarrow\mathbb{R}^{n} \\
            t&\longmapsto(\alpha\circ h)(t),
        \end{align*}
        i per la definició de \myref{def:corba} trobem que~\(\beta\) és una corba.

        Per la definició de \myref{def:canvi-de-parametre} trobem que~\(h\) és un difeomorfisme, i per la definició de \myref{def:difeomorfisme} trobem que~\(h\) és derivable, i per la \myref{thm:regla-de-la-cadena} trobem que
        \[
            \beta'(t)=h'(t)(\alpha'\circ h)(t).
        \]

        Com que per hipòtesi~\(\alpha\) és regular, per la definició de \myref{def:corba-regular} trobem que~\((\alpha'\circ h)(t)\neq\vec{0}\) per a tot~\(t\in J\).
        Ara bé, pel \corollari{} \myref{cor:difeomorfisme-es-equivalent-a-ser-injectiva-i-tenir-diferencial-amb-determinant-no-nul} trobem que~\(h'(t)\neq\vec{0}\) per a tot~\(t\in J\), i per tant tenim que~\(\beta'(t)\neq\vec{0}\) per a tot~\(t\in J\), i per la definició de \myref{def:corba-regular} trobem que~\(\beta\) és una corba regular.
    \end{proof}
    \subsection{La longitud d'una corba i el paràmetre arc}
    \begin{definition}[Longitud d'una corba]
        \labelname{longitud d'una corba}\label{def:longitud-duna-corba}
        Siguin~\(\alpha\) una corba sobre~\(I\) i~\(a\),~\(b\) dos punts de~\(I\).
        Aleshores definim
        \[
            \Long_{a}^{b}(\alpha)=\int_{a}^{b}\norm{\alpha'(t)}\diff t
        \]
        com la longitud de~\(\alpha\) entre~\(a\) i~\(b\).
    \end{definition}
    \begin{example}
        \label{ex:longitud-duna-corba}
        Volem calcular la longitud de la corba
        \[
            \alpha(t)=(\cos(t),\sin(t),t)
        \]
        entre~\(0\) i~\(x\) per a tot~\(x\) positiu.
    \end{example}
    \begin{solution}
        Per la definició de \myref{def:longitud-duna-corba} trobem que això és
        \begin{align*}
            \Long_{0}^{x}(\alpha)&=\int_{0}^{x}\norm{\alpha'(t)}\diff t \\
            &=\int_{0}^{x}\norm{(-\sin(t),\cos(t),1)}\diff t \\
            &=\int_{0}^{x}\sqrt{\sin^{2}(t)+\cos^{2}(t)+1}\diff t \\
            &=\int_{0}^{x}\sqrt{2}\diff t \\
            &=[\sqrt{2}t]_{0}^{x}=\sqrt{2}x.\qedhere
        \end{align*}
    \end{solution}
    \begin{proposition}
        Sigui~\(\beta\) una reparametrització d'una corba~\(\alpha\) en~\(I\) amb canvi de paràmetre~\(h\).
        Aleshores per a tot~\(a\) i~\(b\) de~\(I\), amb~\(c=h^{-1}(a)\) i~\(d=h^{-1}(d)\) tenim
        \[
            \Long_{a}^{b}(\alpha)=\Long_{c}^{d}(\beta).
        \]
    \end{proposition}
    \begin{proof}
        Això té sentit per la definició de \myref{def:canvi-de-parametre} i la definició de \myref{def:difeomorfisme}.

        Per la definició de \myref{def:longitud-duna-corba} trobem que
        \begin{align*}
            \Long_{c}^{d}(\beta)&=\int_{c}^{d}\norm{\beta'(s)}\diff s \\
            &=\int_{c}^{d}\norm{(\alpha\circ h)'(s)}\diff s, \tag{\ref{def:canvi-de-parametre}}
        \end{align*}
        i com que, per la definició de \myref{def:canvi-de-parametre} tenim que~\(h\) és un difeomorfisme, per la definició de \myref{def:difeomorfisme} trobem que~\(h\) és diferenciable i per la \myref{thm:regla-de-la-cadena} trobem que
        \[
            \int_{c}^{d}\norm{(\alpha\circ h)'(s)}\diff s=\int_{c}^{d}\norm{h'(s)\alpha'(h(s))}\diff s,
        \]
        i per la definició de norma tenim que% REF
        \[
            \int_{c}^{d}\norm{h'(s)\alpha'(h(s))}\diff s=\int_{c}^{d}\abs{h'(s)}\norm{\alpha'(h(s))}\diff s.
        \]

        Ara bé, com que~\(h\) és un difeomorfisme tenim per la definició de \myref{def:difeomorfisme} que~\(h'\) és contínua, i pel \corollari{} \myref{cor:difeomorfisme-es-equivalent-a-ser-injectiva-i-tenir-diferencial-amb-determinant-no-nul} trobem que~\(h'(t)\neq0\) per a tot~\(t\in[c,d]\).
        Aleshores tenim que ha de ser o bé~\(h'(t)>0\) per a tot~\(t\in[c,d]\) o bé~\(h'(t)<0\) per a tot~\(t\in[c,d]\).

        Comencem suposant que~\(h'(t)>0\) per a tot~\(t\in[c,d]\).
        Aleshores tenim que
        \[
            \int_{c}^{d}\abs{h'(s)}\norm{\alpha'(h(s))}\diff s=\int_{c}^{d}h'(s)\norm{\alpha'(h(s))}\diff s,
        \]
        i pel Teorema del canvi de variable tenim que %REF
        \begin{align*}
            \int_{c}^{d}h'(s)\norm{\alpha'(h(s))}\diff s&=\int_{h(c)}^{h(d)}\norm{\alpha(t)}\diff t \\
            &=\int_{a}^{b}\norm{\alpha(t)}\diff t=\Long_{a}^{b}(\alpha),
        \end{align*}
        i per tant
        \[
            \Long_{c}^{d}(\beta)=\Long_{a}^{b}(\alpha).
        \]

        Suposem ara que~\(h'(t)<0\) per a tot~\(t\in[c,d]\).
        Aleshores tenim que            \begin{align*}
            \int_{c}^{d}\abs{h'(s)}\norm{\alpha'(h(s))}\diff s&=\int_{c}^{d}-h'(s)\norm{\alpha'(h(s))}\diff s \\
            &=\int_{d}^{c}h'(s)\norm{\alpha'(h(s))}\diff s
        \end{align*}
        i pel Teorema del canvi de variable tenim que %REF
        \begin{align*}
            \int_{d}^{c}h'(s)\norm{\alpha'(h(s))}\diff s&=\int_{h(c)}^{h(d)}\norm{\alpha(t)}\diff t \\
            &=\int_{a}^{b}\norm{\alpha(t)}\diff t=\Long_{a}^{b}(\alpha),
        \end{align*}
        i per tant
        \[
            \Long_{c}^{d}(\beta)=\Long_{a}^{b}(\alpha).\qedhere
        \]
    \end{proof}
    \begin{definition}[Funció longitud d'arc]
        \labelname{funció longitud d'arc}\label{def:funcio-longitud-darc}
        Siguin~\(\alpha\) una corba sobre~\(I\) i~\(a\in I\) un punt.
        Aleshores direm que la funció
        \begin{align*}
            \funciolongituddarc_{\alpha}(a)(t)\colon I&\longrightarrow\mathbb{R}\\
            t&\longmapsto\int_{a}^{t}\norm{\alpha'(s)}\diff s
        \end{align*}
        és la funció longitud d'arc de~\(\alpha\) amb origen en~\(a\).
%        \ref{def:diferencial}
    \end{definition}
    \begin{observation}
        \label{obs:la-funcio-longitud-darc-es-creixent}
        Sigui~\(\alpha\) una corba sobre~\(I\) i~\(a\in I\) un punt.
        Aleshores per a tot~\(t\in I\)
        \[
            \frac{\diff\funciolongituddarc_{\alpha}(a)}{\diff t}(t)\geq0.
        \]
    \end{observation}
    \begin{proof}
        Per la definició de \myref{def:funcio-longitud-darc} trobem que
        \[
            \funciolongituddarc_{\alpha}(a)(t)=\int_{a}^{t}\norm{\alpha'(s)}\diff s,
        \]
        i pel \myref{thm:Teorema-Fonamental-del-Calcul} tenim que
        \[
            \frac{\diff\funciolongituddarc_{\alpha}(a)}{\diff t}(t)=\norm{\alpha'(t)}\geq0.\qedhere
        \] % REFS
    \end{proof}
    \begin{proposition}
        \label{prop:la-funcio-longitud-arc-duna-corba-regular-es-un-difeomorfisme}
        Siguin~\(\alpha\) una corba regular sobre~\(I\) i~\(a\in I\) un punt.
        Aleshores la funció~\(\funciolongituddarc_{\alpha}(a)\) és un difeomorfisme.
        % SOBRE J
    \end{proposition}
    \begin{proof}
        Per la definició de \myref{def:corba-regular} trobem que~\(\alpha'(t)\neq0\) per a tot~\(t\in I\), i pel \myref{thm:Teorema-Fonamental-del-Calcul} tenim que
        \[
            \frac{\diff\funciolongituddarc_{\alpha}(a)}{\diff t}(t)=\norm{\alpha'(t)}.
        \]
        Per tant trobem que % REFS
        \[
            \frac{\diff\funciolongituddarc_{\alpha}(a)}{\diff t}(t)\neq0,
        \]
        i per l'observació \myref{obs:diferencial-en-d-1-es-com-derivar} i el \corollari{} \myref{cor:difeomorfisme-es-equivalent-a-ser-injectiva-i-tenir-diferencial-amb-determinant-no-nul} tenim que~\(\funciolongituddarc_{\alpha}(a)\) és un difeomorfisme, com volíem veure.
        % SOBRE J
    \end{proof}
    \begin{proposition} % Veure després de definir corba parametritzada per l'arc
        \label{prop:podem-trobar-una-reparametritzacio-amb-velocitat-unitaria-de-qualsevol-corba-regular}
        Sigui~\(\alpha\) una corba regular sobre~\(I\).
        Aleshores existeix una reparametrització~\(\beta\) de~\(\alpha\) tal que per a tot~\(t\in I\)
        \[
            \norm{\beta'(t)}=1.
        \]
    \end{proposition}
    \begin{proof}
        Per la proposició \myref{prop:la-funcio-longitud-arc-duna-corba-regular-es-un-difeomorfisme} tenim que per a tot~\(a\in I\) la funció~\(\funciolongituddarc_{\alpha}(a)\) és un difeomorfisme, i per la definició de \myref{def:difeomorfisme} trobem que la funció~\(\funciolongituddarc_{\alpha}(a)\) és bijectiva, i pel Teorema \myref{thm:bijectiva-iff-invertible} trobem que~\(\funciolongituddarc_{\alpha}(a)\) és invertible i per la definició d'\myref{def:aplicacio-invertible} tenim que existeix una funció~\(t_{a}\) tal que~\(t_{a}\) sigui la inversa de~\(\funciolongituddarc_{\alpha}(a)\).
        Considerem
        \[
            \beta(s)=\alpha(t(s)).
        \]
        Aleshores tenim que
        \begin{align*} % És positiva pel corol·lari del teorema de la funció inversa?
            \norm{\beta'(s)}&=\norm{\frac{\diff t}{\diff s}\alpha'(s)} \\
            &=\abs{\frac{\diff t}{\diff s}}\norm{\alpha'(s)} \\
            &=\frac{\diff t}{\diff s}(s(t))\frac{\diff s}{\diff t}(t(s))\\
            &=\frac{1}{\frac{\diff s}{\diff t}(t(s))}\frac{\diff s}{\diff t}(t(s))=1.
            \qedhere
        \end{align*}
    \end{proof}
    \begin{definition}[Corba parametritzada per l'arc]
        \labelname{corba parametritzada per l'arc}\label{def:corba-parametritzada-per-larc}
        Sigui~\(\alpha\) una corba regular sobre~\(I\) tal que per a tot~\(s\in I\) es satisfà
        \[
            \norm{\alpha'(s)}=1.
        \]
        Aleshores direm que~\(\alpha\) està parametritzada per l'arc.
    \end{definition}
    \begin{example} % Veure cas general amb Teorema de la funció implícita
        \label{ex:reparametritzacio-per-larc-del-cercle-de-radi-R}
        \label{ex:circumferencia-de-radi-R-parametritzat-per-larc}
        Volem donar una reparametrització de la corba
        \[
            \alpha(t)=(R\cos(t),R\sin(t))
        \]
        tal que estigui parametritzada per l'arc.
    \end{example}
    \begin{solution}
        Tenim que
        \[
            \alpha'(t)=(-R\sin(t),R\cos(t)),
        \]
        i per tant
        \begin{align*}
            \norm{\alpha'(t)}&=\norm{(-R\sin(t),R\cos(t))} \\
            &=\sqrt{{(-R\sin(t))}^{2}+{(R\cos(t))}^{2}} \\
            &=\sqrt{R^{2}(\sin^{2}(t)+\cos^{2}(t))}=R.
        \end{align*}
        Per la definició de \myref{def:funcio-longitud-darc} trobem que
        \begin{align*}
            s(t)&=\int_{0}^{t}\norm{\alpha'(x)}\diff x \\
            &=\int_{0}^{t}R\diff x=Rt.
        \end{align*}
        Prenem ara el canvi de paràmetre
        \[
            h(s)=\frac{s}{R}.
        \]
        Tenim que
        \[
            \beta(s)=(\alpha\circ h)(s)=\left(-R\sin\left(\frac{s}{R}\right),R\cos\left(\frac{s}{R}\right)\right).
        \]

        Tenim que~\(\beta\) és una reparametrització de~\(\alpha\).
        Veiem ara que~\(\beta\) està parametritzada per l'arc.
        Tenim que
        \begin{align*}
            \norm{\beta'(s)}&=\norm{\alpha(h(s))'} \\
            &=\norm{h'(s)\alpha'(h(s))} \tag{\ref{thm:regla-de-la-cadena}}\\
            &=\norm{\frac{1}{R}\left(-R\sin\left(\frac{s}{R}\right),R\cos\left(\frac{s}{R}\right)\right)} \\
            &=\norm{\left(-\sin\left(\frac{s}{R}\right),\cos\left(\frac{s}{R}\right)\right)} \\
            &=\sqrt{\sin^{2}\left(\frac{s}{R}\right)+\cos^{2}\left(\frac{s}{R}\right)}=1,
        \end{align*}
        i per la definició de \myref{def:corba-parametritzada-per-larc} hem acabat.
    \end{solution}
    \begin{observation}
        Sigui~\(\alpha\) una corba en~\(I\) parametritzada per l'arc i~\(a\in I\) un punt.
        Aleshores per a tot~\(s\in I\) tenim
        \[
            \funciolongituddarc_{\alpha}(a)(s)=s-a.
        \]
    \end{observation}
    \begin{proof}
        Per la definició de \myref{def:funcio-longitud-darc} trobem que
        \[
            \funciolongituddarc_{\alpha}(a)(s)=\int_{a}^{s}\norm{\alpha'(\tau)}\diff\tau,
        \]
        i per la definició de \myref{def:corba-parametritzada-per-larc} tenim que  per a tot~\(s\in I\) es satisfà~\(\norm{\alpha'(s)}=1\), i per tant
        \begin{align*}
            \funciolongituddarc_{\alpha}(a)(s)&=\int_{a}^{s}\norm{\alpha'(\tau)}\diff\tau \\
            &=\int_{a}^{s}1\diff\tau=s-a.\qedhere
        \end{align*}
    \end{proof}
    \begin{definition}[Contacte]
        \labelname{contacte}\label{def:contacte-entre-dues-corbes-parametritzades-per-larc}
        Siguin~\(\alpha\) i~\(\beta\) dues corbes en~\(I\) parametritzades per l'arc i~\(s_{0}\in I\) un punt tals que existeix un~\(r\in\mathbb{N}\) satisfent
        \[
            \lim_{s\to s_{0}}\frac{\alpha(s)-\beta(s)}{(s-s_{0})^{p}}=0\quad\text{i}\quad\lim_{s\to s_{0}}\frac{\alpha(s)-\beta(s)}{(s-s_{0})^{r+1}}\neq0
        \]
        per a tot~\(p\leq r\).
        Aleshores direm que~\(\alpha\) i~\(\beta\) tenen contacte d'ordre~\(r\) en~\(s_{0}\).
    \end{definition}
    \begin{proposition}
        \label{prop:contacte-r-es-equivalent-a-tenir-les-r-primeres-derivades-iguals}
        Siguin~\(\alpha\) i~\(\beta\) dues corbes en~\(I\) parametritzades per l'arc i~\(s_{0}\in I\) un punt.
        Aleshores~\(\alpha\) i~\(\beta\) tenen contacte d'ordre~\(r\) en~\(s_{0}\) si i només si
        \[
            \alpha^{(p)}(s_{0})=\beta^{(p)}(s_{0})\quad\text{i}\quad\alpha^{(r+1)}(s_{0})\neq\beta^{(r+1)}(s_{0})
        \]
        per a tot~\(p\leq r\).
    \end{proposition}
    \begin{proof}
        Suposem que
        \[
            \alpha^{(p)}(s_{0})=\beta^{(p)}(s_{0})\quad\text{i}\quad\alpha^{(r+1)}(s_{0})\neq\beta^{(r+1)}(s_{0})
        \]
        per a tot~\(p\leq r\).
        Tenim que, per a tot~\(p\leq r\) es satisfà
        \[
            \alpha^{(p)}(s_{0})=\beta^{(p)}(s_{0}),
        \]
        o equivalentment
        \begin{equation}
            \label{prop:contacte-r-es-equivalent-a-tenir-les-r-primeres-derivades-iguals:eq1}
            \alpha^{(p)}(s_{0})-\beta^{(p)}(s_{0})=0.
        \end{equation}
        Prenem~\(n\leq r\), i per la definició de \myref{def:derivada} trobem que
        \begin{align*}
        \alpha^{(n)}(s_{0})-\beta^{(n)}(s_{0})&=\lim_{s\to s_{0}}\bigg(\frac{\alpha^{(n-1)}(s_{0})-\alpha^{(n-1)}(s)}{s-s_{0}}-\frac{\beta^{(n-1)}(s_{0})-\beta^{(n-1)}(s)}{s-s_{0}}\bigg) \\
        &=\lim_{s\to s_{0}}\bigg(\frac{\alpha^{(n-1)}(s_{0})-\alpha^{(n-1)}(s)-\beta^{(n-1)}(s_{0})+\beta^{(n-1)}(s)}{s-s_{0}}\bigg) \\
        &=\lim_{s\to s_{0}}\bigg(\frac{\alpha^{(n-1)}(s_{0})-\beta^{(n-1)}(s_{0})}{s-s_{0}}-\frac{\alpha^{(n-1)}(s)-\beta^{(n-1)}(s)}{s-s_{0}}\bigg)
        \intertext{Ara bé, tenim que~\(\alpha^{(n-1)}(s_{0})-\beta^{(n-1)}(s_{0})=0\), i per tant}
        &=\lim_{s\to s_{0}}\frac{\beta^{(n-1)}(s)-\alpha^{(n-1)}(s)}{s-s_{0}}=0,
        \end{align*}
        i tenim que
        \[
            \lim_{s\to s_{0}}\frac{\alpha(s)-\beta(s)}{(s-s_{0})^{p}}=0.
        \]

        Considerem ara
        \[
            \alpha^{(r+1)}(s_{0})\neq\beta^{(r+1)}(s_{0}).
        \]
        Tenim que
        \[
            \alpha^{(r+1)}(s_{0})-\beta^{(r+1)}(s_{0})\neq0.
        \]
        Ara bé, per la definició de \myref{def:derivada} trobem que
        \begin{align*}
            \alpha^{(r+1)}(s_{0})-\beta^{(r+1)}(s_{0})&=\lim_{s\to s_{0}}\left(\frac{\alpha^{(r)}(s_{0})-\alpha^{(r)}(s)}{s-s_{0}}-\frac{\beta^{(r)}(s_{0})-\beta^{(r)}(s)}{s-s_{0}}\right) \\
             &=\lim_{s\to s_{0}}\frac{\alpha^{(r)}(s_{0})-\alpha^{(r)}(s)-\beta^{(r)}(s_{0})+\beta^{(r)}(s)}{s-s_{0}} \\
             &=\lim_{s\to s_{0}}\frac{\beta^{(r)}(s)-\alpha^{(r)}(s)}{s-s_{0}}\neq0 \tag{\ref{prop:contacte-r-es-equivalent-a-tenir-les-r-primeres-derivades-iguals:eq1}}
        \end{align*}
        i per tant
        \[
            \lim_{s\to s_{0}}\frac{\alpha(s)-\beta(s)}{(s-s_{0})^{r+1}}\neq0
        \]
        i per la definició de \myref{def:contacte-entre-dues-corbes-parametritzades-per-larc} tenim que~\(\alpha\) i~\(\beta\) tenen contacte~\(r\).
    \end{proof}
\section{Curvatura i torsió}
    \subsection{Producte escalar i producte vectorial}
    \begin{proposition}
        \label{prop:derivada-del-producte-escalar-de-dues-corbes}
        Siguin~\(\alpha\) i~\(\beta\) dues corbes diferenciables sobre~\(I\).
        Aleshores
        \[
            \frac{\diff\prodesc{\alpha(t)}{\beta(t)}}{\diff t}=\prodesc{\alpha'(t)}{\beta(t)}+\prodesc{\alpha(t)}{\beta'(t)}
        \]
    \end{proposition}
    \begin{proof}
        Tenim per la definició de \myref{def:derivada} que
        \[
            \frac{\diff\prodesc{\alpha(t)}{\beta(t)}}{\diff t}=\lim_{h\to0}\frac{\prodesc{\alpha(t+h)}{\beta(t+h)}-\prodesc{\alpha(t)}{\beta(t)}}{h},
        \]
        i que
        \[
            \alpha'(t)=\lim_{h\to0}\frac{\alpha(t+h)-\alpha(t)}{h}\quad\text{i}\quad\beta'(t)=\lim_{h\to0}\frac{\beta(t+h)-\beta(t)}{h}.
        \]
        Per tant trobem que quan~\(h\to0\) tenim que
        \[
            \alpha(t+h)=\alpha(t)+h\alpha'(t)\quad\text{i}\quad\beta(t+h)=\beta(t)+h\beta'(t).
        \]
        Per tant tenim que
        \[
            \frac{\diff\prodesc{\alpha(t)}{\beta(t)}}{\diff t}=\lim_{h\to0}\frac{\prodesc{\alpha(t)+h\alpha'(t)}{\beta(t)+h\beta'(t)}-\prodesc{\alpha(t)}{\beta(t)}}{h},
        \]
        i per la definició de \myref{def:producte-escalar} trobem que
        \begin{multline*}
            \prodesc{\alpha(t)+h\alpha'(t)}{\beta(t)+h\beta'(t)}-\prodesc{\alpha(t)}{\beta(t)}=\\
            =\prodesc{\alpha(t)}{\beta(t)+h\beta'(t)}+\prodesc{h\alpha'(t)}{\beta(t)+h\beta'(t)}-\prodesc{\alpha(t)}{\beta(t)}
        \end{multline*}
        i
        \begin{multline*}
            \prodesc{\alpha(t)+h\alpha'(t)}{\beta(t)+h\beta'(t)}=\\
            =\prodesc{\alpha(t)}{\beta(t)}+\prodesc{\alpha(t)}{h\beta'(t)}+\prodesc{h\alpha'(t)}{\beta(t)+h\beta'(t)}=\\
            =\prodesc{\alpha(t)}{\beta(t)}+\prodesc{\alpha(t)}{h\beta'(t)}+\prodesc{h\alpha'(t)}{\beta(t)}+\prodesc{h\alpha'(t)}{h\beta'(t)}
        \end{multline*}
        i per tant
        \begin{multline*}
            \prodesc{\alpha(t)+h\alpha'(t)}{\beta(t)+h\beta'(t)}-\prodesc{\alpha(t)}{\beta(t)}=\\
            =\prodesc{\alpha(t)}{h\beta'(t)}+\prodesc{h\alpha'(t)}{\beta(t)}+\prodesc{h\alpha'(t)}{h\beta'(t)}=\\
            =h\prodesc{\alpha(t)}{\beta'(t)}+h\prodesc{\alpha'(t)}{\beta(t)}+h^{2}\prodesc{\alpha'(t)}{\beta'(t)}.
        \end{multline*}
        Per tant ens queda que
        \begin{multline*}
            \lim_{h\to0}\frac{\prodesc{\alpha(t)+h\alpha'(t)}{\beta(t)+h\beta'(t)}-\prodesc{\alpha(t)}{\beta(t)}}{h}=\\
            =\lim_{h\to0}\frac{h\prodesc{\alpha(t)}{\beta'(t)}+h\prodesc{\alpha'(t)}{\beta(t)}+h^{2}\prodesc{\alpha'(t)}{\beta'(t)}}{h}=\\
            =\prodesc{\alpha(t)}{\beta'(t)}+\prodesc{\alpha'(t)}{\beta(t)}+\lim_{h\to0}h\prodesc{\alpha'(t)}{\beta'(t)}=\\
            =\prodesc{\alpha(t)}{\beta'(t)}+\prodesc{\alpha'(t)}{\beta(t)},
        \end{multline*}
        i trobem
        \[
            \frac{\diff\prodesc{\alpha(t)}{\beta(t)}}{\diff t}=\prodesc{\alpha'(t)}{\beta(t)}+\prodesc{\alpha(t)}{\beta'(t)}.\qedhere
        \]
    \end{proof}
    \begin{proposition}
        \label{prop:unicitat-del-producte-vectorial-entre-dos-vectors}
        Siguin~\(\vec{u}\) i~\(\vec{v}\) dos vectors de~\(\mathbb{R}^{3}\).
        Aleshores existeix un únic vector~\(\vec{w}\) de~\(\mathbb{R}^{3}\) tal que per a tot vector~\(\vec{x}\) de~\(\mathbb{R}^{3}\)
        \[
            \prodesc{\vec{w}}{\vec{x}}=\det(\vec{u},\vec{v},\vec{x}).
        \]
    \end{proposition}
    \begin{proof}
        %TODO
    \end{proof}
    \begin{definition}[Producte vectorial]
        \labelname{producte vectorial}\label{def:producte-vectorial}
        Siguin~\(\vec{u}\) i~\(\vec{v}\) dos vectors de~\(\mathbb{R}^{3}\) i~\(\vec{w}\) el vector de~\(\mathbb{R}^{3}\) tal que
        \[
            \prodesc{\vec{w}}{\vec{x}}=\det(\vec{u},\vec{v},\vec{x}).
        \]
        Aleshores definim el producte vectorial de~\(\vec{u}\) i~\(\vec{v}\) com
        \[
            \vec{u}\prodvec\vec{v}=\vec{w}.
        \]
        Aquesta definició té sentit per la proposició \myref{prop:unicitat-del-producte-vectorial-entre-dos-vectors}.
    \end{definition}
    \begin{observation}
        \label{obs:formula-del-determinant-segons-el-producte-vectorial-i-el-producte-escalar}
        \(\prodesc{\vec{u}\prodvec\vec{v}}{\vec{x}}=\det(\vec{u},\vec{v},\vec{x})\).
    \end{observation}
    \begin{proposition}
    \label{prop:el-producte-vectorial-canvia-de-signe-en-permutar-els-vectors}
        Siguin~\(\vec{u}\) i~\(\vec{v}\) dos vectors de~\(\mathbb{R}^{3}\).
        Aleshores
        \[
            \vec{u}\prodvec\vec{v}=-\vec{v}\prodvec\vec{u}.
        \]
    \end{proposition}
    \begin{proof}
        Sigui~\(\vec{x}\) un vector de~\(\mathbb{R}^{3}\).
        Considerem
        \[
            \det(\vec{u},\vec{v},\vec{x}).
        \]
        Per la definició de \myref{def:determinant-duna-matriu} trobem que % REF, no és la def del det
        \begin{equation}
            \label{prop:el-producte-vectorial-canvia-de-signe-en-permutar-els-vectors:eq1}
            \det(\vec{u},\vec{v},\vec{x})=-\det(\vec{v},\vec{u},\vec{x}),
        \end{equation}
        i per l'observació \myref{obs:formula-del-determinant-segons-el-producte-vectorial-i-el-producte-escalar} tenim que
        \begin{align*}
            \prodesc{\vec{u}\prodvec\vec{v}}{\vec{x}}&=\det(\vec{u},\vec{v},\vec{x}) \\
            &=-\det(\vec{v},\vec{u},\vec{x}) \tag{\ref{prop:el-producte-vectorial-canvia-de-signe-en-permutar-els-vectors:eq1}}\\
            &=-\prodesc{\vec{v}\prodvec\vec{u}}{\vec{x}},
        \end{align*}
        i per tant~\(\vec{u}\prodvec\vec{v}=-\vec{v}\prodvec\vec{u}\), com volíem veure.
        % REF
    \end{proof}
    \begin{proposition}
        \label{prop:el-producte-vectorial-es-zero-si-i-nomes-si-els-vectors-no-son-linealment-independents}
        Siguin~\(\vec{u}\) i~\(\vec{v}\) dos vectors.
        Aleshores~\(\vec{u}\prodvec\vec{v}\neq\vec{0}\) si i només si~\(\vec{u}\) i~\(\vec{v}\) són linealment independents.
    \end{proposition}
    \begin{proof}
        Suposem que~\(\vec{u}\prodvec\vec{v}=\vec{0}\).
        Aleshores tenim, per a tot vector~\(\vec{x}\) de~\(\mathbb{R}^{3}\), que
        \[
            \prodesc{\vec{u}\prodvec\vec{v}}{\vec{x}}=\prodesc{\vec{0}}{\vec{x}},
        \]
        i per la definició de \myref{def:producte-escalar} trobem que
        \[
            \prodesc{\vec{u}\prodvec\vec{v}}{\vec{x}}=0.
        \]
        Ara bé, per l'observació \myref{obs:formula-del-determinant-segons-el-producte-vectorial-i-el-producte-escalar} trobem que per a tot vector~\(\vec{x}\) de~\(\mathbb{R}^{3}\) tenim
        \[
            \det(\vec{u},\vec{v},\vec{x})=0,
        \]
        i per tant tenim que~\(\vec{u}\) i~\(\vec{v}\) no són linealment independents.
        %REF
    \end{proof}
    \begin{proposition}
        \label{prop:dos-vectors-linealment-independents-son-perpendiculars-al-seu-producte-vectorial}
        \label{prop:el-producte-vectorial-es-perpendicular-als-vectors}
        Siguin~\(\vec{u}\) i~\(\vec{v}\) dos vectors linealment independents de~\(\mathbb{R}^{3}\).
        Aleshores~\(\vec{u}\prodvec\vec{v}\) és perpendicular a~\(\vec{u}\) i~\(\vec{v}\).
    \end{proposition}
    \begin{proof}
        %TODO
    \end{proof}
    \begin{proposition}
        \label{prop:el-determinant-de-dos-vectors-linealment-independents-i-el-seu-producte-vectorial-es-diferent-de-zero}
        Siguin~\(\vec{u}\) i~\(\vec{v}\) dos vectors linealment independents de~\(\mathbb{R}^{3}\).
        Aleshores~\(\det(\vec{u},\vec{v},\vec{u}\prodvec\vec{v})\neq0\).
    \end{proposition}
    \begin{proof}
        Per l'observació \myref{obs:formula-del-determinant-segons-el-producte-vectorial-i-el-producte-escalar} tenim que
        \[
            \prodesc{\vec{u}\prodvec\vec{v}}{\vec{u}\prodvec\vec{v}}=\det(\vec{u},\vec{v},\vec{u}\prodvec\vec{v}).
        \]
        Ara bé, com que per hipòtesi els vectors~\(\vec{v}\) i~\(\vec{u}\) són linealment independents, per la proposició \myref{prop:el-producte-vectorial-es-zero-si-i-nomes-si-els-vectors-no-son-linealment-independents} trobem que
        \[
            \vec{u}\prodvec\vec{v}\neq\vec{0},
        \]
        i per la definició de \myref{def:producte-escalar} trobem que
        \[
            \prodesc{\vec{u}\prodvec\vec{v}}{\vec{u}\prodvec\vec{v}}\neq0.
        \]
        Per tant tenim que
        \[
            \det(\vec{u},\vec{v},\vec{u}\prodvec\vec{v})\neq0.\qedhere
        \]
    \end{proof}
    \begin{definition}[Orientació d'una base]
        \labelname{orientació d'una base}\label{def:orientacio-duna-base}
        \labelname{base positiva}\label{def:base-positiva}
        Sigui~\((\vec{u},\vec{v},\vec{w})\) una base tal que~\(\det(\vec{u},\vec{v},\vec{w})>0\).
        Aleshores direm que~\((\vec{u},\vec{v},\vec{w})\) és una base positiva.
    \end{definition}
    \begin{proposition}
        \label{prop:dos-vectors-linealment-independents-i-el-seu-producte-vectorial-formen-una-base-positiva}
        Siguin~\(\vec{u}\) i~\(\vec{v}\) dos vectors linealment independents.
        Aleshores la base~\((\vec{u},\vec{v},\vec{u}\prodvec\vec{v})\) és una base positiva.
    \end{proposition}
    \begin{proof}
        Aquest enunciat té sentit per la proposició \myref{prop:el-determinant-de-dos-vectors-linealment-independents-i-el-seu-producte-vectorial-es-diferent-de-zero}.

        Per l'observació \myref{obs:formula-del-determinant-segons-el-producte-vectorial-i-el-producte-escalar} trobem que
        \[
            \prodesc{\vec{u}\prodvec\vec{v}}{\vec{u}\prodvec\vec{v}}=\det(\vec{u},\vec{v},\vec{u}\prodvec\vec{v}),
        \]
        i per la definició de \myref{def:producte-escalar} trobem que
        \[
            \prodesc{\vec{u}\prodvec\vec{v}}{\vec{u}\prodvec\vec{v}}>0.
        \]
        Per tant tenim que
        \[
            \det(\vec{u},\vec{v},\vec{u}\prodvec\vec{v})>0,
        \]
        i per la definició de \myref{def:base-positiva} hem acabat.
    \end{proof}
    \begin{proposition}[Fórmula de Lagrange]
        \labelname{fórmula de Lagrange}\label{prop:formula-de-Lagrange}
        Siguin~\(\vec{u}\) i~\(\vec{v}\) dos vectors de~\(\mathbb{R}^{3}\).
        Aleshores per a tots~\(\vec{x}\) i~\(\vec{y}\) de~\(\mathbb{R}^{3}\) es satisfà
        \[\prodesc{\vec{u}\prodvec\vec{v}}{\vec{x}\prodvec\vec{y}}=\det\left[\begin{matrix}
            \prodesc{\vec{u}}{\vec{x}} & \prodesc{\vec{v}}{\vec{x}} \\
            \prodesc{\vec{u}}{\vec{y}} & \prodesc{\vec{v}}{\vec{y}}
        \end{matrix}\right].\]
    \end{proposition}
    \begin{proof}
        %TODO
    \end{proof}
    \begin{proposition}[Fórmula de Leibniz]
        \labelname{}
        \label{prop:formula-de-Leibniz}
        \label{prop:formula-per-la-derivada-del-producte-vectorial-de-dues-corbes}
        Siguin~\(\alpha\) i~\(\beta\) dues corbes sobre~\(I\) en~\(\mathbb{R}^{3}\) diferenciables en~\(I\).
        Aleshores
        \[
            \frac{\diff(\alpha(t)\prodvec\beta(t))}{\diff t}=\frac{\diff\alpha(t)}{\diff t}\prodvec\beta(t)+\alpha(t)\prodvec\frac{\diff\beta(t)}{\diff t}.
        \]
    \end{proposition}
    \begin{proof}
        Per la definició de \myref{def:derivada} tenim que
        \begin{equation}
            \label{prop:formula-de-Leibniz:eq2}
            \frac{\diff(\alpha(t)\prodvec\beta(t))}{\diff t}=\lim_{h\to0}\frac{\alpha(t+h)\prodvec\beta(t+h)-\alpha(t)\prodvec\beta(t)}{h}.
        \end{equation}
        Per la definició de \myref{def:producte-vectorial} tenim que per a tot vector~\(\vec{x}\) de~\(\mathbb{R}^{3}\) es satisfà
        \[
            \prodesc{\alpha(t+h)\prodvec\beta(t+h)}{\vec{x}}=\det(\alpha(t+h),\beta(t+h),\vec{x}).
        \]
        De nou per la definició de \myref{def:derivada} trobem que quan~\(h\to0\) es satisfà
        \begin{equation}
            \label{prop:formula-de-Leibniz:eq1}
            \alpha(t+h)=\alpha(t)+h\alpha'(t)\quad\text{i}\quad\beta(t+h)=\beta(t)+h\beta'(t),
        \end{equation}
        i per tant tenim que per a tot vector~\(\vec{x}\) de~\(\mathbb{R}^{3}\) es satisfà
        \begin{align*}
            \prodesc{\alpha(t+h)\prodvec\beta(t+h)}{\vec{x}}&=\det(\alpha(t+h),\beta(t+h),\vec{x}) \\
            &=\det(\alpha(t)+h\alpha'(t),\beta(t)+h\beta'(t),\vec{x}) \tag{\ref{prop:formula-de-Leibniz:eq1}}
        \end{align*}
        i per la definició de \myref{def:determinant-duna-matriu} tenim que
        \begin{multline*}
            \det(\alpha(t)+h\alpha'(t),\beta(t)+h\beta'(t),\vec{x})=\\
            =\det(h\alpha'(t),\beta(t)+h\beta'(t),\vec{x})+\det(\alpha(t),\beta(t)+h\beta'(t),\vec{x})
        \end{multline*}
        i de nou per la definició de \myref{def:determinant-duna-matriu} trobem que
        \begin{multline*}
            \det(\alpha(t),\beta(t)+h\beta'(t),\vec{x})+\det(h\alpha'(t),\beta(t)+h\beta'(t),\vec{x})=\\
            =\det(\alpha(t),\beta(t),\vec{x})+\det(\alpha(t),h\beta'(t),\vec{x})+\\
            +\det(h\alpha'(t),\beta(t),\vec{x})+\det(h\alpha'(t),h\beta'(t),\vec{x}),
        \end{multline*}
        i per la definició de \myref{def:producte-vectorial} això és
        \begin{multline*}
            \prodesc{\alpha(t+h)\prodvec\beta(t+h)}{\vec{x}}=\\
            =\prodesc{\alpha(t)\prodvec\beta(t)}{\vec{x}}+\prodesc{\alpha(t)\prodvec h\beta'(t)}{\vec{x}}+\prodesc{h\alpha'(t)\prodvec\beta(t)}{\vec{x}}+\prodesc{h\alpha'(t)\prodvec h\beta'(t)}{\vec{x}},
        \end{multline*}
        i de nou per la definició de \myref{def:producte-vectorial} trobem que
        \begin{multline*}
            \prodesc{\alpha(t+h)\prodvec\beta(t+h)}{\vec{x}}=\\
            =\prodesc{\alpha(t)\prodvec\beta(t)+h\alpha(t)\prodvec\beta'(t)+h\alpha'(t)\prodvec\beta(t)+h^{2}\alpha'(t)\prodvec \beta'(t)}{\vec{x}}.
        \end{multline*}

        Ara bé, ens queda que per a tot vector~\(\vec{x}\) de~\(\mathbb{R}^{3}\) es satisfà
        \[
            \prodesc{\alpha(t+h)\prodvec\beta(t+h)}{\vec{x}}=\det(\alpha(t+h),\beta(t+h),\vec{x})
        \]
        i
        \begin{multline*}
            \prodesc{\alpha(t)\prodvec\beta(t)+h\alpha(t)\prodvec\beta'(t)+h\alpha'(t)\prodvec\beta(t)+h^{2}\alpha'(t)\prodvec \beta'(t)}{\vec{x}}=\\=\det(\alpha(t+h),\beta(t+h),\vec{x}),
        \end{multline*}
        i per la proposició \myref{prop:unicitat-del-producte-vectorial-entre-dos-vectors} trobem que
        \[
            \alpha(t+h)\prodvec\beta(t+h)=\alpha(t)\prodvec\beta(t)+h\alpha(t)\prodvec\beta'(t)+h\alpha'(t)\prodvec\beta(t)+h^{2}\alpha'(t)\prodvec \beta'(t).
        \]
        Per tant per \eqref{prop:formula-de-Leibniz:eq2} tenim que
        \begin{multline*}
            \frac{\diff(\alpha(t)\prodvec\beta(t))}{\diff t}=\lim_{h\to0}\frac{\alpha(t+h)\prodvec\beta(t+h)-\alpha(t)\prodvec\beta(t)}{h}\\
            =\lim_{h\to0}\frac{h\alpha(t)\prodvec\beta'(t)+h\alpha'(t)\prodvec\beta(t)+h^{2}\alpha'(t)\prodvec \beta'(t)}{h}\\
            =\alpha(t)\prodvec\beta'(t)+\alpha'(t)\prodvec\beta(t)+\lim_{h\to0}\frac{h^{2}\alpha'(t)\prodvec \beta'(t)}{h}\\
            =\alpha(t)\prodvec\beta'(t)+\alpha'(t)\prodvec\beta(t),
        \end{multline*}
        i trobem
        \[
            \frac{\diff(\alpha(t)\prodvec\beta(t))}{\diff t}=\frac{\diff\alpha(t)}{\diff t}\prodvec\beta(t)+\alpha(t)\prodvec\frac{\diff\beta(t)}{\diff t}.\qedhere
        \]
    \end{proof}
    \subsection{Fórmules de Frenet}
    \begin{proposition}
        \label{prop:la-primera-derivada-i-la-segona-derivada-duna-corba-son-perpendiculars}
        Sigui~\(\alpha\) una corba parametritzada per l'arc sobre~\(I\) tal que~\(\alpha\) és dues vegades diferenciable en un punt~\(s_{0}\).
        Aleshores tenim que~\(\alpha'(s_{0})\) i~\(\alpha''(s_{0})\) són perpendiculars.
%        \begin{proof}
    \begin{proof}
        Per la definició de \myref{def:corba-parametritzada-per-larc} trobem que per a tot~\(s\in I\) es satisfà
        \[
            \norm{\alpha'(s)}=1.
        \]
        Aleshores tenim que
        \begin{align*}
            0&=\frac{\diff}{\diff s}1 \\
            &=\frac{\diff}{\diff s}\norm{\alpha'(s_{0})} \\
            &=\frac{\diff}{\diff s}\norm{\alpha'(s_{0})}^{2} \\
            &=\frac{\diff}{\diff s}\prodesc{\alpha'(s_{0})}{\alpha'(s_{0})} \tag{\ref{def:norma-dun-vector}} \\
            &=\prodesc{\alpha''(s_{0})}{\alpha'(s_{0})}+\prodesc{\alpha'(s_{0})}{\alpha''(s_{0})} \tag{\ref{prop:derivada-del-producte-escalar-de-dues-corbes}} \\
            &=2\prodesc{\alpha'(s_{0})}{\alpha''(s_{0})}.
        \end{align*}
        Per tant trobem que
        \[
            \prodesc{\alpha'(s_{0})}{\alpha''(s_{0})}=0,
        \]
        i per la definició de \myref{def:vectors-perpendiculars} tenim que~\(\alpha'(s_{0})\) i~\(\alpha''(s_{0})\) són perpendiculars, com volíem veure.
    \end{proof}
%            Per la definició de \myref{def:derivada} trobem que
%            \[\alpha''(t_{0})=\lim_{h\to0}\frac{\alpha'(t_{0}+h)-\alpha'(t_{0})}{h},\]
%            i per tant tenim que
%            \[\prodesc{\alpha'(t_{0})}{\alpha''(t_{0})}=\prodesc[\Big]{\alpha'(t_{0})}{\lim_{h\to0}\frac{\alpha'(t_{0}+h)-\alpha'(t_{0})}{h}},\]
%            i per la definició de \myref{def:producte-escalar} trobem que
%            \begin{align*}
%                \prodesc{\alpha'(t_{0})}{\alpha''(t_{0})}    &=\prodesc[\big]{\alpha'(t_{0})}{\lim_{h\to0}\frac{\alpha'(t_{0})}{h}}-\prodesc[\big]{\alpha'(t_{0})}{\lim_{h\to0}\frac{\alpha'(t_{0}+h)}{h}} \\
%                &=\prodesc{\alpha'(t_{0})}{\vec{0}}-\prodesc{\alpha'(t_{0})}{\vec{0}}=0,
%            \end{align*}
%            i per la definició de \myref{def:vectors-perpendiculars} hem acabat.
%        \end{proof}
    \end{proposition}
    \begin{definition}[Curvatura]
        \labelname{curvatura}\label{def:curvatura}
        Sigui~\(\alpha\) una corba parametritzada per l'arc sobre~\(I\) i dues vegades diferenciable.
        Aleshores direm que l'aplicació
        \begin{align*}
            \curvatura_{\alpha}\colon I&\longrightarrow\mathbb{R} \\
            s&\longmapsto\norm{\alpha''(s)}
        \end{align*}
        és la curvatura de~\(\alpha\).

        Si~\(\curvatura_{\alpha}(s)\neq0\) direm que~\(\alpha\) té curvatura no \nulla{} en~\(s\).
    \end{definition}
    \begin{example}
        \label{ex:curvatura-de-la-circumferencia}
        Volem calcular la curvatura d'una circumferència.% de radi~\(R\).
    \end{example}
    \begin{solution}
        Per l'exercici \myref{ex:circumferencia-de-radi-R-parametritzat-per-larc} tenim que podem parametritzar una circumferència de radi~\(R\) per l'arc com
        \[
            \alpha(s)=\left(-R\sin\left(\frac{s}{R}\right),R\cos\left(\frac{s}{R}\right)\right).
        \]

        Aleshores tenim que
        \[
            \alpha''(s)=\frac{\diff}{\diff s}\left(-\cos\left(\frac{s}{R}\right),-\sin\left(\frac{s}{R}\right)\right)=\left(\frac{1}{R}\sin\left(\frac{s}{R}\right),\frac{-1}{R}\cos\left(\frac{s}{R}\right)\right),
        \]
        i trobem que
        \[
            \norm{\alpha''(s)}=\sqrt{\frac{1}{R^{2}}\sin^{2}\left(\frac{s}{R}\right)+\frac{1}{R^{2}}\cos^{2}\left(\frac{s}{R}\right)}=\frac{1}{R},
        \]
        i per la definició de \myref{def:curvatura} trobem que~\(\curvatura_{\alpha}(s)=\frac{1}{R}\).
    \end{solution}
    \begin{proposition}
        \label{prop:una-corba-te-curvatura-zero-si-i-nomes-si-es-una-recta}
        Sigui~\(\alpha\) una corba sobre~\(I\) parametritzada per l'arc i dues vegades diferenciable.
        Aleshores~\(\alpha\) és una recta si i només si~\(\curvatura_{\alpha}(s)=0\) per a tot~\(s\in I\).
    \end{proposition}
    \begin{proof}
        Comencem veient que la condició és necessària (\(\implica\)).
        Suposem doncs que~\(\alpha\) és una recta sobre~\(\mathbb{R}^{n}\).
        Tenim que existeixen~\(K_{1}\) i~\(K_{2}\) de~\(\mathbb{R}^{n}\) tals que
        \[
            \alpha(s)=K_{1}s+K_{2}.
        \]
        Aleshores tenim que~\(\alpha''(s)=\vec{0}\), i per la definició de \myref{def:curvatura} trobem que~\(\curvatura_{\alpha}(s)=0\).

        Veiem ara que la condició és suficient (\(\implicatper\)).
        Suposem doncs que~\(\curvatura_{\alpha}(s)=0\) per a tot~\(s\in I\).
        Tenim doncs que~\(\alpha''(s)=\vec{0}\), i per tant existeixen~\(K_{1}\) i~\(K_{2}\) de~\(\mathbb{R}^{n}\) tals que~\(\alpha'(s)=K_{1}\) i~\(\alpha''(s)=K_{1}s+K_{2}\) i trobem que~\(\alpha\) és una recta.
    \end{proof}
    \begin{definition}[Normal]
        \labelname{normal}\label{def:normal}
        Sigui~\(\alpha\) una corba sobre~\(I\) parametritzada per l'arc i dues vegades diferenciable amb curvatura no \nulla{}.
        Aleshores direm que l'aplicació
        \begin{align*}
            \normal_{\alpha}\colon I&\longrightarrow\mathbb{R} \\
            s&\longmapsto\frac{\alpha''(s)}{\norm{\alpha''(s)}}
        \end{align*}
        és la normal de~\(\alpha\).
    \end{definition}
    \begin{observation}
        \label{obs:la-normal-a-una-corba-es-unitaria}
        Tenim que~\(\norm{\normal_{\alpha}(s)}=1\).
    \end{observation}
    \begin{definition}[Tangent]
        \labelname{tangent}\label{def:tangent}
        Sigui~\(\alpha\) una corba sobre~\(I\) parametritzada per l'arc.
        Aleshores direm que l'aplicació
        \begin{align*}
            \tangent_{\alpha}\colon I&\longrightarrow\mathbb{R} \\
            s&\longmapsto\alpha'(s)
        \end{align*}
        és la tangent de~\(\alpha\).
    \end{definition}
    \begin{observation}
        \label{obs:la-derivada-de-la-tangent-es-la-curvatura-per-la-normal}
        Tenim que~\(\tangent_{\alpha}'(s)=\curvatura_{\alpha}(s)\normal_{\alpha}(s)\).
    \end{observation}
    \begin{proposition}
        \label{prop:la-tangent-i-la-normal-duna-corba-amb-curvatura-no-nula-son-perpendiculars}
        Sigui~\(\alpha\) una corba sobre~\(I\) amb curvatura no \nulla{} i~\(s\in I\) un real.
        Aleshores els vectors~\(\tangent_{\alpha}(s)\) i~\(\normal_{\alpha}(s)\) són perpendiculars.
    \end{proposition}
    \begin{proof}
        Considerem
        \[
            \prodesc{\tangent_{\alpha}(s)}{\normal_{\alpha}(s)}.
        \]
        Per la definició de \myref{def:tangent} i \myref{def:normal} trobem que
        \begin{align*}
            \prodesc{\tangent_{\alpha}(s)}{\normal_{\alpha}(s)}&=\prodesc[\bigg]{\alpha'(s)}{\frac{\alpha''(s)}{\norm{\alpha''(s)}}} \\
            &=\frac{1}{\norm{\alpha''(s)}}\prodesc{\alpha'(s)}{\alpha''(s)} \tag{\ref{def:producte-escalar}} \\
            &=0.
            \tag{\ref{prop:la-primera-derivada-i-la-segona-derivada-duna-corba-son-perpendiculars}}
        \end{align*}
        Aleshores per la definició de \myref{def:vectors-perpendiculars} trobem que~\(\tangent_{\alpha}(s)\) i~\(\normal_{\alpha}(s)\) són perpendiculars.
    \end{proof}
    \begin{proposition}
        \label{prop:circumferencia-osculadora}
        Siguin~\(\alpha\) una corba parametritzada per l'arc sobre~\(I\) i~\(s_{0}\in I\) un punt tal que~\(\alpha''(s_{0})\neq\vec{0}\).
        Aleshores existeix una única circumferència~\(\beta\) en~\(\mathbb{R}^{3}\) tal que~\(\alpha\) i~\(\beta\) tenen contacte d'ordre~\(2\) en~\(\alpha(s_{0})\), que té radi~\(\radicircosculadora_{\alpha}(s_{0})=\frac{1}{\curvatura_{\alpha}(s_{0})}\) i és de la forma
        \[
            \beta(s)=Q+\radicircosculadora_{\alpha}(s_{0})\left(-\normal_{\alpha}(s_{0})\cos\left(\frac{s}{\radicircosculadora_{\alpha}(s_{0})}\right)+\tangent_{\alpha}(s_{0})\sin\left(\frac{s}{\radicircosculadora_{\alpha}(s_{0})}\right)\right).
        \]
    \end{proposition}
    \begin{proof}
        %TODO
%            Sigui~\((\vec{u}_{1},\vec{u}_{2})\) una base ortonormal de~\(\mathbb{R}^{2}\). Aleshores les circumferències de~\(\mathbb{R}^{3}\) de radi~\(R\) i centre~\(Q\) parametritzades per l'arc són de la forma
%            \[\beta(t)=Q+R\left(\cos\left(\frac{t}{R}\right),\sin\left(\frac{t}{R}\right)\right).\]
%
%            Per la proposició \myref{prop:contacte-r-es-equivalent-a-tenir-les-r-primeres-derivades-iguals} tenim que
%            \[\alpha(t_{0})=\beta(t_{0}),\quad\alpha'(t_{0})=\beta'(t_{0})\quad\text{i}\quad\alpha''(t_{0})=\beta''(t_{0}).\]
%            Tenim que
%            \[\beta'(t)=\]
    \end{proof}
    \begin{definition}[Circumferència osculadora]
        \labelname{circumferència osculadora}\label{def:circumferencia-osculadora}
        Sigui~\(\alpha\) una corba parametritzada per l'arc sobre~\(I\) i~\(s_{0}\in I\) un punt tal que~\(\alpha''(s_{0})\neq\vec{0}\) i~\(\radicircosculadora_{\alpha}(s_{0})=\frac{1}{\curvatura_{\alpha}(s_{0})}\).
        Aleshores direm que la circumferència
        \[
            \beta(s)=Q+\radicircosculadora_{\alpha}(s_{0})\left(-\normal_{\alpha}(s_{0})\cos\left(\frac{s}{\radicircosculadora_{\alpha}(s_{0})}\right)+\tangent_{\alpha}(s_{0})\sin\left(\frac{s}{\radicircosculadora_{\alpha}(s_{0})}\right)\right).
        \]
        és la circumferència osculadora de~\(\alpha\) en~\(s_{0}\).
        Denotarem per~\(\radicircosculadora_{\alpha}(s_{0})=\frac{1}{\curvatura_{\alpha}(s_{0})}\) el radi de la circumferència osculadora.

        Aquesta definició té sentit per la proposició \myref{prop:circumferencia-osculadora}.
    \end{definition}
    \begin{definition}[Binormal]
        \labelname{binormal}\label{def:binormal}
        Sigui~\(\alpha\) una corba amb curvatura no \nulla{}.
        Aleshores definim
        \[
            \binormal_{\alpha}(s)=\tangent_{\alpha}(s)\prodvec\normal_{\alpha}(s)
        \]
        com la binormal de~\(\alpha\).
    \end{definition}
    \begin{proposition}
        \label{prop:triedre-de-Frenet}
        Sigui~\(\alpha\) una corba amb curvatura no \nulla{}.
        Aleshores~\((\tangent_{\alpha}(s),\normal_{\alpha}(s),\binormal_{\alpha}(s))\) és una base ortonormal i positiva de~\(\mathbb{R}^{3}\).
    \end{proposition}
    \begin{proof}
        Per la proposició \myref{prop:la-tangent-i-la-normal-duna-corba-amb-curvatura-no-nula-son-perpendiculars} trobem que~\(\tangent_{\alpha}(s)\) i~\(\normal_{\alpha}(s)\) són perpendiculars.
        Aleshores per la definició de \myref{def:binormal} trobem que~\(\binormal_{\alpha}(s)=\tangent_{\alpha}(s)\prodvec\normal_{\alpha}(s)\) i per la proposició \myref{prop:dos-vectors-linealment-independents-i-el-seu-producte-vectorial-formen-una-base-positiva} tenim que la base~\((\tangent_{\alpha}(s),\normal_{\alpha}(s),\binormal_{\alpha}(s))\) és positiva.

        Per la definició de \myref{def:producte-vectorial} tenim que per a tot~\(\vec{x}\) de~\(\mathbb{R}^{3}\) es satisfà
        \[
            \prodesc{\binormal_{\alpha}(s)}{\vec{x}}=\det(\tangent_{\alpha}(s),\normal_{\alpha}(s),\vec{x}).
        \]
        Si prenem~\(\vec{x}\) tal que~\((\tangent_{\alpha}(s),\normal_{\alpha}(s),\vec{x})\) sigui una base ortonormal de~\(\mathbb{R}^{3}\) trobem que
        \[
            \det(\tangent_{\alpha}(s),\normal_{\alpha}(s),\vec{x})=1,
        \]
        i per tant
        \[
            \prodesc{\binormal_{\alpha}(s)}{\vec{x}}=1,
        \]
        i tenim que~\(\norm{\binormal_{\alpha}(s)}=1\).
        Aleshores per l'observació \myref{obs:la-normal-a-una-corba-es-unitaria} i la definició de \myref{def:corba-parametritzada-per-larc} tenim que~\(\norm{\tangent_{\alpha}(s)}=1\) i que~\(\norm{\normal_{\alpha}(s)}=1\), i per tant~\((\tangent_{\alpha}(s),\normal_{\alpha}(s),\binormal_{\alpha}(s))\) és una base ortonormal i positiva de~\(\mathbb{R}^{3}\).
        % REF
    \end{proof}
    \begin{definition}[Triedre de Frenet]
        \labelname{triedre de Frenet}\label{def:triedre-de-Frenet}
        Sigui\(\alpha\) una corba amb curvatura no \nulla{}.
        Aleshores direm que la base~\((\tangent_{\alpha}(s),\normal_{\alpha}(s),\binormal_{\alpha}(s))\) és el triedre de Frenet de~\(\alpha\) en~\(s\).

        Aquesta definició té sentit per la proposició \myref{prop:triedre-de-Frenet}.
    \end{definition}
    \begin{proposition}
        \label{prop:torsio}
        Sigui~\(\alpha\) una corba amb curvatura no \nulla{}.
        Aleshores existeix una~\(\torsio_{\alpha}(s)\in\mathbb{R}\) tal que
        \[
            \binormal'_{\alpha}(s)=\torsio_{\alpha}(s)\normal_{\alpha}(s).
        \]
    \end{proposition}
    \begin{proof}
        Tenim per la proposició \myref{prop:triedre-de-Frenet} que~\((\tangent_{\alpha}(s),\normal_{\alpha}(s),\binormal_{\alpha}(s))\) és una base ortonormal de~\(\mathbb{R}^{3}\), i per tant tenim que %REF
        \[
            \binormal'_{\alpha}(s)=\prodesc{\binormal'_{\alpha}(s)}{\tangent_{\alpha}(s)}\tangent_{\alpha}(s)+\prodesc{\binormal'_{\alpha}(s)}{\normal_{\alpha}(s)}\normal_{\alpha}(s)+\prodesc{\binormal'_{\alpha}(s)}{\binormal_{\alpha}(s)}\binormal_{\alpha}(s).
        \]
        Tenim que~\(\prodesc{\binormal_{\alpha}(s)}{\binormal_{\alpha}(s)}=1\), per tant trobem que
        \[
            \frac{\diff\prodesc{\binormal_{\alpha}(s)}{\binormal_{\alpha}(s)}}{\diff s}=0,
        \]
        i per la proposició \myref{prop:formula-de-Leibniz} trobem que
        \[
            \prodesc{\binormal'_{\alpha}}{\binormal_{\alpha}}+\prodesc{\binormal_{\alpha}}{\binormal'_{\alpha}}=0,
        \]
        i per la definició de \myref{def:producte-escalar} tenim que
        \[
            2\prodesc{\binormal'_{\alpha}(s)}{\binormal_{\alpha}(s)}=0,
        \]
        i per tant
        \[
            \prodesc{\binormal'_{\alpha}(s)}{\binormal_{\alpha}(s)}=0.
        \]

        Per la proposició \myref{prop:triedre-de-Frenet} trobem que
        \[
            \prodesc{\binormal_{\alpha}(s)}{\tangent_{\alpha}(s)}=0,
        \]
        i per tant
        \[
            \frac{\diff\prodesc{\binormal_{\alpha}(s)}{\tangent_{\alpha}(s)}}{\diff s}=0,
        \]
        i de nou per la proposició \myref{prop:formula-de-Leibniz} tenim que
        \[
            \frac{\diff\prodesc{\binormal_{\alpha}(s)}{\tangent_{\alpha}(s)}}{\diff s}=\prodesc{\binormal'_{\alpha}(s)}{\tangent_{\alpha}(s)}+\prodesc{\binormal_{\alpha}(s)}{\tangent'_{\alpha}(s)}.
        \]
        Ara bé, tenim que
        \begin{align*}
            \prodesc{\binormal_{\alpha}(s)}{\tangent'_{\alpha}(s)}&=\prodesc{\binormal_{\alpha}(s)}{\curvatura_{\alpha}(s)\normal_{\alpha}(s)} \tag{\ref{obs:la-derivada-de-la-tangent-es-la-curvatura-per-la-normal}} \\
            &=0, \tag{\ref{prop:triedre-de-Frenet}}
        \end{align*}
        i per tant
        \[
            \prodesc{\binormal'_{\alpha}(s)}{\tangent_{\alpha}(s)}=0,
        \]
        i per tant, denotant~\(\torsio_{\alpha}(s)=\prodesc{\binormal'_{\alpha}(s)}{\normal_{\alpha}(s)}\) trobem que
        \[
            \binormal'_{\alpha}(s)=\torsio_{\alpha}(s)\normal_{\alpha}(s).\qedhere
        \]
    \end{proof}
    \begin{definition}[Torsió]
        \labelname{torsió}\label{def:torsio}
        Sigui~\(\alpha\) una corba parametritzada per l'arc amb curvatura no \nulla{} i~\(\torsio_{\alpha}(s)\) el real tal que
        \[
            \binormal'_{\alpha}(s)=\torsio_{\alpha}(s)\normal_{\alpha}(s).
        \]
        Aleshores direm que~\(\torsio_{\alpha}(s)\) és la torsió de~\(\alpha\).

        Aquesta definició té sentit per la proposició \myref{prop:torsio}.
    \end{definition}
    % FER Exemple calcular torsió?
    \begin{theorem}[Fórmules de Frenet]
        \labelname{fórmules de Frenet}\label{thm:formules-de-Frenet-per-corbes-parametritzades-per-larc}
        Sigui~\(\alpha\) una corba parametritzada per l'arc amb curvatura no \nulla{}.
        Aleshores
        \[\begin{bmatrix}
            \tangent'_{\alpha}(s) \\
            \normal'_{\alpha}(s) \\
            \binormal'_{\alpha}(s)
        \end{bmatrix}=
        \begin{bmatrix}
            0 & \curvatura_{\alpha}(s) & 0 \\
            -\curvatura_{\alpha}(s) & 0 & -\torsio_{\alpha}(s) \\
            0 & \torsio_{\alpha}(s) & 0
        \end{bmatrix}
        \begin{bmatrix}
            \tangent_{\alpha}(s) \\
            \normal_{\alpha}(s) \\
            \binormal_{\alpha}(s)
        \end{bmatrix}.\]
    \end{theorem}
    \begin{proof}
        Per l'observació \myref{obs:la-derivada-de-la-tangent-es-la-curvatura-per-la-normal} trobem que
        \begin{equation}
            \label{thm:formules-de-Frenet:eq1}
            \tangent_{\alpha}'(s)=\curvatura_{\alpha}(s)\normal_{\alpha}(s),
        \end{equation}
        i per la definició de \myref{def:torsio} trobem que
        \begin{equation}
            \label{thm:formules-de-Frenet:eq2}
            \binormal'_{\alpha}(s)=\torsio_{\alpha}(s)\normal_{\alpha}(s).
        \end{equation}

        Per la definició de \myref{def:binormal} trobem que
        \[
            \binormal_{\alpha}(s)=\tangent_{\alpha}(s)\prodvec\normal_{\alpha}(s),
        \]
        i per tant % REF
        \[
            \normal_{\alpha}(s)=\binormal_{\alpha}(s)\prodvec\tangent_{\alpha}(s).
        \]
        Aleshores per la proposició \myref{prop:formula-de-Leibniz} tenim que
        \[
            \normal'_{\alpha}(t)=\binormal'_{\alpha}(s)\prodvec\tangent_{\alpha}(s)+\binormal_{\alpha}(s)\prodvec\tangent'_{\alpha}(s).
        \]
        Ara bé, per \eqref{thm:formules-de-Frenet:eq1} i \eqref{thm:formules-de-Frenet:eq2} trobem que
        \begin{align*}
            \normal'_{\alpha}(s)&=\torsio_{\alpha}(s)\normal_{\alpha}(s)\prodvec\tangent_{\alpha}(s)+\binormal_{\alpha}(s)\prodvec\curvatura_{\alpha}(s)\normal_{\alpha}(s) \\
            &=\torsio_{\alpha}(s)\big(\normal_{\alpha}(s)\prodvec\tangent_{\alpha}(s)\big)+\curvatura_{\alpha}(s)\big(\binormal_{\alpha}(s)\prodvec\normal_{\alpha}(s)\big) \\ % REF
            &=-\torsio_{\alpha}(s)\binormal_{\alpha}(s)+\curvatura_{\alpha}(s)\big(\binormal_{\alpha}(s)\prodvec\normal_{\alpha}(s)\big) \tag{\ref{def:binormal}} \\
            &=-\torsio_{\alpha}(s)\binormal_{\alpha}(s)-\curvatura_{\alpha}(s)\tangent_{\alpha}(s).
            \tag{\ref{def:binormal}}
        \end{align*}

        Per tant ens queda
        \begin{align*}
            \tangent_{\alpha}'(s)&=\curvatura_{\alpha}(s)\normal_{\alpha}(s) \\
            \normal'_{\alpha}(s)&=-\torsio_{\alpha}(s)\binormal_{\alpha}(s)-\curvatura_{\alpha}(s)\tangent_{\alpha}(s) \\
            \binormal'_{\alpha}(s)&=\torsio_{\alpha}(s)\normal_{\alpha}(s)
        \end{align*}
        i per la definició de \myref{def:producte-de-matrius} hem acabat.
    \end{proof}
    \begin{proposition}
        \label{prop:condicio-equivalent-per-que-la-reparametritzacio-per-larc-duna-corba-tingui-curvatura-no-nula}
        Sigui~\(\alpha\) una corba amb curvatura no \nulla{} i~\(h\) un canvi de paràmetre de~\(\alpha\) tal que~\(\tilde{\alpha}=\alpha\circ h\) estigui parametritzada per l'arc.
        Aleshores~\(\tilde{\alpha}\) té curvatura no \nulla{} si i només si~\(\alpha'\prodvec\alpha''\neq\vec{0}\).
    \end{proposition}
    \begin{proof}
        Tenim que~\(\tilde{\alpha}=\alpha\circ h\), i per la proposició \myref{prop:formula-de-Leibniz} i la \myref{thm:regla-de-la-cadena} tenim que
        \[
            \alpha'=h'(\tilde{\alpha}'\circ h)
        \]
        i
        \[
            \alpha''=h''(\tilde{\alpha}'\circ h)+(h')^{2}(\tilde{\alpha}''\circ h).
        \]
        Per tant tenim que
        \begin{align*}
            \alpha'\prodvec\alpha''&=(h'(\tilde{\alpha}'\circ
             h))\prodvec(h''(\tilde{\alpha}'\circ h)+(h')^{2}(\tilde{\alpha}''\circ h)) \\
             &=(h'(\tilde{\alpha}'\circ h))\prodvec(h''(\tilde{\alpha}'\circ h))+(h'(\tilde{\alpha}'\circ h))\prodvec((h')^{2}(\tilde{\alpha}''\circ h)) \\
             &=h'h''(\tilde{\alpha}'\circ h)\prodvec(\tilde{\alpha}'\circ h)+(h')^{2}(h'(\tilde{\alpha}'\circ h))\prodvec(\tilde{\alpha}''\circ h) \\
            &=(h')^{2}(\tilde{\alpha}'\circ h)\prodvec(\tilde{\alpha}''\circ h).
        \end{align*}

        Ara bé, per la definició de \myref{def:canvi-de-parametre} tenim que~\(h\) és un difeomorfisme, i per la definició de \myref{def:difeomorfisme} trobem que~\(h\neq0\).
        Tenim també que~\(h''\neq0\), i per tant trobem que
        \[
            \alpha'\prodvec\alpha''=\vec{0}\Sii(\tilde{\alpha}'\circ h)\prodvec(\tilde{\alpha}''\circ h)=\vec{0},
        \]
        i com que, per hipòtesi,~\(\tilde{\alpha}\) està parametritzada per l'arc, per la definició de \myref{def:corba-parametritzada-per-larc} tenim que~\(\tilde{\alpha}'\circ h\neq\vec{0}\), i per la proposició \myref{prop:la-primera-derivada-i-la-segona-derivada-duna-corba-son-perpendiculars} trobem que~\((\tilde{\alpha}'\circ h)\prodvec(\tilde{\alpha}''\circ h)=\vec{0}\) si i només si~\(\tilde{\alpha}''=\vec{0}\), i per tant
        \[
            \alpha'\prodvec\alpha''=\vec{0}\Sii\norm{\tilde{\alpha}''}=\vec{0}.\qedhere
        \]
    \end{proof}
    \begin{definition}[Curvatura per una reparametrització]
        \labelname{curvatura}\label{def:curvatura  per una reparametrització}
        Sigui~\(\alpha\) una corba amb curvatura no \nulla{} i~\(\alpha'\prodvec\alpha''=\vec{0}\) i~\(h\) un canvi de paràmetre de~\(\alpha\) tal que~\(\tilde{\alpha}=\alpha\circ h\) estigui parametritzada per l'arc.
        Aleshores direm que
        \[
            \curvatura_{\alpha}=\curvatura_{\tilde{\alpha}}\circ h^{-1}
        \]
        és la curvatura de~\(\alpha\).

        Això té sentit per la definició de \myref{def:curvatura} i la proposició \myref{prop:condicio-equivalent-per-que-la-reparametritzacio-per-larc-duna-corba-tingui-curvatura-no-nula}.
    \end{definition}
    \begin{definition}[Torsió per una reparametrització]
        \labelname{torsió}\label{def:torsió  per una reparametrització}
        Sigui~\(\alpha\) una corba amb curvatura no \nulla{} i~\(h\) un canvi de paràmetre de~\(\alpha\) tal que~\(\tilde{\alpha}=\alpha\circ h\) estigui parametritzada per l'arc.
        Aleshores direm que
        \[
            \torsio_{\alpha}=\torsio_{\tilde{\alpha}}\circ h^{-1}
        \]
        és la curvatura de~\(\alpha\).

        Aquesta definició té sentit per la definició de \myref{def:torsio}.
    \end{definition}
    \begin{definition}[Rapidesa]
        \labelname{rapidesa}\label{def:rapidesa}
        Sigui~\(\alpha\) una corba regular.
        Aleshores direm que
        \[
            \rapidesa_{\alpha}(t)=\norm{\alpha'(t)}
        \]
        és la rapidesa de~\(\alpha\).
    \end{definition}
    \begin{theorem}[Fórmules de Frenet]
        \labelname{fórmules de Frenet}\label{thm:formules-de-Frenet}
        Sigui~\(\alpha\) una corba amb curvatura no \nulla{} i tal que~\(\alpha'\prodvec\alpha''\neq\vec{0}\).
        Aleshores
        \[\begin{bmatrix}
            \tangent_{\alpha}'(t) \\
            \normal_{\alpha}'(t) \\
            \binormal_{\alpha}'(t)
        \end{bmatrix}
        =\rapidesa_{\alpha}(t)\begin{bmatrix}
            0 & \curvatura_{\alpha}(t) & 0 \\
            -\curvatura_{\alpha}(t) & 0 & -\torsio_{\alpha}(t) \\
            0 & \torsio_{\alpha}(t) & 0
        \end{bmatrix}
        \begin{bmatrix}
            \tangent_{\alpha}(t) \\
            \normal_{\alpha}(t) \\
            \binormal_{\alpha}(t)
        \end{bmatrix}.\]
    \end{theorem}
    \begin{proof}
        Sigui~\(h\) una reparametrització de~\(\alpha\) tal que~\(\tilde{\alpha}=\alpha\circ h\) estigui parametritzada per l'arc.

        Ai haig de tornar a definir les coses aquestes per corbes no parametritzades per l'arc.
        Quina mandra.
        Imagineu una demostració patrocinada per \myref{thm:regla-de-la-cadena} i les \myref{thm:formules-de-Frenet-per-corbes-parametritzades-per-larc}.
        Us prometo que tot quadra, com volíem veure.
    \end{proof}
    \begin{proposition}
        \label{prop:triedre-de-frenet-duna-corba-regular}
        \label{prop:tangent-duna-corba-regular}
        \label{prop:normal-duna-corba-regular}
        \label{prop:binormal-duna-corba-regular}
        Sigui~\(\alpha\) una corba regular amb~\(\alpha'\prodvec\alpha''\neq\vec{0}\).
        Aleshores
        \[
            \tangent_{\alpha}=\frac{\alpha'}{\norm{\alpha'}},\qquad\normal_{\alpha}=\binormal_{\alpha}\prodvec\binormal_{\beta}\qquad\text{i}\qquad\tangent_{\alpha}=\frac{\alpha'\prodvec\alpha''}{\norm{\alpha'\prodvec\alpha''}}.
        \]
    \end{proposition}
    \begin{proof}
        %TODO
    \end{proof}
    \begin{proposition}
        \label{prop:curvatura-i-torsio-duna-corba-regular}
        \label{prop:curvatura-duna-corba-regular}
        \label{prop:torsio-duna-corba-regular}
        Sigui~\(\alpha\) una corba regular amb~\(\alpha'\prodvec\alpha''\neq0\).
        Aleshores
        \[
            \curvatura_{\alpha}=\frac{\norm{\alpha'\prodvec\alpha''}}{\norm{\alpha'}^{3}}\qquad\text{i}\qquad\torsio_{\alpha}=\frac{\prodesc{\alpha'\prodvec\alpha''}{\alpha'''}}{\norm{\alpha'\prodvec\alpha''}^{2}}.
        \]
    \end{proposition}
    \begin{proof}
        %TODO
    \end{proof}
    \subsection{Teorema Fonamental de la teoria local de corbes}
    %TODO Es venen moltes coses que moure a altres assignatures en el futur.
    \begin{example}[Grup ortogonal]
        \labelname{}\label{ex:grup-ortogonal}
        Volem veure que el conjunt
        \[
            \GrupOrtogonal(n)=\{A\in\matrius_{n}(\mathbb{R})\mid\text{per a tot }\vec{u},\vec{v}\in\matrius_{n\times1}(\mathbb{R})\text{ tenim }\prodesc{A\vec{u}}{A\vec{v}}=\prodesc{\vec{u}}{\vec{v}}\}
        \]
        amb el producte de matrius és un grup.
    \end{example}
    \begin{solution}
        %TODO
        % Moure a estructures
    \end{solution}
    \begin{example}[Grup especial ortogonal]
        \labelname{}\label{ex:grup-especial-ortogonal}
        Volem veure que
        \[
            \GrupEspecialOrtogonal(n)=\{A\in\GrupOrtogonal\mid\det(A)=1\}
        \]
        és un subgrup de~\(\GrupOrtogonal(n)\).
    \end{example}
    \begin{solution}
        %TODO
        % Moure a estructures
    \end{solution}
    \begin{proposition}
        \label{prop:els-valors-propis-duna-matriu-ortogonal-son-1-o-1}
        Sigui~\(A\in\GrupOrtogonal(n)\) una matriu.
        Aleshores els valors propis reals de~\(A\) són~\(-1\) ó~\(1\).
    \end{proposition}
    \begin{proof}
        %TODO
    \end{proof}
    \begin{proposition}
        \label{prop:caracteritzacio-de-les-matrius-ortogonals-2x2}
        Es satisfà
        \[\GrupEspecialOrtogonal(2)=\left\{\begin{bmatrix}
            \cos(t) & -\sin(t) \\
            \sin(t) & \cos(t)
        \end{bmatrix}\mid t\in\mathbb{R}\right\}\]
        i
        \[\GrupOrtogonal(2)\setminus\GrupEspecialOrtogonal(2)=\left\{\begin{bmatrix}
            \cos(t) & \sin(t) \\
            \sin(t) & -\cos(t)
        \end{bmatrix}\mid t\in\mathbb{R}\right\}.\]
    \end{proposition}
    \begin{proof}
        %TODO
    \end{proof}
    \begin{proposition}
        Sigui~\(A\in\GrupEspecialOrtogonal(3)\) una matriu.
        Aleshores existeixen una base ortonormal~\(\base{B}\) de~\(\mathbb{R}^{3}\) i un~\(t\in\mathbb{R}\) tals que
        \[M(\base{B},\basecanonica)AM(\basecanonica,\base{B})=\begin{bmatrix}
            \pm1 & 0 & 0 \\
            0 & \cos(t) & \sin(t) \\
            0 & \sin(t) & -\cos(t)
        \end{bmatrix}.\]
    \end{proposition}
    \begin{proof}
        %TODO
    \end{proof}
    \begin{proposition}
        \label{prop:forma-matricial-de-les-aplicacions-que-conserven-distancies}
        Sigui~\(f\colon\mathbb{R}^{n}\longrightarrow\mathbb{R}^{n}\) una aplicació conserva distàncies.
        Aleshores existeixen una matriu~\(A\in\GrupOrtogonal(n)\) i un~\(\vec{C}\in\mathbb{R}^{n}\) tals que
        \[
            f(\vec{v})=A\vec{v}+\vec{C}.
        \]
    \end{proposition}
    \begin{proof}
        %TODO
    \end{proof}
    \begin{proposition}
        \label{prop:derivada-del-producte-duna-matriu-per-una-corba}
        Siguin~\(\alpha\) una corba regular i~\(A\in\matrius_{n}(\mathbb{R})\) una matriu.
        Aleshores
        \[
            \frac{\diff(A\alpha(t))}{\diff t}=A\alpha'(t).
        \]
    \end{proposition}
    \begin{proof}
        %TODO
    \end{proof}
    \begin{proposition}
        \label{prop:les-matrius-especials-ortogonals-conserven-el-producte-vectorial}
        Siguin~\(A\in\GrupEspecialOrtogonal(n)\) una matriu i~\(\vec{u}\),~\(\vec{v}\) dos vectors de~\(\mathbb{R}^{n}\).
        Aleshores
        \[
            A(\vec{u}\prodvec\vec{v})=(A\vec{u})\prodvec(A\vec{v}).
        \]
    \end{proposition}
    \begin{proof}
        %TODO
    \end{proof}
    \begin{corollary}
        \label{cor:una-corba-parametritzada-per-larc-i-la-seva-imatge-per-una-aplicacio-que-conserva-les-distancies-son-equivalents}
        Siguin~\(\alpha\) una corba parametritzada per l'arc,~\(A\in\GrupEspecialOrtogonal(3)\) una matriu,~\(\vec{C}\) un vector de~\(\mathbb{R}^{3}\) i~\(\beta\) una corba tal que
        \[
            \beta(t)=A\alpha(t)+\vec{C}.
        \]
        Aleshores~\(\beta(t)\) està parametritzada per l'arc i es satisfà
        \[
            \tangent_{\beta}=A\tangent_{\alpha},\quad\normal_{\beta}=A\normal_{\alpha},\quad\binormal_{\beta}=A\binormal_{\alpha},\quad\curvatura_{\beta}=A\curvatura_{\alpha}\quad\text{i}\quad\torsio_{\beta}=A\torsio_{\alpha}.
        \]
    \end{corollary}
    \begin{proof}
        %TODO
    \end{proof}
    \begin{lemma}
        \label{lemma:Teorema-Fonamental-de-la-teoria-local-de-corbes}
        Siguin~\(\alpha\) i~\(\beta\) dues corbes.
        Aleshores la relació
        \[
            \alpha\sim\beta \Sii \text{Existeixen }A\in\GrupEspecialOrtogonal(3)\text{ i }\vec{C}\in\mathbb{R}^{3}\text{ tals que }\alpha=A\beta+\vec{C}
        \]
        és una relació d'equivalència.
    \end{lemma}
    \begin{proof}
        %TODO
    \end{proof}
    \begin{theorem}[Teorema Fonamental de la teoria local de corbes]
        \labelname{Teorema Fonamental de la teoria local de corbes}\label{thm:Teorema-Fonamental-de-la-teoria-local-del-corbes}
        Siguin~\(\curvatura(s)\) i~\(\torsio(s)\) dues funcions diferenciables sobre~\(I\), amb~\(\curvatura(s)>0\) per a tot~\(s\in I\).
        Aleshores existeix una única corba~\(\alpha\) sobre~\(I\) parametritzada per l'arc satisfent~\(\curvatura_{\alpha}(s)=\curvatura(s)\) i~\(\torsio_{\alpha}(s)=\torsio(s)\), llevat d'equivalència.
        % Reescriure la part de llevat d'equivalència
    \end{theorem}
    \begin{proof}
        %TODO
    \end{proof}
\end{document}
