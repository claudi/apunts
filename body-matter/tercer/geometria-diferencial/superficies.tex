\documentclass[../geometria-diferencial.tex]{subfiles}

\begin{document}
\chapter{Superfícies}
\section{Superfícies regulars}
    \subsection{Immersions i submersions}
    \begin{definition}[Immersió]
        \labelname{immersió}\label{def:immersio}
        Siguin~\(n\leq m\) dos naturals,~\(\obert{U}\subseteq\mathbb{R}^{n}\) un obert,~\(x_{0}\in\obert{U}\) un punt i~\(f\colon\obert{U}\longrightarrow\mathbb{R}^{m}\) una funció diferenciable tal que~\(df(a)\) sigui injectiva.
        Aleshores direm que~\(f\) és una immersió en~\(x_{0}\).
    \end{definition}
    \begin{observation} % REF álgebra lineal
        \label{obs:immersio-si-i-nomes-si-te-rang-maximal}
        \(f\) és una immersió en~\(x_{0}\) si i només si~\(\rang(df(a))=n\).
    \end{observation}
    \begin{example}
        \label{ex:la-inclusio-canonica-es-una-immersio}
        Volem veure que la inclusió canònica és una immersió.
    \end{example}
    \begin{solution}
        La inclusió canònica ve definida com
        \begin{align*}
            i\colon\mathbb{R}^{p}&\longrightarrow\mathbb{R}^{p+q} \\
            x&\longmapsto(x;0), \tag{\ref{notation:punts-per-grups}}
        \end{align*}
        i per la definició de \myref{def:Jacobiana} tenim que la seva diferencial és
        \[\left[\begin{array}{c}
            I_{p} \\\hline
            0_{q\times(p-q)}
        \end{array}\right].\]
        Aleshores per la definició de \myref{def:rang-duna-matriu} trobem que aquesta té rang~\(p\), i per l'observació \myref{obs:immersio-si-i-nomes-si-te-rang-maximal} tenim que~\(i\) és una immersió.
    \end{solution}
    \begin{definition}[Submersió]
        \labelname{submersió}\label{def:submersio}
        Siguin~\(n\geq m\) dos naturals,~\(\obert{U}\subseteq\mathbb{R}^{n}\) un obert,~\(x_{0}\in\obert{U}\) un punt i~\(f\colon\obert{U}\longrightarrow\mathbb{R}^{m}\) una funció diferenciable tal que~\(df(a)\) sigui exhaustiva.
        Aleshores direm que~\(f\) és una submersió en~\(x_{0}\).
    \end{definition}
    \begin{observation}
        \label{obs:submersio-si-i-nomes-si-te-rang-mes-petit-o-igual}
        \(f\) és una submersió en~\(x_{0}\) si i només si~\(\rang(df(a))=m\).
    \end{observation}
    \begin{example}
        \label{ex:la-projeccio-canonica-es-una-submersio}
        Volem veure que la projecció canònica és una submersió.
    \end{example}
    \begin{solution}
        La projecció canònica ve definida com
        \begin{align*}
            \pi\colon\mathbb{R}^{p+q}&\longrightarrow\mathbb{R}^{p} \\
            (x;y)&\longmapsto x, \tag{\ref{notation:punts-per-grups}}
        \end{align*}
        i per la definició de \myref{def:Jacobiana} trobem que la seva diferencial és
        \[\left[\begin{array}{c|c}
        I_{p} & 0_{p\times(p+q)}
        \end{array}\right].\]
        Aleshores per la definició de \myref{def:rang-duna-matriu} trobem que aquesta té rang~\(p\) i per l'observació \myref{obs:submersio-si-i-nomes-si-te-rang-mes-petit-o-igual} tenim que~\(\pi\) és una submersió.
    \end{solution}
    \begin{proposition}
        \label{prop:la-composicio-de-submersions-es-submersio}
        Siguin~\(\obert{U}\subseteq\mathbb{R}^{n}\) i~\(\obert{V}\subseteq\mathbb{R}^{m}\) dos oberts,~\(g\colon\obert{U}\longrightarrow\obert{V}\) i~\(f\colon\obert{V}\longrightarrow\mathbb{R}^{d}\) dues submersions.
        Aleshores la funció~\(f\circ g\) és una submersió.
    \end{proposition}
    \begin{proof}
        Per la \myref{thm:regla-de-la-cadena} tenim que
        \[
            d(f\circ g)=dgdf(g),
        \]
        i per tant trobem que~\(\rang(d(f\circ g))\leq\min\{\rang(dg),\rang(df(g))\}\), i per l'observació \myref{obs:submersio-si-i-nomes-si-te-rang-mes-petit-o-igual} hem acabat.
    \end{proof}
    \begin{theorem}[Teorema d'estructura local de les immersions]
        \labelname{Teorema d'estructura local de les immersions}\label{thm:Teorema-destructura-local-de-les-immersions}
        Siguin~\(\obert{U}\subseteq\mathbb{R}^{n}\) un obert i~\(f\colon\obert{U}\longrightarrow\mathbb{R}^{m}\) una immersió en un punt~\(x_{0}\in\obert{U}\).
        Aleshores existeixen dos oberts~\(\obert{U}'\subseteq\mathbb{R}^{n}\) i~\(\obert{V}\subseteq\mathbb{R}^{m}\) satisfent~\(x_{0}\subseteq\obert{U}'\subseteq\obert{U}\) i existeix un difeomorfisme~\(g\colon\obert{V}\longleftrightarrow g(\obert{U}')\subseteq\mathbb{R}^{n}\) i satisfent~\(i(x_{0})\in\obert{V}\) tals que el diagrama
        \[\begin{tikzcd}
            \obert{U}' \arrow{r}{i} \arrow[swap]{dr}{f} & \obert{V} \arrow{d}{g} \\
            & \mathbb{R}^{m}
        \end{tikzcd}\]
        és un diagrama commutatiu.
    \end{theorem}
    \begin{proof}
        Denotem
        \[
            f(x)=(f_{1}(x),\dots,f_{m}(x)).
        \]
        Per la definició de \myref{def:immersio} trobem que~\(n\leq m\) i per l'observació \myref{obs:immersio-si-i-nomes-si-te-rang-maximal} trobem que
        \[
            \rang\left(\frac{\partial f_{i}}{\partial x_{j}}(x_{0})\right){}_{1\leq j\leq m}^{1\leq i\leq n}=n.
        \]
        Aleshores tenim que existeix una permutació~\(\sigma\in\GrupSimetric_{n}\) tal que
        \[
            \rang\left(\frac{\partial f_{i}}{\partial x_{\sigma(j)}}(x_{0})\right){}_{1\leq j\leq n}^{1\leq i\leq n}=n,
        \]
        i per la proposició \myref{prop:determinant-diferent-de-zero-linealment-independents} trobem que
        \[
            \det\left(\frac{\partial f_{i}}{\partial x_{\sigma(j)}}(x_{0})\right){}_{1\leq j\leq n}^{1\leq i\leq n}\neq0.
        \]
        Denotem doncs
        \[
            F(x)=(f_{\sigma(1)}(x),\dots,f_{\sigma(m)}(x))
        \]
        i tenim que
        \begin{equation}
            \label{thm:Teorema-destructura-local-de-les-immersions:eq:1}
            \det(dF(x_{0}))\neq0.
        \end{equation}

        Considerem la funció
        \begin{align*}
            g\colon\mathbb{R}^{n}\times\mathbb{R}^{m-n}&\longrightarrow\mathbb{R}^{m} \\
            (x,y)&\longmapsto F(x)+(0,y).
        \end{align*}
        Tenim que~\(g(x,0)=F(x)\), i per l'exemple \myref{ex:la-inclusio-canonica-es-una-immersio} trobem que
        \[
            (g\circ i)(x)=g(x,0)=F(x).
        \]

        Per la definició de \myref{def:Jacobiana} i la proposició \myref{prop:justificacio-Jacobiana} tenim que
        \[dg(x_{0})=\left[\begin{array}{c|ccc}
            \begin{matrix}
                \frac{\partial F_{1}}{\partial x_{1}} & \cdots & \frac{\partial F_{1}}{\partial x_{n}} \\
                \vdots & & \vdots \\
                \frac{\partial F_{n}}{\partial x_{1}} & \cdots & \frac{\partial F_{n}}{\partial x_{n}}
            \end{matrix} & & 0_{n\times(m-n)} & \\ \hline
            \begin{matrix}
                \frac{\partial F_{n+1}}{\partial x_{1}} & \cdots & \frac{\partial F_{n+1}}{\partial x_{n}} \\
                \vdots & & \vdots \\
                \frac{\partial F_{m}}{\partial x_{1}} & \cdots & \frac{\partial F_{m}}{\partial x_{n}}
            \end{matrix} & & I_{(m-n)\times(m-n)} &
        \end{array}\right].\]
        Aleshores tenim que
        \[
            \det(dg(x_{0}))=\det(dF(x_{0})),
        \]
        i per \eqref{thm:Teorema-destructura-local-de-les-immersions:eq:1} trobem que
        \[
            \det(dg(x_{0}))\neq0,
        \]
        i pel \myref{thm:Funcio-inversa} hem acabat.
    \end{proof}
    \begin{theorem}[Teorema d'estructura local de les submersions]
        \labelname{Teorema d'estructura local de les submersions}\label{thm:Teorema-destructura-local-de-les-submersions}
        Siguin~\(\obert{U}\subseteq\mathbb{R}^{n}\) un obert i~\(f\colon\obert{U}\longrightarrow\mathbb{R}^{m}\) una submersió en un punt~\(x_{0}\in\obert{U}\).
        Aleshores existeix un obert~\(\obert{U}'\) satisfent~\(x_{0}\subseteq\obert{U}'\subseteq\obert{U}\) i existeix un difeomorfisme~\(g\colon\obert{U}'\longleftrightarrow g(\obert{U}')\subseteq\mathbb{R}^{n}\) tals que el diagrama
        \[\begin{tikzcd}
            \obert{U}' \arrow{r}{g} \arrow[swap]{dr}{f} & g(\obert{U}') \arrow{d}{\pi} \\
            & \mathbb{R}^{m}
        \end{tikzcd}\]
        és un diagrama commutatiu.
    \end{theorem}
    \begin{proof}
        Denotem
        \[
            f(x)=(f_{1}(x),\dots,f_{m}(x)).
        \]
        Per la definició de \myref{def:submersio} trobem que~\(n\geq m\) i per l'observació \myref{obs:submersio-si-i-nomes-si-te-rang-mes-petit-o-igual} trobem que
        \[
            \rang\left(\frac{\partial f_{i}}{\partial x_{j}}(x_{0})\right){}_{1\leq j\leq m}^{1\leq i\leq n}=m.
        \]
        Aleshores tenim que existeix una permutació~\(\sigma\in\GrupSimetric_{m}\) tal que
        \[
            \rang\left(\frac{\partial f_{i}}{\partial x_{\sigma(j)}}(x_{0})\right){}_{1\leq j\leq m}^{1\leq i\leq m}=m,
        \]
        i per la proposició \myref{prop:determinant-diferent-de-zero-linealment-independents} trobem que
        \[
            \det\left(\frac{\partial f_{i}}{\partial x_{\sigma(j)}}(x_{0})\right){}_{1\leq j\leq m}^{1\leq i\leq m}\neq0.
        \]
        Denotem doncs
        \[
            F(x)=(f_{\sigma(1)}(x),\dots,f_{\sigma(m)}(x))
        \]
        i tenim que
        \begin{equation}
            \label{thm:Teorema-destructura-local-de-les-submersions:eq:1}
            \det(dF(x_{0}))\neq0.
        \end{equation}

        Considerem la funció
        \begin{align*}
            g\colon\mathbb{R}^{n}&\longrightarrow\mathbb{R}^{n} \\
            x&\longmapsto(F(x),0)+(0,\pi_{2}(x)).
        \end{align*}
        Per l'exemple \myref{ex:la-projeccio-canonica-es-una-submersio} trobem que
        \[
            (\pi_{1}\circ g)(x)=\pi_{1}(F(x),\pi_{2}(x))=F(x).
        \]

        Per la definició de \myref{def:Jacobiana} i la proposició \myref{prop:justificacio-Jacobiana} tenim que
        \[dg(x_{0})=\left[\begin{array}{c|c}
            \begin{matrix}
                \frac{\partial F_{1}}{\partial x_{1}} & \cdots & \frac{\partial F_{1}}{\partial x_{n}} \\
                \vdots & & \vdots \\
                \frac{\partial F_{m}}{\partial x_{n+1}} & \cdots & \frac{\partial F_{m}}{\partial x_{n}}
            \end{matrix} & \begin{matrix}
                \frac{\partial F_{1}}{\partial x_{1}} & \cdots & \frac{\partial F_{1}}{\partial x_{n}} \\
                \vdots & & \vdots \\
                \frac{\partial F_{m}}{\partial x_{n+1}} & \cdots & \frac{\partial F_{m}}{\partial x_{m}}
            \end{matrix} \\ \hline \\
            0_{(n-m)\times n} & I_{(n-m)\times(n-m)} \\
            &
        \end{array}\right].\]
        Aleshores tenim que
        \[
            \det(dg(x_{0}))=\det(dF(x_{0})),
        \]
        i per \eqref{thm:Teorema-destructura-local-de-les-submersions:eq:1} trobem que
        \[
            \det(dg(x_{0}))\neq0,
        \]
        i pel \myref{thm:Funcio-inversa} hem acabat.
    \end{proof}
    \subsection{Superfícies}
    \begin{definition}[Superfície]
        \labelname{superfície}\label{def:superficie}
        Sigui~\(S\subseteq\mathbb{R}^{3}\) un subconjunt tal que per a tot~\(p\in S\) existeix un entorn obert~\(\obert{U}\subseteq\mathbb{R}^{3}\) de~\(p\) i un difeomorfisme~\(g\colon\obert{U}\longrightarrow g(\obert{U})\) tal que
        \[
            g(\obert{U}\cap S)=g(\obert{U})\cap(\mathbb{R}^{2}\times\{0\}).
        \]
        Aleshores direm que~\(S\) és una superfície regular.
    \end{definition}
    \begin{theorem}
        \label{thm:condicions-equivalents-a-la-definicio-de-superficie}
        Sigui~\(S\subseteq\mathbb{R}^{3}\) un subconjunt.
        Aleshores~\(S\) és una superfície si i només si per a tot punt~\(p\in S\) existeixen un entorn obert~\(\obert{U}\subseteq\mathbb{R}^{3}\) de~\(p\) i una submersió~\(F\colon\obert{U}\longrightarrow\mathbb{R}\) tals que
        \[
            S\cap\obert{U}=F^{-1}(\{0\}).
        \]
    \end{theorem}
    \begin{proof}
        Comencem veient que la condició és suficient (\(\implica\)).
        Suposem doncs que~\(S\) és una superfície.
        Per la definició de \myref{def:superficie} trobem que que per a tot~\(p\in S\) existeix un entorn obert~\(\obert{U}\subseteq\mathbb{R}^{3}\) de~\(p\) i un difeomorfisme~\(g\colon\obert{U}\longrightarrow g(\obert{U})\) tal que
        \begin{equation}
            \label{eq:thm:condicions-equivalents-a-la-definicio-de-superficie-1}
            g(\obert{U}\cap S)=g(\obert{U})\cap(\mathbb{R}^{2}\times\{0\}).
        \end{equation}
        Considerem la funció
        \begin{align*}
            F\colon\obert{U}&\longrightarrow\mathbb{R} \\
            x&\longmapsto\pi_{3}(g(x)).
        \end{align*}
        Aleshores tenim que
        \begin{align*}
            F^{-1}(\{0\})&=g^{-1}(\pi_{3}^{-1}(\{0\})) \\
            &=g^{-1}(\mathbb{R}^{2}\times\{0\}) \\
            &=g^{-1}\big(g(\obert{U})\big)\cap g^{-1}(\mathbb{R}^{2}\times\{0\})\\
            &=g^{-1}\big(g(\obert{U})\cap(\mathbb{R}^{2}\times\{0\})\big)=\obert{U}\cap S, \tag{\ref{eq:thm:condicions-equivalents-a-la-definicio-de-superficie-1}} %TODO % Y THO
        \end{align*}
        i hem acabat.

        Veiem ara que la condició és necessària (\(\implicatper\)).
        Suposem doncs que per a tot punt~\(p\in S\) existeixen un entorn obert~\(\obert{U}\subseteq\mathbb{R}^{3}\) de~\(p\) i una submersió~\(F\colon\obert{U}\longrightarrow\mathbb{R}\) tals que~\(S\cap\obert{U}=F^{-1}(\{0\})\).
        Pel \myref{thm:Teorema-destructura-local-de-les-submersions} tenim que per a tot~\(p\in S\) existeix un obert~\(\obert{U}'\) satisfent~\(x_{0}\subseteq\obert{U}'\subseteq\obert{U}\) i existeix un difeomorfisme~\(g\colon\obert{U}'\longleftrightarrow g(\obert{U}')\subseteq\mathbb{R}^{3}\) tals que per a tot~\(x\in\obert{U}'\) tenim
        \[
            F(x)=(\pi_{3}\circ g)(x).
        \]
        Per tant, si considerem l'aplicació
        \begin{align*}
            G\colon\obert{U}'&\longrightarrow\mathbb{R} \\
            x&\longmapsto F(x)
        \end{align*}
        la restricció de~\(F\) en~\(\obert{U}'\) trobem que per a tot~\(x\in\obert{U}'\) tenim
        \[
            G(x)=(\pi_{3}\circ g)(x),
        \]
        i es satisfà
        \begin{equation}
            \label{eq:thm:condicions-equivalents-a-la-definicio-de-superficie-2}
            G^{-1}(\{0\})=S\cap\obert{U}'.
        \end{equation}

        Aleshores tenim que
        \begin{align*}
            S\cap\obert{U}'&=G^{-1}(\{0\}) \tag{\ref{eq:thm:condicions-equivalents-a-la-definicio-de-superficie-2}}\\
            &=g^{-1}\big(\pi_{3}^{-1}(\{0\})\big) \\
            &=g^{-1}(\mathbb{R}^{2}\times\{0\}),
        \end{align*}
        i per tant tenim
        \begin{align*}
            g(S\cap\obert{U}')&=g(\obert{U}')\cap g(S\cap\obert{U}') \\
            &=g(\obert{U}')\cap g(g^{-1}(\mathbb{R}^{2}\times\{0\})) \\
            &=g(\obert{U}')\cap(\mathbb{R}^{2}\times\{0\}),
        \end{align*}
        i per la definició de \myref{def:superficie} hem acabat.
%            \[g(S\cap\obert{U}')=g(g^{-1}(\mathbb{R}^{2}\times\{0\}))=g(\obert{U}')\mathbb{R}^{2}\times\{0\}\]
%            continuar
        %TODO continuar. Revisar que no l'hagi cagat
    \end{proof}
    \begin{example}
        \label{ex:un-tor-es-una-superficie}
        Volem veure que un tor és una superfície
    \end{example}
    \begin{solution}
        Un tor de radi exterior~\(R\) i radi interior~\(r\) ve parametritzat per
        \[\begin{cases*}
            x=(r\cos(\theta)+R)\cos(\varphi) \\
            y=(r\cos(\theta)+R)\sin(\varphi) \\
            z=r\sin(\theta),
        \end{cases*}\]
        amb~\(\theta\in(0,2\pi)\) i~\(\varphi\in(0,2\pi)\).
        %TODO
%            Si definim
%            \[F(\theta,\varphi)=((r\cos(\theta)+R)\cos(\varphi),(r\cos(\theta)+R)\sin(\varphi),r\in(\theta))\]
%            trobem que
%            \[DJ=\begin{bmatrix}
%                -r\sin(\theta)\cos(\varphi) & -(r\cos(\theta)+R)\sin(\varphi) \\
%                -r\sin(\theta)\sin(\varphi) & (r\cos(\theta)+R)\cos(\varphi) \\
%                r\cos(\theta) & 0
%            \end{bmatrix},\]
    \end{solution}
    \begin{definition}[Carta local]
        \labelname{carta local}\label{def:carta-local}
        \labelname{parametrització local d'una superfície}\label{def:parametritzacio-local-duna-superficie}
        \labelname{parametrització d'una superfície}\label{def:parametritzacio-duna-superficie}
        Sigui~\(S\subseteq\mathbb{R}^{3}\) un conjunt i~\(p\in S\) un punt tals que existeix un entorn obert~\(\obert{U}\) de~\(p\), un conjunt~\(\Omega\subseteq\mathbb{R}^{2}\) i una aplicació~\(\Psi\colon\Omega\longrightarrow S\cap\obert{U}\) que és una immersió i un homeomorfisme.
        Aleshores direm que~\(\Omega\) amb~\(\Psi\) és una carta local de~\(p\) en~\(S\).

        També direm que~\(\Psi\) és una parametrització o parametrització local de~\(S\).
    \end{definition}
    \begin{theorem}
        \label{thm:ser-superficie-es-equivalent-a-tenir-cartes-locals}
        Sigui~\(S\subseteq\mathbb{R}^{3}\) un subconjunt.
        Aleshores~\(S\) és una superfície si i només si per a tot~\(p\in S\) existeix una carta local de~\(p\) en~\(S\).
    \end{theorem}
    \begin{proof}
        %TODO
    \end{proof}
    \begin{example}
        Volem veure que si tenim~\(S\subseteq\mathbb{R}^{3}\) una superfície i~\(\Psi\colon\Omega\longrightarrow\Psi(\Omega)\subseteq S\) una immersió injectiva, aleshores~\(\Omega\) amb~\(\Psi\) és una carta local si i només si~\(\Psi\colon\Omega\longrightarrow\Psi(\Omega)\) és un homeomorfisme.
        % Potser és només en una direcció (cap a la dreta).
    \end{example}
    \begin{solution}
        %TODO
    \end{solution}
    \subsection{Funcions diferenciables i l'espai tangent}
    % Veure com ho fa el Reventós (pàg. 96, en particular llegir les primeres línies de la 97). M'agrada més.
    \begin{definition}[Funció diferenciable]
        \labelname{funció diferenciable}\label{def:funcio-diferenciable}
        Siguin~\(S\subseteq\mathbb{R}^{3}\) una superfície i~\(f\colon S\longrightarrow\mathbb{R}^{k}\) una funció tal que per a tot~\(p\in S\) existeix una carta local~\(\Omega\) amb~\(\Psi\) satisfent~\(p\in\Psi(\Omega)\) tals que
        \[
            f\circ\Psi\colon\Omega\longrightarrow\mathbb{R}^{k}
        \]
        és un difeomorfisme.
        Aleshores direm que~\(f\) és diferenciable.
    \end{definition}
    \begin{theorem}
        \label{thm:la-composicio-de-parametritzacions-locals-es-una-funcio-diferenciable}
        Siguin~\(S\subseteq\mathbb{R}^{3}\) una superfície i~\(\Omega_{1}\) amb~\(\Psi_{1}\) i~\(\Omega_{2}\) amb~\(\Psi_{2}\) dues parametritzacions locals de~\(S\) amb
        \[
            \Psi_{1}(\Omega_{1})\cap\Psi_{2}(\Omega_{2})\neq\emptyset.
        \]
        Aleshores la funció
        \[
            \Psi_{2}^{-1}\circ\Psi_{1}\colon\Omega_{1}\cap\Omega_{2}\longrightarrow\Omega_{1}\cap\Omega_{2}
        \]
        és diferenciable.
    \end{theorem}
    \begin{proof}
%            Prenem un element~\(z\in\Psi_{1}(\Omega_{1})\cap\Psi_{2}(\Omega_{2})\). Aleshores tenim que existeixen dos punts~\(x_{1}\in\Omega_{1}\) i~\(x_{2}\in\Omega_{2}\) tals que~\(\Psi_{1}(x_{1})=z\) i~\(\Psi_{2}(x_{2})=z\).
%
%            Com que, per hipòtesi,~\(S\) és una superfície, per la definició de \myref{def:superficie} trobem que existeixen un entorn obert~\(\obert{U}\) i un difeomorfisme~\(g\colon\obert{U}\longrightarrow g(\obert{U})\) tals que
%            \[g(\obert{U}\cap S)=g(\obert{U})\cap(\mathbb{R}^{2}\times\{0\}).\]
%            %Ara això ha de ser un difeomorfisme d'alguna manera
%            \[\pi_{1}\circ g\circ\Psi_{1}\]
    \end{proof}
    \begin{proposition}
        \label{prop:per-a-tot-punt-i-funcio-diferenciable-duna-superficie-podem-trobar-una-segona-funcio-diferenciable-igual-en-un-entorn-del-punt}
        Siguin~\(S\subseteq\mathbb{R}^{3}\) una superfície,~\(f\colon S\longrightarrow\mathbb{R}^{k}\) una funció diferenciable i~\(p\in S\) un punt.
        Aleshores  existeix un entorn obert~\(\obert{U}\subseteq\mathbb{R}^{3}\) de~\(p\) i una funció diferenciable~\(\tilde{f}\colon\obert{U}\longrightarrow\mathbb{R}^{k}\) tals que per a tot~\(x\in S\cap\obert{U}\) tenim
        \[
            f(x)=\tilde{f}(x).
        \]
    \end{proposition}
    \begin{proof}
        %TODO
    \end{proof}
    \begin{definition}[Funció diferenciable entre superfícies]
        \labelname{funció diferenciable entre superfícies}\label{def:funcio-diferenciable-entre-superficies}
        Siguin~\(S_{1}\subseteq\mathbb{R}^{3}\) i~\(S_{2}\subseteq\mathbb{R}^{3}\) dues superfícies i~\(f\colon S_{1}\longrightarrow\mathbb{R}^{3}\) una funció diferenciable.
        Aleshores diem que la funció~\(f\colon S_{1}\longleftarrow S_{2}\) és diferenciable.
    \end{definition}
    \begin{proposition}
        Siguin~\(S_{1}\subseteq\mathbb{R}^{3}\) i~\(S_{2}\subseteq\mathbb{R}^{3}\) dues superfícies i~\(f\colon S_{1}\longrightarrow S_{2}\) una funció.
        Aleshores~\(f\) és diferenciable si i només si existeixen dues cartes locals~\(\Omega_{1}\) amb~\(\Psi_{1}\) i~\(\Omega_{2}\) amb~\(\Psi_{2}\) de~\(S_{1}\) i~\(S_{2}\), respectivament, satisfent que existeix un~\(p\in\Psi_{1}(\Omega_{1})\) i~\(f(p)\in\Psi_{2}(\Omega_{2})\) i tals que l'aplicació
        \[
            \Psi_{2}^{-1}\circ f\circ \Psi_{1}\colon\Omega_{1}\cap\Omega_{2}\longrightarrow\Omega_{1}\cap\Omega_{2}
        \]
        és un difeomorfisme.
    \end{proposition}
    \begin{proof}
        %TODO
    \end{proof}
    \section{Primera forma fonamental}
    \subsection{Espai tangent a una superfície}
    \begin{definition}[Vectors tangents]
        \labelname{vectors tangents}\label{def:vectors-tangents-a-una-superficie}
        \labelname{espai tangent}\label{def:espai-tangent-a-una-superficie}
        Siguin~\(S\subseteq\mathbb{R}^{3}\) una superfície i~\(p\in S\) un punt.
        Aleshores direm que el conjunt
        \[
            \EspaiTangent_{p}(S)=\{\alpha'(0)\in\mathbb{R}^{3}\mid\alpha\text{ és una corba regular en }S\text{ amb }\alpha(0)=p\}
        \]
        és l'espai tangent a~\(S\) en~\(p\), i els elements de~\(\EspaiTangent_{p}(S)\) són els vectors tangents a~\(S\) en~\(p\).
    \end{definition}
    % Veure que T_{p}(S) és un espai vectorial (afí?)
    \begin{proposition}
        \label{prop:les-derivades-duna-parametritzacio-en-una-carta-local-son-base-del-pla-tangent}
        Siguin~\(S\subseteq\mathbb{R}^{3}\) una superfície,~\(p\in S\) un punt i~\(\varphi(u,v)\) amb~\(\Omega\) una carta local de~\(S\) en~\(p\).
        Aleshores
        \[
            (\varphi_{u}(p),\varphi_{v}(p))
        \]
        és una base de~\(\EspaiTangent_{p}(S)\).
    \end{proposition}
    \begin{proof}
        Per la definició de \myref{def:carta-local} tenim que~\(\varphi\) és una immersió, i per l'observació \myref{obs:immersio-si-i-nomes-si-te-rang-maximal} tenim que~\(\rang(d\varphi(p))=2\), i per la definició de \myref{def:Jacobiana} tenim que
        \[
            d\varphi(p)=(\varphi_{u}(p),\varphi_{v}(p)).
        \]
        Per la definició d'\myref{def:espai-tangent-a-una-superficie} tenim que~\(\varphi_{u}(p)\) i~\(\varphi_{v}(p)\) pertanyen a~\(\EspaiTangent_{p}(S)\), i per tant tenim que~\((\varphi_{u}(p),\varphi_{v}(p))\) és una base de~\(\EspaiTangent_{p}(S)\).
        %REF
    \end{proof}
    \begin{definition}[Primera forma fonamental]
        Siguin~\(S\subseteq\mathbb{R}^{3}\) una superfície i~\(p\in S\) un punt.
        Aleshores direm que l'aplicació
        \begin{align*}
            \I_{p}\colon\EspaiTangent_{p}(S)\times\EspaiTangent_{p}(S)&\longrightarrow\mathbb{R} \\
            (\vec{u},\vec{v})&\longmapsto\prodesc{\vec{u}}{\vec{v}}
        \end{align*}
        és la primera forma fonamental de~\(S\) en~\(p\).
    \end{definition}
    \begin{proposition}
        \label{prop:la-primera-forma-fonamental-es-una-forma-bilineal}
        Siguin~\(S\subseteq\mathbb{R}^{3}\) una superfície i~\(p\in S\) un punt.
        Aleshores la primera forma fonamental~\(\I_{p}\) és una forma bilineal.
    \end{proposition}
    \begin{proof}
        Siguin~\(\vec{u}\),~\(\vec{v}\) i~\(\vec{w}\) tres vectors de~\(\EspaiTangent_{p}(S)\).
        Aleshores
        \begin{align*}
            \I_{p}(\vec{u}+\vec{v},\vec{w})&=\prodesc{\vec{u}+\vec{v}}{\vec{w}} \\
            &=\prodesc{\vec{u}}{\vec{w}}+\prodesc{\vec{v}}{\vec{w}} \tag{\ref{def:producte-escalar}}\\
            &=\I_{p}(\vec{u},\vec{w})+\I_{p}(\vec{v},\vec{w}),
        \end{align*}
        i
        \begin{align*}
            \I_{p}(\vec{u},\vec{v}+\vec{w})&=\prodesc{\vec{u}}{\vec{v}+\vec{w}} \\
            &=\prodesc{\vec{u}}{\vec{w}}+\prodesc{\vec{v}}{\vec{w}} \tag{\ref{def:producte-escalar}}\\
            &=\I_{p}(\vec{u},\vec{w})+\I_{p}(\vec{v},\vec{w}).
        \end{align*}

        Si prenem un escalar~\(\lambda\in\mathbb{R}\) trobem que
        \begin{align*}
            \I_{p}(\lambda\vec{u},\vec{v})&=\prodesc{\lambda\vec{u}}{\vec{v}} \\
            &=\lambda\prodesc{\vec{u}}{\vec{v}} \tag{\ref{def:producte-escalar}}\\
            &=\lambda\I_{p}(\vec{u},\vec{v}).
        \end{align*}
        i
        \begin{align*}
            \I_{p}(\vec{u},\lambda\vec{v})&=\prodesc{\vec{u}}{\lambda\vec{v}} \\
            &=\lambda\prodesc{\vec{u}}{\vec{v}} \tag{\ref{def:producte-escalar}}\\
            &=\lambda\I_{p}(\vec{u},\vec{v}).
        \end{align*}
        i per la definició de \myref{def:forma-bilineal} hem acabat.
    \end{proof}
    \begin{proposition}
        \label{prop:coeficients-de-la-primera-forma-fonamental}
        Siguin~\(S\subseteq\mathbb{R}^{3}\) una superfície parametritzada per~\(\varphi(u,v)\),~\(p\in S\) un punt i~\(\vec{u}\) i~\(\vec{v}\) dos vectors de~\(\EspaiTangent_{p}(S)\).
        Aleshores
        \[\I_{p}(\vec{u},\vec{v})=\vec{u}^{\transposta}\begin{bmatrix}
            \prodesc{\varphi_{u}(p)}{\varphi_{u}(p)} & \prodesc{\varphi_{u}(p)}{\varphi_{v}(p)} \\
            \prodesc{\varphi_{v}(p)}{\varphi_{u}(p)} & \prodesc{\varphi_{v}(p)}{\varphi_{v}(p)}
        \end{bmatrix}\vec{v}.\]
    \end{proposition}
    \begin{proof}
        Aquest enunciat té sentit per la proposició \myref{prop:la-primera-forma-fonamental-es-una-forma-bilineal}.
        Tenim que existeixen quatre reals~\(A\),~\(B\),~\(C\),~\(D\in\mathbb{R}\) tals que
        \[\I_{p}(\vec{u},\vec{v})=\vec{u}^{\transposta}\begin{bmatrix}
            A & B \\
            C & D
        \end{bmatrix}\vec{v}.\]

        Per la proposició \myref{prop:les-derivades-duna-parametritzacio-en-una-carta-local-son-base-del-pla-tangent} tenim que~\((\varphi_{u},\varphi_{v})\) és una base de~\(\EspaiTangent_{p}(S)\), i per tant
        \[
            \vec{u}=u_{1}\varphi_{u}(p)+u_{2}\varphi_{v}(p),\qquad\text{i}\qquad\vec{v}=v_{1}\varphi_{u}(p)+v_{2}\varphi_{v}(p),
        \]
        i per tant en la base~\((\varphi_{u},\varphi_{v})\) tenim
        \[
            \vec{u}=(u_{1},u_{2})\qquad\text{i}\qquad\vec{v}=(v_{1},v_{2}).
        \]
        Aleshores obtenim que
        \begin{align*}
            \I_{p}(\vec{u},\vec{v})&=\prodesc{\vec{u}(p)}{\vec{v}(p)} \\
            &=u_{1}v_{1}\prodesc{\varphi_{u}(p)}{\varphi_{u}(p)}+u_{1}v_{2}\prodesc{\varphi_{u}(p)}{\varphi_{v}(p)}+\\
            &\phantom{=}+u_{2}v_{1}\prodesc{\varphi_{u}(p)}{\varphi_{v}(p)}+u_{2}v_{2}\prodesc{\varphi_{v}(p)}{\varphi_{v}(p)},
        \end{align*}
        i tenim que
        \[\vec{u}^{\transposta}\begin{bmatrix}
            A & B \\
            C & D
        \end{bmatrix}\vec{v}=u_{1}v_{1}A+u_{1}v_{2}B+u_{2}v_{1}C+u_{2}v_{2}D,\]
        i per tant trobem que
        \[
            A=\prodesc{\varphi_{u}(p)}{\varphi_{u}(p)},\qquad B=C=\prodesc{\varphi_{u}(p)}{\varphi_{v}(p)}\quad\text{i}\quad D=\prodesc{\varphi_{v}(p)}{\varphi_{v}(p)}.\qedhere
        \]
    \end{proof}
    \begin{notation}[Primera forma fonamental]
        \label{notation:primera-forma-fonamental}
        Denotarem
        \[\I_{\varphi(u,v)}=\begin{bmatrix}
            E(u,v) & F(u,v) \\
            F(u,v) & G(u,v)
        \end{bmatrix}=
        \begin{bmatrix}
            \prodesc{\varphi_{u}}{\varphi_{u}} & \prodesc{\varphi_{u}}{\varphi_{v}} \\
            \prodesc{\varphi_{v}}{\varphi_{u}} & \prodesc{\varphi_{v}}{\varphi_{v}}
        \end{bmatrix}.\]
    \end{notation}
    \begin{example}
        \label{ex:primera-forma-fonamental-duna-esfera-de-radi-1}
        Volem calcular la primera forma fonamental d'una esfera parametritzada per
        \[
            \Psi(\theta,\varphi)=(\sin(\theta)\cos(\varphi),\sin(\theta)\sin(\varphi),\cos(\theta)).
        \]
    \end{example}
    \begin{solution}
        Calculem
        \[
            \Psi_{\theta}(\theta,\varphi)=(\cos(\theta)\cos(\varphi),\cos(\theta)\sin(\varphi),-\sin(\theta))
        \]
        i
        \[
            \Psi_{\varphi}(\theta,\varphi)=(-\sin(\theta)\sin(\varphi),\sin(\theta)\cos(\varphi),0).
        \]

        Aleshores tenim
        \[
            E=\prodesc{\Psi_{\theta}}{\Psi_{\theta}}=1,\quad F=\prodesc{\Psi_{\theta}}{\Psi_{\varphi}}=0\quad\text{i}\quad G=\prodesc{\Psi_{\varphi}}{\Psi_{\varphi}}=\sin^{2}(\theta),
        \]
        i per tant
        \[\I=\begin{bmatrix}
            1 & 0 \\
            0 & \sin^{2}(\theta)
        \end{bmatrix}.\qedhere\]
    \end{solution}
    \subsection{Longitud d'una corba}
    \begin{proposition}
        \label{prop:longitud-duna-corba-en-una-superficie}
        Siguin~\(S\subseteq\mathbb{R}^{3}\) una superfície,~\(\Psi(u,v)\) amb~\(\Omega\) una carta local de~\(S\) i~\(\alpha\) una corba en~\(\Psi(\Omega)\).
        Aleshores
        \[
            \funciolongituddarc_{\alpha}(t_{0})(t)=\int_{t_{0}}^{t}\sqrt{Eu'(\tau)^{2}+2Fu'(\tau)v'(\tau)+Gv'(\tau)^{2}}\diff \tau.
        \]
    \end{proposition}
    \begin{proof}
        Com que~\(\alpha\) està continguda en~\(\Psi(\Omega)\) tenim que existeixen dues funcions~\(u(t)\),~\(v(t)\) tals que
        \[
            \alpha(t)=\Psi(u(t),v(t)).
        \]

        Calculem
        \[
            \alpha'(t)=u'(t)\varphi_{u}(u,v)+v'(t)\varphi_{v}(u,v),
        \]
        i tenim que
        \begin{align*}
            \norm{\alpha'(t)}&=\sqrt{\prodesc{\alpha'(t)}{\alpha'(t)}} \\
            &=\sqrt{\alpha'(t)^{\transposta}\I_{p}\alpha'(t)} \\
            &=\sqrt{\begin{bmatrix}
                u'(t) & v'(t)
            \end{bmatrix}
            \begin{bmatrix}
                E & F \\
                F & G
            \end{bmatrix}
            \begin{bmatrix}
                u'(t) \\
                v'(t)
            \end{bmatrix}}
        \end{align*}
    \end{proof}

%    \begin{proposition}
%        Siguin~\(S\subseteq\mathbb{R}^{3}\) una superfície,~\(p\in S\) un punt i~\(\Psi\) amb~\(\Omega\) una carta local de~\(S\) en~\(p\). Aleshores
%    \end{proposition}

\begin{comment}
    Calcular curvatura de Gauss i curvatura mitjana d'un helicoide.
    (6.7)
    \[
        \varphi(u,v)=(u\cos(v),u\sin(v),av).
    \]
    Línies de curvatura (7.3)
    Fer-ho tot per helicoides lol
\end{comment}
% TODO Desfer-se dels _{p} en la primera forma fonamental...
\end{document}
