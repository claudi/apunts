\documentclass[../Apunts.tex]{subfiles}

\begin{document}
\chapter{Les topologies}
	\section{Espais mètrics}
	\subsection{Distància en conjunts arbitraris}
	\begin{definition}[Espai mètric]
		\labelname{espai mètric}\label{def:espai mètric}
		\labelname{distància}\label{def:distància}
		Sigui \(X\) un conjunt i \(d\colon X\times X\longrightarrow\mathbb{R}\) una aplicació que per a tot \(x\), \(y\) i \(z\) de \(X\) satisfà
		\begin{enumerate}
			\item \(d(x,y)=0\) si i només si \(x=y\).
			\item \(d(x,y)=d(y,x)\).
			\item \(d(x,y)\leq d(x,z)+d(z,y)\).
			\item \(d(x,y)\leq0\).
		\end{enumerate}
		Aleshores direm que \((X,d)\) és un espai mètric. També direm que \(d\) és la distància o mètrica de l'espai mètric.
	\end{definition}
	\begin{definition}[Bola]
		\labelname{bola}\label{def:bola}
		Siguin \((X,d)\) un espai mètric, \(a\) un element de \(X\) i \(r>0\) un nombre real. Aleshores definim
		\[\B(a,r)=\{x\in X\mid d(x,a)<r\}\]
		com la bola de radi \(r\) centrada en \(a\).
	\end{definition}
	\begin{definition}[Obert]
		\labelname{obert}\label{def:obert}
		Sigui \((X,d)\) un espai mètric tal que per a tot element \(a\) de \(X\) existeix un \(\varepsilon>0\) real tal que \(\B(a,\varepsilon)\subset A\). Aleshores direm que \(X\) és un obert.
	\end{definition}
	\begin{proposition}
		Siguin \((X,d)\) un espai mètric i \(\B(a,r)\) una bola de \(X\). Aleshores \(\B(a,r)\) és un obert.
		\begin{proof}
			Prenem un element \(b\in\B(a,r)\) i definim
			\[\varepsilon=\frac{r-d(a,b)}{2}.\]
			Aleshores considerem la bola \(\B(b,\varepsilon)\) i tenim que \(\B(b,\varepsilon)\subset\B(a,r)\)
		\end{proof}
	\end{proposition}
	\subsection{L'espai topològic}
	\subsection{Entorns, interior i adherència}
	\subsection{Aplicacions contínues}
	\subsection{Subespais}
	\subsection{La topologia producte}
	\subsection{La topologia quocient}
	\subsection{Espais compactes}
	\subsection{Espais de Hausdorff}
	\subsection{Connexió}
	\subsection{Varietats}
	\subsection{Teorema de classificació de les superfícies compactes}
\end{document}