\documentclass[../Apunts.tex]{subfiles}

\begin{document}
\chapter{Equacions diferencials de primer ordre en una variable}
	\section{Teoremes d'existència i unicitat}
	\subsection{Equacions diferencials ordinàries}
	\begin{definition}[Equació diferencial ordinària d'ordre \(n\)]
		\labelname{equació diferencial ordinària d'odre $n$}\label{def:equació diferencial ordinària d'ordre n}
		\labelname{solució de l'equació diferencial ordinària}\label{def:solució de l'equació diferencial ordinària}
		Siguin \(\Omega=\mathbb{R}\times\mathbb{R}^{n}\) un conjunt i \(F\colon\Omega\longrightarrow\mathbb{R}^{m}\), \(u\colon\mathbb{R}\longrightarrow\mathbb{R}^{m}\) dues funcions tals que
		\[\frac{\partial^{n}u(x)}{\partial x^{n}}=F\left(t,u(x),\frac{\partial u(x)}{\partial x},\dots,\frac{\partial^{n-1}u(x)}{\partial x^{n-1}}\right).\]
		Aleshores direm que
		és una equació diferencial ordinària. També direm que \(t\) és la variable temporal, \(u(x)\) la variable espacial.
	\end{definition}
	\begin{example}
		\label{ex:edos1:1}
		Considerem l'equació diferencial ordinària d'ordre \(1\) següent:
		\begin{equation}
			\label{edos1:ex1:eq1}
			\frac{\partial u(x)}{\partial x}=f(x).
		\end{equation}
		on \(I\) és un interval de \(\mathbb{R}\) i \(f\colon I\longrightarrow\mathbb{R}^{m}\) una funció contínua.
		\begin{solution}
			Observem primer que pel Teorema \myref{thm:Weierstrass màxims i mínims múltiples variables} la funció \(f(x)\) està acotada en \(I\), i pel Teorema \myref{thm:Contínua + acotada implica integrable Riemann} tenim que la funció \(f(x)\) és integrable Riemann, i per tant, pel \myref{thm:Teorema Fonamental del Càlcul}, trobem que per a tot \(x_{0}\) de \(I\)
			\[u(x)=\int_{x_{0}}^{x}f(t)dt\]
			és una solució de l'equació diferencial ordinària \eqref{edos1:ex1:eq1}.
		\end{solution}
	\end{example}
	\begin{observation}
		Observem que, en l'exemple \myref{ex:edos1:1}, si la funció \(f\) no és integrable en \(I\) aleshores l'equació diferencial \eqref{edos1:ex1:eq1} no té solució. %Integrable Riemann?
	\end{observation}
	\begin{example}[Problema de Cauchy]
		\label{ex:Problema de Cauchy}
		Siguin \((a,b)\) un interval de \(\mathbb{R}\), \(f\colon(a,b)\longrightarrow\mathbb{R}\) una funció contínua tal que \(f(x)\neq0\) per a tot \(x\) de \((a,b)\), \(x_{0}\) un element de \((a,b)\) i \(t_{0}\) un real. Considerem el sistema
		\begin{equation}
			\label{ex:Problema de Cauchy:eq1}
			\begin{cases}
				\displaystyle \frac{\partial u(x)}{\partial x}=f(x) \\
				\displaystyle u(t_{0})=x_{0}
			\end{cases}.
		\end{equation}
		Volem trobar una solució a aquest sistema.
		\begin{solution}
			Observem que, per la definició de \myref{def:equació diferencial ordinària d'ordre n} l'equació
			\[\frac{\partial u(x)}{\partial x}=f(x)\]
			és una equació ordinària d'orde \(1\).
			
			Com que, per hipòtesi, la funció \(f(x)\) és contínua en \([a,b]\) trobem pel Teorema \myref{thm:Weierstrass màxims i mínims múltiples variables} que \(f(x)\) està acotada en \([a,b]\), i pel Teorema \myref{thm:Contínua + acotada implica integrable Riemann} que la \(f(x)\) és integrable en \([a,b]\) 
			
			\eqref{ex:Problema de Cauchy:eq1}.
		\end{solution}
	\end{example}
\end{document}