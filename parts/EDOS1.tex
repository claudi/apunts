\documentclass[../Apunts.tex]{subfiles}

\begin{document}
\chapter{Equacions diferencials de primer ordre en una variable}
	\section{Teoremes d'existència i unicitat}
	\subsection{Equacions diferencials ordinàries}
	\begin{definition}[Equació diferencial ordinària d'ordre \(n\)]
		\labelname{equació diferencial ordinària d'odre $n$}\label{def:equació diferencial ordinària d'ordre n}
		\labelname{solució de l'equació diferencial ordinària}\label{def:solució de l'equació diferencial ordinària}
		Siguin \(\Omega=\mathbb{R}\times\mathbb{R}^{n}\) un conjunt i \(F\colon\Omega\longrightarrow\mathbb{R}^{m}\), \(u\colon\mathbb{R}\longrightarrow\mathbb{R}^{m}\) dues funcions tals que
		\[\frac{\partial^{n}u(x)}{\partial x^{n}}=F\left(t,u(x),\frac{\partial u(x)}{\partial x},\dots,\frac{\partial^{n-1}u(x)}{\partial x^{n-1}}\right).\]
		Aleshores direm que
		és una equació diferencial ordinària. També direm que \(t\) és la variable temporal, \(u(x)\) la variable espacial.
	\end{definition}
	\begin{example}
		Considerem l'equació diferencial ordinària d'ordre \(1\) següent:
		\[\frac{\partial u(x)}{\partial x}=f(t).\]
		on \(I\) és un interval de \(\mathbb{R}\) i \(f\colon I\longrightarrow\mathbb{R}^{m}\).
	\end{example}
\end{document}