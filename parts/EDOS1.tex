\documentclass[../Apunts.tex]{subfiles}

\begin{document}
\chapter{Equacions diferencials de primer ordre en una variable}
	\section{Teoremes d'existència i unicitat}
	\subsection{Equacions diferencials ordinàries}
	\begin{definition}[Equació diferencial ordinària d'ordre \(n\)]
		\labelname{equació diferencial ordinària d'odre $n$}\label{def:equació diferencial ordinària d'ordre n}
		\labelname{solució de l'equació diferencial ordinària}\label{def:solució de l'equació diferencial ordinària}
		Siguin \(\Omega=\mathbb{R}\times\mathbb{R}^{n}\) un conjunt i \(F\colon\Omega\longrightarrow\mathbb{R}^{m}\), \(u\colon\mathbb{R}\longrightarrow\mathbb{R}^{m}\) dues funcions tals que
		\[\frac{\partial^{n}u(x)}{\partial x^{n}}=F\left(t,u(x),\frac{\partial u(x)}{\partial x},\dots,\frac{\partial^{n-1}u(x)}{\partial x^{n-1}}\right).\]
		Aleshores direm que
		és una equació diferencial ordinària. També direm que \(t\) és la variable temporal, \(u(x)\) la variable espacial.
	\end{definition}
	\begin{proposition}
		\label{prop:edos1:1}
		Siguin \(I\) un interval de \(\mathbb{R}\) i \(f\colon I\longrightarrow\mathbb{R}^{m}\) una funció contínua. Aleshores l'equació diferencial ordinària d'ordre \(1\)
		\begin{equation}
			\label{edos1:prop1:eq1}
			\frac{\partial u(x)}{\partial x}=f(u(x)).
		\end{equation}
		té una única solució de la forma
		\[u(x)=\int_{x_{0}}^{x}f(u(t))\text{d}t.\]
		\begin{proof}
			Observem primer que pel \myref{thm:Weierstrass màxims i mínims múltiples variables} la funció \(f(x)\) està acotada en \(I\), i pel Teorema \myref{thm:Contínua + acotada implica integrable Riemann} tenim que la funció \(f(x)\) és integrable Riemann, i per tant, pel \myref{thm:Teorema Fonamental del Càlcul}, trobem que per a tot \(x_{0}\) de \(I\)
			\[u(x)=\int_{x_{0}}^{x}f(u(t))\text{d}t\]
			és una solució de l'equació diferencial ordinària \eqref{edos1:prop1:eq1}.
			
			Aquesta demostració també ens serveix per veure la unicitat.
		\end{proof}
	\end{proposition}
	\begin{observation}
		Observem que, en la proposició \myref{prop:edos1:1}, si la funció \(f\) no és integrable en \(I\) aleshores l'equació diferencial \eqref{edos1:prop1:eq1} no té solució. %Integrable Riemann?
	\end{observation}
	\begin{proposition}[Problema de Cauchy]
		\labelname{}\label{prop:problema de Cauchy}
		Siguin \((a,b)\) un interval obert de \(\mathbb{R}\), \(f\colon(a,b)\longrightarrow\mathbb{R}\) una funció contínua tal que \(f(x)\neq0\) per a tot \(x\) de \((a,b)\), \(x_{0}\) un element de \((a,b)\) i \(t_{0}\) un real. Aleshores el sistema
		\begin{equation}
			\label{prop:problema de Cauchy:eq1}
			\begin{cases}
				\displaystyle \frac{\partial u(x)}{\partial x}=f(x) \\
				\displaystyle u(t_{0})=x_{0}
			\end{cases}
		\end{equation}
		té una única solució de la forma
		\[u(t)=F^{-1}(t-t_{0})\]
		on
		\[F(x)=\int_{x_{0}}^{x}\frac{\text{d}t}{f(t)}.\]
		\begin{proof}
			Observem que, per la definició de \myref{def:equació diferencial ordinària d'ordre n} l'equació
			\[\frac{\partial u(x)}{\partial x}=f(x)\]
			és una equació ordinària d'orde \(1\).
			
			Com que, per hipòtesi, tenim \(f(x)\neq0\) per a tot \(x\) de \((a,b)\) trobem que la funció \(\frac{1}{f(x)}\) és derivable %REF
			i per la proposició \myref{prop:Diferenciable implica contínua} trobem que \(\frac{1}{f(x)}\) és contínua. També tenim pel \myref{thm:Weierstrass màxims i mínims múltiples variables} que \(\frac{1}{f(x)}\) està acotada en \((a,b)\) i pel \myref{thm:Contínua + acotada implica integrable Riemann} trobem que \(\frac{1}{f(x)}\) és integrable en \((a,b)\). Definim doncs
			\[F(x)=\int_{x_{0}}^{x}\frac{\text{d}t}{f(t)}.\]
			Pel \myref{thm:Teorema Fonamental del Càlcul} tenim que
			\begin{equation}
				\label{prop:problema de Cauchy:eq2}
				\frac{\partial F(x)}{\partial x}=\frac{1}{f(x)},
			\end{equation}                
			i com que \(\frac{1}{f(x)}\neq0\) per a tot \(x\) de \((a,b)\) tenim que \(F(x)\) és invertible en \((a,b)\). Denotem amb \(F^{-1}\) la seva inversa.
			
			Ara bé, tenim que
			\begin{equation}
				\label{prop:problema de Cauchy:eq3}
				\frac{\partial F(u(x))}{\partial x}=1,
			\end{equation}
			ja que
			\begin{align*}
				\frac{\partial F(u(x))}{\partial x}&=\frac{\partial F(u(x))}{\partial u(x)}\frac{\partial u(x)}{\partial x}\tag{\myref{thm:regla de la cadena}}\\
				&=\frac{1}{f(u(x))}\frac{\partial u(x)}{\partial x}\tag{\ref{prop:problema de Cauchy:eq2}}\\
				&=\frac{1}{f(u(x))}f(u(x))=1.\tag{\ref{prop:problema de Cauchy:eq1}}\\
			\end{align*}
			I per tant, pel \myref{thm:Teorema Fonamental del Càlcul} trobem que
			\[\int_{t_{0}}^{t}\frac{\partial F(u(x))}{\partial x}\text{d}x=F(u(t))-F(u(t_{0})),\]
			i per \eqref{prop:problema de Cauchy:eq2} trobem \(F(u(t))-F(u(t_{0}))=t-t_{0}\)
			
			Ara bé, per hipòtesi, tenim que \(u(t_{0})=x_{0}\), i trobem
			\[F(x_{0})=\int_{x_{0}}^{x_{0}}\frac{1}{f(t)}\text{d}t=0.\]
			Per tant
			\[F(u(t))=t-t_{0},\]
			i com que, com ja hem vist, \(F(x)\) és invertible en \((F(a),F(b))\) trobem
			\[u(t)=F^{-1}(t-t_{0})\]
			per a tot \(t\) de \((t_{0}+F(a), t_{0}+F(b))\).
			
			Amb aquesta demostració també podem veure que aquesta és la única solució de l'equació diferencial.
		\end{proof}
	\end{proposition}
	\begin{example}
		\label{ex:edos problema de Cauchy 1}
		Sigui \(\alpha\) un real. Volem trobar una solució al sistema
		\begin{equation}
			\label{ex:edos problema de Cauchy 1:eq1}
			\begin{cases*}
				\frac{\partial u(x)}{\partial x}=\alpha x^{2} \\
				u(1)=1.
			\end{cases*}
		\end{equation}
		\begin{solution}
			Observem que, per la definició de \myref{def:equació diferencial ordinària d'ordre n} que l'equació \(\frac{\partial u(x)}{\partial x}=\alpha x^{2}\) és una equació diferencial ordinària d'odre \(1\). per la proposició \myref{prop:problema de Cauchy} trobem que les solucions de \eqref{ex:edos problema de Cauchy 1:eq1} són de la forma
			\[u(x)=F^{-1}(x-1)\]
			on
			\[F(x)=\int_{x_{0}}^{x}\frac{\text{d}t}{\alpha t^{2}}.\]
			
			Per tant, tenim que
			\[F(x)=\frac{x-1}{\alpha x}\]
			i trobem
			\[u(x)=\frac{1}{1-\alpha (x-1)}.\qedhere\]
		\end{solution}
	\end{example}
	\begin{proposition}[Equacions de variables separades]
		\labelname{}\label{prop:equacions de variables separades}
		Siguin \((a,b)\) i \((t_{1},t_{2})\) dos intervals oberts de \(\mathbb{R}\), \(f\colon(a,b)\longrightarrow\mathbb{R}\) i \(g\colon(t_{1},t_{2})\longrightarrow\mathbb{R}\) dues funcions contínues amb \(f(x)\neq0\) per a tot \(x\) de \((a,b)\), \(x_{0}\) un element de \((a,b)\) i \(t_{0}\) un element de \((t_{1},t_{2})\). Aleshores el problema de Cauchy
		\begin{equation}
			\begin{cases}
				\displaystyle \frac{\partial u(x)}{\partial x}=g(x)f(u(x)) \\
				\displaystyle u(t_{0})=x_{0}.
			\end{cases}
		\end{equation}
		té una única solució de la forma
		\[u(x)=F^{-1}\left(\int_{t_{0}}^{x}g(t)\text{d}t\right)\]
		on
		\[F(x)=\int_{x_{0}}^{x}\frac{\text{d}t}{f(t)}.\]
		\begin{proof}
			Com que, per hipòtesi, \(f(x)\neq0\) per a tot \(x\) de \((a,b)\) definim \(h=\frac{1}{f(u(x))}\) i observem que
			\[h(u(x))\frac{\partial u(x)}{\partial x}=g(x)\]
			i per tant
			\[\int h(u(x))\frac{\partial u(x)}{\partial x}\text{d}{x}=\int g(x)\text{d}x.\]
			
			Ara bé, per la \myref{thm:regla de la cadena} i el \myref{thm:Teorema Fonamental del Càlcul} tenim que això és
			\[H(x)=G(x)+C\]
			on \(C\) és un real i
			\[H(x)=\int_{t_{0}}^{x}h(t)\text{d}t\quad\text{i}\quad G(x)=\int_{t_{0}}^{x}g(t)\text{d}t\]
			i per tant
			\[H(u(x))=G(x)+C\]
%			Podem seguir la solució de la proposició \myref{prop:problema de Cauchy} per trobar que les úniques solucions són
%			\[u(x)=F^{-1}\left(\int_{t_{0}}^{x}g(t)\text{d}t\right)\]
%			on
%			\[F(x)=\int_{x_{0}}^{x}\frac{\text{d}t}{f(t)}.\qedhere\]
		\end{proof}
	\end{proposition}
	\begin{example}
		\label{ex:equacions de variables separades}
		Volem trobar una solució al sistema
		\begin{equation}
			\label{ex:equacions de variables separades:eq1}
			\begin{cases*}
				\displaystyle \frac{\partial u(x)}{\partial x}=\frac{u(x)^{3}x}{\sqrt{1+x^{2}}} \\
				u(0)=1.
			\end{cases*}
		\end{equation}
		\begin{solution}
			Definim
			\[f(u(x))=u(x)^{-3}\quad\text{i}\quad g(x)=\frac{x}{\sqrt{1+x^{2}}}.\]
			Per tant tenim que
			\[f(u(x))\frac{\partial u(x)}{\partial x}=g(x)\]
			
			Observem que
			\[u(x)=\int_{x_{0}}^{x}\frac{yu(x)}{\sqrt{1+y^{2}}}\text{d}x\]
			Per la proposició \myref{prop:equacions de variables separades} tenim que les solucions al sistema \eqref{ex:equacions de variables separades:eq1} són de la forma
			\[u(t)=F^{-1}\left(\int_{t_{0}}^ {t}\frac{y}{\sqrt{1+y^{2}}}\text{d}y\right)\]
			on
			\[F(x)=\int_{x_{0}}^{x}\frac{\text{d}t}{u(t)^{3}}.\]
			Per tant tenim que
			\[\]
		\end{solution}
	\end{example}
	\begin{proposition}[Equacions diferencials lineals]
		\labelname{}\label{prop:equacions diferencials lineals}
		Siguin \((t_{1},t_{2})\) un interval obert de \(\mathbb{R}\) i \(a\colon(t_{1},t_{2})\longrightarrow\mathbb{R}\), \(b\colon(t_{1},t_{2})\longrightarrow\mathbb{R}\) dues funcions contínues. Aleshores el problema de Cauchy
		\begin{equation}
			\label{prop:equacions diferencials lineals:eq1}
			\begin{cases}
				\displaystyle \frac{\partial u(x)}{\partial x}=a(t)u(x)+b(t)\\
				\displaystyle u(t_{0})=x_{0}.
			\end{cases}
		\end{equation}
		té una única solució de la forma
		\[u(x)=\left(x_{0}+\int_{t_{0}}^{t}b(s)e^{-\int_{t_{0}}^{s}a(\tau)\text{d}\tau}\text{d}s\right)e^{\int_{t_{0}}^{x}a(s)\text{d}s}.\]
		\begin{proof}
			Considerem el canvi de variable
			\begin{equation}
				\label{prop:equacions diferencials lineals:eq2}
				u(x)=c(x)e^{\int_{t_{0}}^{x}a(s)\text{d}s}.
			\end{equation}
			Aleshores pel \myref{thm:Teorema Fonamental del Càlcul} tenim %I ref de la derivada del producte
			\[\frac{\partial u(x)}{\partial x}=\frac{\partial c(x)}{\partial x}e^{\int_{t_{0}}^{x}a(s)\text{d}s}+c(x)a(x)e^{\int_{t_{0}}^{x}a(s)\text{d}s}\]
			i per \eqref{prop:equacions diferencials lineals:eq1} i \eqref{prop:equacions diferencials lineals:eq2} tenim
			\[a(t)u(x)+b(t)=\frac{\partial c(x)}{\partial x}e^{\int_{t_{0}}^{x}a(s)\text{d}s}+a(x)u(x).\]
			Per tant trobem
			\[\frac{\partial c(x)}{\partial x}=b(x)e^{\int_{t_{0}}^{x}a(s)\text{d}s}\]
			i tenim
			\begin{equation}
				\label{prop:equacions diferencials lineals:eq3}
				\begin{cases}
					\displaystyle \frac{\partial c(x)}{\partial x}=b(x)e^{\int_{t_{0}}^{x}a(s)\text{d}s}\\
					\displaystyle c(t_{0})=x_{0}.
				\end{cases}
			\end{equation}
			
			Aleshores per la proposició \myref{prop:problema de Cauchy} trobem que aquest problema té una única solució, i és de la forma
			\[c(t)=x_{0}+\int_{t_{0}}^{t}b(s)e^{-\int_{t_{0}}^{s}a(\tau)\text{d}\tau}\text{d}s\]
			i desfent el canvi de variables \eqref{prop:equacions diferencials lineals:eq2} trobem que les solucions són de la forma
			\[u(x)=\left(x_{0}+\int_{t_{0}}^{t}b(s)e^{-\int_{t_{0}}^{s}a(\tau)\text{d}\tau}\text{d}s\right)e^{\int_{t_{0}}^{x}a(s)\text{d}s}.\qedhere\]
		\end{proof}
	\end{proposition}
\end{document}

% http://issc.uj.ac.za/downloads/problems/ordinary.pdf