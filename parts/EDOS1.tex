\documentclass[../Apunts.tex]{subfiles}

\begin{document}
\chapter{Equacions diferencials de primer ordre en una variable}
	\section{Teoremes d'existència i unicitat}
	\subsection{Equacions diferencials ordinàries}
	\begin{definition}[Equació diferencial ordinària d'ordre \(n\)]
		\labelname{equació diferencial ordinària d'odre $n$}\label{def:equació diferencial ordinària d'ordre n}
		\labelname{solució de l'equació diferencial ordinària}\label{def:solució de l'equació diferencial ordinària}
		Siguin \(\Omega=\mathbb{R}\times\mathbb{R}^{n}\) un conjunt i \(F\colon\Omega\longrightarrow\mathbb{R}^{m}\), \(u\colon\mathbb{R}\longrightarrow\mathbb{R}^{m}\) dues funcions tals que
		\[\frac{\partial^{n}u(x)}{\partial x^{n}}=F\left(t,u(x),\frac{\partial u(x)}{\partial x},\dots,\frac{\partial^{n-1}u(x)}{\partial x^{n-1}}\right).\]
		Aleshores direm que
		és una equació diferencial ordinària. També direm que \(t\) és la variable temporal, \(u(x)\) la variable espacial.
	\end{definition}
	\begin{example}
		\label{ex:edos1:1}
		Considerem l'equació diferencial ordinària d'ordre \(1\) següent:
		\begin{equation}
			\label{edos1:ex1:eq1}
			\frac{\partial u(x)}{\partial x}=f(u(x)).
		\end{equation}
		on \(I\) és un interval de \(\mathbb{R}\) i \(f\colon I\longrightarrow\mathbb{R}^{m}\) una funció contínua.
		\begin{solution}
			Observem primer que pel Teorema \myref{thm:Weierstrass màxims i mínims múltiples variables} la funció \(f(x)\) està acotada en \(I\), i pel Teorema \myref{thm:Contínua + acotada implica integrable Riemann} tenim que la funció \(f(x)\) és integrable Riemann, i per tant, pel \myref{thm:Teorema Fonamental del Càlcul}, trobem que per a tot \(x_{0}\) de \(I\)
			\[u(x)=\int_{x_{0}}^{x}f(u(t))dt\]
			és una solució de l'equació diferencial ordinària \eqref{edos1:ex1:eq1}.
		\end{solution}
	\end{example}
	\begin{observation}
		Observem que, en l'exemple \myref{ex:edos1:1}, si la funció \(f\) no és integrable en \(I\) aleshores l'equació diferencial \eqref{edos1:ex1:eq1} no té solució. %Integrable Riemann?
	\end{observation}
	\begin{example}[Problema de Cauchy]
		\label{ex:Problema de Cauchy}
		Siguin \((a,b)\) un interval obert de \(\mathbb{R}\), \(f\colon(a,b)\longrightarrow\mathbb{R}\) una funció contínua tal que \(f(x)\neq0\) per a tot \(x\) de \((a,b)\), \(x_{0}\) un element de \((a,b)\) i \(t_{0}\) un real. Considerem el sistema
		\begin{equation}
			\label{ex:Problema de Cauchy:eq1}
			\begin{cases}
				\displaystyle \frac{\partial u(x)}{\partial x}=f(x) \\
				\displaystyle u(t_{0})=x_{0}
			\end{cases}.
		\end{equation}
		Volem trobar totes les solucions a aquest sistema.
		\begin{solution}
			Observem que, per la definició de \myref{def:equació diferencial ordinària d'ordre n} l'equació
			\[\frac{\partial u(x)}{\partial x}=f(x)\]
			és una equació ordinària d'orde \(1\).
			
			Com que, per hipòtesi, tenim \(f(x)\neq0\) per a tot \(x\) de \((a,b)\) trobem que la funció \(\frac{1}{f(x)}\) és derivable %REF
			i per la proposició \myref{prop:Diferenciable implica contínua} trobem que \(\frac{1}{f(x)}\) és contínua. També tenim pel \myref{thm:Weierstrass màxims i mínims múltiples variables} que \(\frac{1}{f(x)}\) està acotada en \((a,b)\) i pel \myref{thm:Contínua + acotada implica integrable Riemann} trobem que \(\frac{1}{f(x)}\) és integrable en \((a,b)\). Definim doncs
			\[F(x)=\int_{x_{0}}^{x}\frac{dt}{f(t)}.\]
			Pel \myref{thm:Teorema Fonamental del Càlcul} tenim que
			\begin{equation}
				\label{ex:problema de Cauchy:eq1}
				\frac{\partial F(x)}{\partial x}=\frac{1}{f(x)},
			\end{equation}                
			i com que \(\frac{1}{f(x)}\neq0\) per a tot \(x\) de \((a,b)\) tenim que \(F(x)\) és invertible en \((a,b)\). Denotem amb \(F^{-1}\) la seva inversa.
			
			Ara bé, tenim que
			\begin{equation}
				\label{ex:problema de Cauchy:eq2}
				\frac{\partial F(u(x))}{\partial x}=1,
			\end{equation}
			ja que
			\begin{align*}
				\frac{\partial F(u(x))}{\partial x}&=\frac{\partial F(u(x))}{\partial u(x)}\frac{\partial u(x)}{\partial x}\tag{\myref{thm:regla de la cadena}}\\
				&=\frac{1}{f(u(x))}\frac{\partial u(x)}{\partial x}\tag{\ref{ex:problema de Cauchy:eq1}}\\
				&=\frac{1}{f(u(x))}f(u(x))=1.\tag{\ref{ex:Problema de Cauchy:eq1}}\\
			\end{align*}
			I per tant, pel \myref{thm:Teorema Fonamental del Càlcul} trobem que
			\[\int_{t_{0}}^{t}\frac{\partial F(u(x))}{\partial x}dx=F(u(t))-F(u(t_{0})),\]
			i per \eqref{ex:problema de Cauchy:eq2} trobem \(F(u(t))-F(u(t_{0}))=t-t_{0}\)
			
			Ara bé, per hipòtesi, tenim que \(u(t_{0})=x_{0}\), i trobem
			\[F(x_{0})=\int_{x_{0}}^{x_{0}}\frac{1}{f(t)}dt=0.\]
			Per tant
			\[F(u(t))=t-t_{0},\]
			i com que, com ja hem vist, \(F(x)\) és invertible en \((F(a),F(b))\) trobem
			\[u(t)=F^{-1}(t-t_{0})\]
			per a tot \(t\) de \((t_{0}+F(a), t_{0}+F(b))\).
			
			Amb aquesta solució també podem veure que aquesta és la única solució de l'equació diferencial.
		\end{solution}
	\end{example}
	\begin{example}[Equacions de variables separades]
		\label{ex:Equacions de variables separades}
		Siguin \((a,b)\) i \((t_{1},t_{2}\) dos intervals oberts de \(\mathbb{R}\), \(f\colon(a,b)\longrightarrow\mathbb{R}\) i \(g\colon(t_{1},t_{2})\longrightarrow\mathbb{R}\) dues funcions contínues amb \(f(x)\neq0\) per a tot \(x\) de \((a,b)\), \(x_{0}\) un element de \((a,b)\) i \(t_{0}\) un element de \((t_{1},t_{2})\). Considerem el problema de Cauchy
		\begin{equation}
			\begin{cases}
				\displaystyle \frac{\partial u(x)}{\partial x}=g(t)f(u(x)) \\
				\displaystyle u(t_{0})=x_{0}.
			\end{cases}
		\end{equation}
		Volem trobar les solucions d'aquest problema de Cauchy.
		\begin{solution}
			Podem seguir la solució de l'exemple \myref{ex:Problema de Cauchy} per trobar que les úniques solucions són
			\[u(t)=F^{-1}\left(\int_{t_{0}}{t}g(x)dx\right)\]
			on
			\[F(x)=\int_{x_{0}}^{x}\frac{dt}{f(t)}.\]
		\end{solution}
	\end{example}
	\begin{example}[Equacions diferencials lineals]
		\label{ex:Equacions diferencials lineals}
		:D
	\end{example}
\end{document}