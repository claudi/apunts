\documentclass[../Apunts.tex]{subfiles}

\begin{document}
\part{Probabilitat i modelització estocàstica}
\chapter{Models probabilístics}
\section{El model probabilístic}
\subsection{Experiments aleatoris i espai mostral}
	\begin{definition}[Experiment aleatori]
		\labelname{experiment aleatori}\label{def:experiment aleatori}
		\labelname{espai mostral}\label{def:espai mostral}
		Definirem de manera informal els fenòmens que estudiarem, els \emph{experiments aleatoris}. Aquests tenen les següents propietats:
		\begin{enumerate}
			\item Coneixem tots els possibles resultats de l'experiment, però no el que sortirà. El conjunt de possibles resultats~\(\Omega\) serà l'\emph{espai mostral}.
			\item Tenim alguna manera d'assignar probabilitats als resultats, o a conjunts de resultats.
		\end{enumerate}
	\end{definition}
	\begin{example}
		\label{ex:espais mostrals}
		Volem determinar l'espai mostral dels següents experiments:
		\begin{enumerate}
			\item\label{ex:espais mostrals:eq1} Llençar un dau de sis cares.
			\item\label{ex:espais mostrals:eq2} Un partit de bàsquet.
			\item\label{ex:espais mostrals:eq3} Llençar una moneda.
		\end{enumerate}
		\begin{solution}
			Així, els seus espais mostrals són, respectivament,
			\begin{enumerate}
				\item \(\Omega=\{1,2,3,4,5,6\}\).
				\item \(\Omega=\NN\times\NN\).
				\item \(\Omega=\{\heads,\tails\}\).\qedhere
			\end{enumerate}
			%TODO (?)
		\end{solution}
	\end{example}
	\begin{definition}[Esdeveniment]
		\labelname{esdeveniment}\label{def:esdeveniment}
		\labelname{esdeveniment segur}\label{def:esdeveniment segur}
		\labelname{esdeveniment contrari}\label{def:esdeveniment contrari}
		\labelname{esdeveniment impossible}\label{def:esdeveniment impossible}
		Sigui~\(A\subseteq\Omega\) un subconjunt d'un espai mostral~\(\Omega\). Aleshores direm que~\(A\) és un esdeveniment.
		
		Si el resultat~\(\omega\in A\subseteq\Omega\) s'ha realitzat direm que l'esdeveniment~\(A\) s'ha realitzat. També definim
		\begin{enumerate}
			\item Direm que~\(A=\Omega\) és l'esdeveniment segur i~\(A=\emptyset\) és l'esdeveniment impossible.
			\item Direm que~\(B=\Omega\setminus A\) és l'esdeveniment contrari a l'esdeveniment~\(A\).
%			\item Direm que~\(A=\emptyset\) és l'esdeveniment impossible.
		\end{enumerate}
	\end{definition}
	\begin{example}
		Volem determinar els conjunts d'esdeveniments següents:
		\begin{enumerate}
			\item Tirar un dau de sis cares i que surti un nombre parell.
			\item Una partit de bàsquet on guanya el visitant.
			\item Un món on acabo aquests apunts abans que la carrera.
		\end{enumerate}
		\begin{solution}
			Els conjunts d'esdeveniments són, respectivament,
			\begin{enumerate}
				\item \(A=\{2,4,6\}\).
				\item \(A=\{(m,n)\in\NN\times\NN\mid n>m\}\).
				\item \(A=\emptyset\).\qedhere
			\end{enumerate}
			%TODO (?)
		\end{solution}
	\end{example}
\subsection{Àlgebres i \ensuremath{\sigma}-àlgebres}
	\begin{definition}[Àlgebra]
		\labelname{àlgebra}\label{def:àlgebra}
		Sigui~\(\algebra{A}\) una co{\lgem}ecció de subconjunts~\(\Omega\) tal que
		\begin{enumerate}
			\item \(\Omega\in\algebra{A}\).
			\item Si~\(A\in\algebra{A}\), aleshores~\(\Omega\setminus A\in\algebra{A}\).
			\item Si~\(A\), \(B\in\algebra{A}\), aleshores~\(A\cup B\in\algebra{A}\).
		\end{enumerate}
		Aleshores direm que~\(\algebra{A}\) és un àlgebra de conjunts sobre~\(\Omega\).
	\end{definition}
	\begin{observation}
		\label{obs:el conjunt buit pertany a qualsevol àlgebra}
		Si~\(\algebra{A}\) és un àlgebra, aleshores~\(\emptyset\in\algebra{A}\).
	\end{observation}
	\begin{example}
		\label{ex:exemple d'àlgebra}
		Sigui~\(\Omega=\{1,2,\dots,6\}\). Volem veure que
		\[\algebra{A}=\{\emptyset,\Omega,\{1,2\},\{3,4,5,6\}\}\]
		és un àlgebra.
		\begin{solution}
			Comprovem la definició~d'\myref{def:àlgebra}. Tenim que~\(\Omega\in\algebra{A}\). Prenem un esdeveniment~\(A\in\algebra{A}\) i calculem~\(\Omega\setminus A\). Podem veure tots els casos fent
			\[
				\Omega\setminus\emptyset=\Omega,\qquad
				\Omega\setminus\Omega=\emptyset,\qquad
				\Omega\setminus\{1,2\}=\{3,4,5,6\},\quad\text{i}\quad
				\Omega\setminus\{3,4,5,6\}=\{1,2\}.
			\]
			
			Prenen ara~\(A\),~\(B\in\algebra{A}\). Si~\(A=B\) aleshores~\(A\cup B=A\), i per tant~\(A\cup B\in\algebra{A}\), i si tenim~\(A\cup B\in\algebra{A}\) aleshores~\(B\cup A=A\cup B\in\algebra{A}\) i hem acabat. També tenim que si~\(A\in\algebra{A}\), aleshores~\(\emptyset\cup A=A\in\algebra{A}\) i~\(\Omega\cup A=\Omega\in\algebra{A}\). Per tant només ens cal veure el cas
			\[
				\{1,2\}\cup\{3,4,5,6\}=\{1,2,3,4,5,6\}=\Omega\in\algebra{A},
			\]
			i per la definició~d'\myref{def:àlgebra} trobem que~\(\algebra{A}\) és un àlgebra sobre~\(\Omega\)
		\end{solution}
	\end{example}
	\begin{proposition}
		\label{prop:les àlgebras són tancades per interseccions}
		Siguin~\(A\),~\(B\) dos elements d'un àlgebra~\(\algebra{A}\). Aleshores
		\[A\cap B\in\algebra{A}.\]
		\begin{proof}
			Per la definició~d'\myref{def:àlgebra} tenim que~\(A^{\complement}\),~\(B^{\complement}\in\algebra{A}\) i, de nou per la definició~d'\myref{def:àlgebra}, tenim que~\(A^{\complement}\cup B^{\complement}\in\algebra{A}\). Ara bé,%REF
			tenim que~\(A^{\complement}\cup B^{\complement}=A\cap B\), i per tant~\(A\cap B\in\algebra{A}\).
		\end{proof}
	\end{proposition}
	\begin{definition}[\(\sigma\)-àlgebra]
		\labelname{\ensuremath{\sigma}-àlgebra}\label{def:sigma àlgebra}
		Sigui~\(\Algebra{A}\) un àlgebra tal que per a tota família~\(\{A_{n}\}_{n\in\NN}\) d'elements~d'\(\Algebra{A}\) tenim que
		\[\bigcup_{n\in\NN}A_{n}\in\Algebra{A}.\]
		Aleshores direm que~\(\Algebra{A}\) és una~\(\sigma\)-àlgebra.
	\end{definition}
	\begin{observation}
		Sigui~\(\Algebra{A}\) un àlgebra sobre un conjunt~\(\Omega\) finit. Aleshores~\(\Algebra{A}\) és una~\(\sigma\)-àlgebra.
	\end{observation}
	\begin{proposition}
		\label{prop:les sigma àlgebras conserven interseccions numerables}
		Sigui~\(\{A_{n}\}_{n\in\NN}\) una família d'elements d'un~\(\sigma\)-àlgebra~\(\Algebra{A}\). Aleshores
		\[A=\bigcap_{n\in\NN}A_{n}\]
		és un element de~\(\Algebra{A}\).
		\begin{proof}
			Per la definició~d'\myref{def:àlgebra} tenim que~\(A_{n}^{\complement}\in\Algebra{A}\) per a tot~\(n\in\NN\). %TODO
		\end{proof}
	\end{proposition}
\subsection{L'àlgebra de Borel}
	\begin{proposition}
		\label{prop:existeix una sigma àlgebra mínima}
		Sigui~\(\{\Algebra{A}_{i}\}_{i\in I}\) una família de~\(\sigma\)-àlgebres sobre un conjunt~\(\Omega\). Aleshores
		\[\Algebra{A}=\bigcap_{i\in I}\Algebra{A}_{i}\]
		és una~\(\sigma\)-àlgebra sobre~\(\Omega\).
		\begin{proof}
			Comprovem que~\(\Algebra{A}\) satisfà la definició~\myref{def:sigma àlgebra}. Veiem que~\(\Omega\in\Algebra{A}\). Tenim per la definició de~\myref{def:àlgebra} que~\(\Omega\in\Algebra{A}_{i}\) per a tot~\(i\in I\), i per la definició d'\myref{def:intersecció de conjunts} trobem que~\(\Omega\in\Algebra{A}\).
			
			Prenem ara un~\(A\in\Algebra{A}\) i considerem~\(A^{\complement}\). Si~\(A\in\Algebra{A}\) tenim que per a tot~\(i\in I\) es satisfà~\(A\in\Algebra{A}_{i}\), i per la definició~d'\myref{def:àlgebra} trobem que~\(A^{\complement}\in\Algebra{A}_{i}\) per a tot~\(i\in I\), i per la definició d'\myref{def:intersecció de conjunts} tenim que es satisfà~\(A^{\complement}\in\Algebra{A}\).
			
			Per acabar premen~\(A\),~\(B\in\Algebra{A}\). Tenim que~\(A\),~\(B\in\Algebra{A}_{i}\) per a tot~\(i\in I\), i per la definició~d'\myref{def:àlgebra} trobem que~\(A\cup B\in\Algebra{A}_{i}\) per a tot~\(i\in I\), i per la definició~d'\myref{def:intersecció de conjunts} trobem que~\(A\cup B\in\Algebra{A}\).
			
			Per tant per la definició~d'\myref{def:àlgebra} tenim que~\(\Algebra{A}\) és un àlgebra sobre~\(\Omega\).
			
			Prenem ara una família~\(\{A_{n}\}_{n\in\mathbb{N}}\) d'elements~d'\(\Algebra{A}\) i denotem
			\[A=\bigcup_{n\in\NN}A_{n}.\]
			Tenim de nou que~\(\{A_{n}\}_{n\in\mathbb{N}}\) també és una família d'elements~d'\(\Algebra{A}_{i}\) per a tot~\(i\in I\), i per la definició~d'\myref{def:sigma àlgebra} tenim que~\(A\in\Algebra{A}_{i}\) per a tot~\(i\in I\), i per la definició~d'\myref{def:intersecció de conjunts} trobem que~\(A\in\Algebra{A}\), i per la definició~d'\myref{def:sigma àlgebra} trobem que~\(\Algebra{A}\) és una~\(\sigma\)-àlgebra sobre~\(\Omega\).
		\end{proof}
	\end{proposition}
	\begin{definition}[\ensuremath{\sigma}-àlgebra generada per un conjunt]
		\labelname{\ensuremath{\sigma}-àlgebra generada per un conjunt}\label{def:sigma àlgebra generada per un conjunt}
		Sigui~\(\Algebra{C}\subseteq\parts(\Omega)\) una família de subconjunts d'un conjunt~\(\Omega\). Denotarem el mínim~\(\sigma\)-àlgebra generada per~\(\Algebra{C}\) com~\(\sigma(\Algebra{C})\) i direm que és la~\(\sigma\)-àlgebra generada per~\(\Algebra{C}\).
		
		Aquesta definició té sentit per la proposició~\myref{prop:existeix una sigma àlgebra mínima}.
	\end{definition}
	\begin{example}[\ensuremath{\sigma}-àlgebra de singletons]
		\labelname{\ensuremath{\sigma}-àlgebra de singletons}\label{ex:sigma àlgebra de singletons}
		Denotem~\(\Algebra{C}=\{\{x\}\subseteq\RR\mid x\in\RR\}\). Volem veure que
		\[\sigma(\Algebra{C})=\{A\subseteq\RR\mid A\text{ ó }A^{\complement}\text{ és finit o numerable}\}.\]
		\begin{solution}
			%TODO % Fer a topo i sorprendre als nens amb això. O al revès potser molaria més. Fer aquí i referenciar a topo com un boss. Per allà on axiomes de separació pot ser útil.
		\end{solution}
	\end{example}
	\begin{definition}[\(\sigma\)-àlgebra de Borel]
		\labelname{\ensuremath{\sigma}-àlgebra}\label{def:sigma àlgebra de Borel}
		Sigui
		\[
			\borel(\RR)=\{\obert{U}\subseteq\RR\mid\obert{U}\text{ és un obert de }\RR\}.
		\]
		Direm que~\(\borel(\RR)\) és la~\(\sigma\)-àlgebra de Borel.
	\end{definition}
\end{document}
