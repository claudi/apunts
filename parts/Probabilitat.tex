\documentclass[../Apunts.tex]{subfiles}

\begin{document}
\part{Probabilitat i modelització estocàstica}
\chapter{Models probabilístics}
\section{El model probabilístic}
	\subsection{Experiments aleatoris i espai mostral}
	\begin{definition}[Experiment aleatori]
		\labelname{experiment aleatori}\label{def:experiment aleatori}
		\labelname{espai mostral}\label{def:espai mostral}
		Definirem de manera informal els fenòmens que estudiarem, els \emph{experiments aleatoris}. Aquests tenen les següents propietats:
		\begin{enumerate}
			\item Coneixem tots els possibles resultats de l'experiment, però no el que sortirà. El conjunt de possibles resultats~\(\Omega\) serà l'\emph{espai mostral}.
			\item Tenim alguna manera d'assignar probabilitats als resultats, o a conjunts de resultats.
		\end{enumerate}
	\end{definition}
	\begin{example}
		Volem determinar l'espai mostral dels següents experiments:
		\begin{enumerate}
			\item Tirar un dau de sis cares.
			\item Un partit de bàsquet.
			\item Tirar una moneda.
		\end{enumerate}
		\begin{solution}
			Els seus espais mostrals són, respectivament,
			\begin{enumerate}
				\item \(\Omega=\{1,2,3,4,5,6\}\).
				\item \(\Omega=\NN\times\NN\).
				\item \(\Omega=\{\heads,\tails\}\).\qedhere
			\end{enumerate}
		\end{solution}
	\end{example}
	\begin{definition}[Esdeveniment]
		\labelname{esdeveniment}\label{def:esdeveniment}
		\labelname{esdeveniment segur}\label{def:esdeveniment segur}
		\labelname{esdeveniment contrari}\label{def:esdeveniment contrari}
		\labelname{esdeveniment impossible}\label{def:esdeveniment impossible}
		Sigui~\(A\subseteq\Omega\) un subconjunt d'un espai mostral~\(\Omega\). Aleshores direm que~\(A\) és un esdeveniment.
		
		Si el resultat~\(\omega\in A\subseteq\Omega\) s'ha realitzat direm que l'esdeveniment~\(A\) s'ha realitzat. També definim
		\begin{enumerate}
			\item Direm que~\(A=\Omega\) és l'esdeveniment segur i~\(A=\emptyset\) és l'esdeveniment impossible.
			\item Direm que~\(B=\Omega\setminus A\) és l'esdeveniment contrari a l'esdeveniment~\(A\).
%			\item Direm que~\(A=\emptyset\) és l'esdeveniment impossible.
		\end{enumerate}
	\end{definition}
	\begin{example}
		Volem determinar els conjunts d'esdeveniments següents:
		\begin{enumerate}
			\item Tirar un dau de sis cares i que surti un nombre parell.
			\item Una partit de bàsquet on guanya el visitant.
			\item Un món on acabo aquests apunts abans que la carrera.
		\end{enumerate}
		\begin{solution}
			Els conjunts d'esdeveniments són, respectivament,
			\begin{enumerate}
				\item \(A=\{2,4,6\).
				\item \(A=\{(m,n)\in\NN\times\NN\mid n>m\}\).
				\item \(A=\emptyset\).\qedhere
			\end{enumerate}
		\end{solution}
	\end{example}
	\subsection{Àlgebres i \(\sigma\)-àlgebres}
	\begin{definition}[Àlgebra]
		\labelname{àlgebra}\label{def:àlgebra}
		Sigui~\(\algebra{A}\) una co{\lgem}ecció de subconjunts~\(\Omega\) tal que
		\begin{enumerate}
			\item \(\Omega\in\algebra{A}\).
			\item Si~\(A\in\algebra{A}\), aleshores~\(\Omega\setminus A\in\algebra{A}\).
			\item Si~\(A\), \(B\in\algebra{A}\), aleshores~\(A\cup B\in\algebra{A}\).
		\end{enumerate}
		Aleshores direm que~\(\algebra{A}\) és un àlgebra de conjunts sobre~\(\Omega\).
	\end{definition}
	\begin{observation}
		\label{obs:el conjunt buit pertany a qualsevol àlgebra}
		Si~\(\algebra{A}\) és un àlgebra, aleshores~\(\emptyset\in\algebra{A}\).
	\end{observation}
\end{document}
